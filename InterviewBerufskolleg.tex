In welchem Umfang besitzen Sie Vorerfahrungen mit Schüler Online 1.0? 	
Keine Erfahrung					

In welchem Umfang besitzen Sie Vorerfahrungen mit der neuen Software?	
Keine Erfahrung		










Welche Probleme können bei herkömmlichen, nicht-digitalen Bewerbungen von Schülern auftreten?	
Nicht lesbar, nicht vollständig, nicht unterschrieben				








Wer nutzt das System hauptsächlich an Ihrer Schule?	
Schülerbüro, Verwaltung, Lehrer










Welche fachlichen und technischen Qualifikationen sind zur Bewältigung der Aufgabe erforderlich (Aufgabenbewältigung / Softwarenutzung)?
Welche Vorkenntnisse fehlen ggf.?	
Schulungen bzgl der Schulverwaltung
Man muss wissen was es für Bildungsgänge an der eigenen Schule gibt
Vorkenntnisse bzgl. der Schullaufbahn (Qualifikationen, BIldungsgänge)








Beschreiben Sie die Ausgangssituation die vorliegt, bevor Sie die Aufgabe "Bewerbung eines Schülers" durchführen.	
Einschulungsdaten in OSA eintragen und das muss dann in Magellan eingetragen
Es liegt ein langer Prozess von Bewerbung des Schülers bis hin zu abrufbaren Daten im Schuvlerwaltungsprogramm vor
	
















Zu Beginn der Durchführung:					


Studienteilnehmer ist durch viele menüpunkte gegangen aber hat sich dann für den Menüpunkt Anmeldung entschieden
Der kleine Bildschirm sorgt dafür, dass die Suchparameter nicht erschienen sind, ein rauszoomen war nötig. 
Auf Rückfrage hätte sich die Sekretärin für Abbruch der Verwendung oder Kontakt der Hotline entschieden. 		

Welche Arbeitsschritte sind durchzuführen?		
Woher bekomme ich die Daten? Papierform
Ich gebe die Daten ein, in der hoffnung alle Daten zu haben
Was passiert, wenn ich nicht alle daten habe, meine Hoffnung ist, dass die Daten schon vorgepflegt sind von den anderen Schulen				






Welche Hilfsmittel sind erforderlich (für die Aufgabenbewältigung / zur Softwarenutzung)? Welche davon fehlen ggf., welche sind zusätzlich gewünscht?
Laptops, 
Learning by doing, 
kein Hexenwerk, 
eine Kurzanleitung für den Einstieg	 	














Welche Ergebnisse / Teilergebnisse entstehen und wie werden diese ggf. verwertet / weitergeführt?		
- Anforderung, dass alles einfach und simpel ist
- Masken sollten erklärt sind
- Alle daten sollten drin sind, die ich brauche
- Dass Daten vorerfasst sind
- vorige Schule sollte erfasst werden, das ist sonst sehr aufwändig - für uns herauszufinden
- Hundekläffen im hintergrund			






Welche wichtigen Sonderfälle müssen berücksichtigt werden? (bzw. fallen dem Benutzer spontan ein; z. B. zur Arbeitsteilung / Zusammenarbeit)		
Förderbedarf, 
Männlich/weiblich (Schüler wissen in dem Alter nicht, was sie sind => Divers),
Fahrkarten (Tickets),
Wenn nicht volljährig sind Daten der Eltern wichtig
"Alles was bei uns drauf steht, ist wichtig"
Auf Bildungsgänge bewerben, halbes Jahr um Voraussetzungen zu sammeln
Wenige Schulkinder wissen hier genau, welchen Bildungsgang sie wollen
		






Während der Durchführung:						
Schüler-Tab
"Was ist denn Id-Schlüssel?"				





















Bildungsgang-Tab		
Diskussion, ob das jetzt das neue Schuljahr ist, oder das Schuljahr, welches der Schüler vorher besucht hat
Einschulungstermin bedeutet, dass der Schüler angenommen wird?
Es war nicht klar, wie der Status funktioniert -> erklärtexte wären hilfreich
Es war nicht klar, wer diese Maske gerade ausfüllen soll
Wir haben eine bestimmte Anzahl an Schulkindern pro Klasse (Bildungsgang) die wir aufnehmen können
Eine Information über den Masernschutz fehlt noch
Schüler wurde zunächst auf Warsteliste gesetzt
Aufnahmestatus ist schwierig, die Lehrer entscheiden über den Status"	"Auto im Hintergrund?
Unterbrechung durch Klopfen im hintergrund
darf ich den hier überhaupt schon setzen? (?)













































Persönliche Daten
Aussage: Sorgeberechtigte kommen auf dem nächsten Reiter
Der Tab ist selbsterklärend 
Hilfreich wäre eine Visualisierung, ob das schulkind volljährig ist, dann muss man nicht selber rechnen


















Sorgeberechtigte
Warum springt die maske?
warum sieht die anders aus
Anschriften-Reiter wird übersehen
wer sagt uns, dass die Daten hier richtig sind?
Mir gefällt das hier so nicht (Nicht übersichtlich und unattraktiv.)
Wozu dient das Postfach?
Das ist nicht ausreichend
Sollte genau so aussehen wie beim Schulkind
Tel, Adresse, Mail, ggf. Beruf + dienstliche Tel, Herkunftsland der Sorgeberechtigten sind wichtig für die Statistik (unabhängig von herkunft des Schulkinds)



























Notfallkontakte
Aber nur, wenn Sorgeberechtigte nicht gepflegt sind, (?)
Im normallfall sind das dieselben Personen wie die Sorgeberechtigten



















Migrationshintergrund		
Ergibt sich aus der staatsangehörigkeit
Das hinterfragt der Klaus immer, ob ein Migrationshintergrund vorliegt
Migrationshintergrund liegt nur vor, wenn Staatsangehörigkeit..." (Die Teilnehmerin fängt an zu spekulieren) - da müsste man das Gesetz kennen			




Letzte Tätigkeit
Letzte Tätigkeit ist kontrovers zu angestrebte Schulstufe
Die korrekte Schule nicht gefunden, es wird alternativ eine Realschule ausgewählt
Die Sekretärin hat die Checkbox "Schule nicht gefunden?" gesehen, aber nicht verstadnen
Doch noch die checkbox verstanden und verwendet
Suchfunktion funktioniert nicht gut, obwohl die Schule im System vorliegt. Es wurde zwar nach dem Ort gesucht, aber nicht die Schule gefunden.
Der Hund unterbricht mal wieder und braucht aufmerksamkeit
Schullaufbahn als Tabüberschrift
Grundschulempfehlung kann nicht mehr abgewählt werden, die Sekretärin hatte diese Information nicht vorliegend, aber dennoch ausgefüllt. 
Laute Geräusche im Hintergrund
Eingaben nicht vollständig
Die Sekretärin äußert zum wiederholten Male, dass Schulkinder auch diesen Dialog ausfüllen werden und Schwierigkeiten daran haben werden

Qualifikationen
Er zeigt nicht alle Abschlüsse an, da fehlen welche. Hauptschule fehlt bspw.

Termine
Termin, wofür?
Man hat keine zeit hier 500 Leute einzuladen, ist für uns nicht machbar
Nur bestimmte Bildungsangebote machen das an unserer Schule wir (als Sekretärin) wollen kein gespräch (lacht).

Bemerkungen
Ach, das geht hier noch weiter (Sekretärin hatte nicht die versteckten Reiter gesehen, da weitere Tabs hinter einem > Symbol versteckt waren)
Maximal für den Einschulungstag brauchen wir diese Felder
Interpretation: Eins ist nur für uns sichtbar, das andere für schüler/betrieb. Hoffentlich!

Zusammenfassung
Muss die Zusammenfassung sein? 
Auf die Frage inwieweit die Seite unnötig ist kommt die Antwort: Unnötig zu viele Infos in der zweiten Zeile
wir fokussieren uns nur auf den Bildungsgang.
Eigene bezeichnungen für die Bildungsgänge wären wünschenswert
Klasse muss hier noch nicht auftauchen
Die Kürzel müssen da nicht stehen, da kann keiner was mit anfangen

Status
Gibt's schüler, die noch keine mail haben?
Interpretation der Seite: Die bekommen den Status der anmeldung mitgeteilt, was gut ist. 

Liste Anmeldungen:
Es ist gut zu sehen, wer das schulkind angelegt hat

Update Dialog:
Der Speichern-button wird nicht als global verstanden
Es wird nicht erkannt, dass 
Wenn man drüber nachdenkt, macht es sinn, dass der Betrieb nicht auswählbar ist
Ein Zwischenspeichern ist gewünscht
Tabs bei sorgeberechtigten-popup farblich schlecht, nicht in grautönen

Abschluss wird geprüft und ggf. korrigiert, nur dass kann ohne weitere informationen gemacht werden
ich würde mir wünschen, dass an dem tab steht, wo der fehler ist
Übersicht der Anmeldungen gefällt, aber filter/Sortierung wurde nicht gefunden, Sortierung mit Sub-Sortierungen gewünscht
anschrift/postfach unnötig
die roten felder sind pflicht, oder? ist das irgendwo erklärt? die sind mir bisher nicht aufgefallen
Der rote rahmen ist gut (bei Fehlern)
Grundsätzlich relativ übersichtlich, bis auf farbliche absetzung
für mich einfacher,
masken alle relativ gleich
es braucht zeit sich in eine andere Form (sorgeberchtigten vs andere tabs) hineinzudenken
Ablauf mit anderen Produkten unklar
Lehrer in die entscheidung mit einbeziehen, aber nicht zwangsläufig schreiberechte






Nach der Durchführung:						
Konnten Sie die Aufgabe aus Ihrer Sicht erfolgreich und vollständig abschließen? Falls nein was hat Sie daran gehindert?
Wenn wir nichts in der Hand haben, was sollen wir dann ändern?
Wir sind nicht Zuständigkeit für eine Aufnahmeentscheidung, nur die Aufnahme der Daten
Der Abteilungsleiter trifft entscheidung, ob der Schüler aufgenommen wird.












Wie effektiv unterstützt die Webanwendung Sie bei der Aufgabe? Gab es positive oder negative Erfahrungen?
Das programm so ist ja nicht schwierig
Abschlüsse besser erklären
Wenn man das einmal verstanden hat, weiß man, worauf man achten soll
Überschaubar
Infos von den Schülern, welche Klasse/Schule besucht wurde, wird nicht benötigt









Haben Sie sämtliche Inhalte der Aufgabe verstanden? Gab es Stellen, an denen Sie sich mehr Unterstützung gewünscht hätten?	
Das war doch jetzt verständlich alles
für uns ist das ja alles klar, als erfahrene Sekretärinnen
(im gegensatz zu Anfängern)

Gab es Schwierigkeiten oder Verwirrungen bei der Aufgabe? Wenn ja, welche?
Bezeichnungen sollten besser sein (Bisherige Schullaufbahn bspw. anstatt Letzte Tätigkeit)
Übersicht simpler (?)












Wie verständlich waren die Rückmeldungen der Anwendung?		
Manchmal wäre ne kleine erklärung daneben gut (vorbildung...)				










Welche Fähigkeiten setzt die Anwendung Ihrer Einschätzung nach voraus?
Lesen
Wissen über angebote der eigenen Schule, den normalen schulwerdegang

Wie sehr entspricht die Umsetzung in der Software der Realität? 	
Tatsächlich genau das gleich nur in einer anderen maske, 
sieht nur anders aus ist, anders aufgebaut
Anmeldezeiträume werden hoffentlich nciht mehr vergessen	











Was handhaben Sie in Ihrem Arbeitsalltag bei nicht-digitalen Bewerbungen gewöhnlicherweise anders als in der Anwendung?
Dokumente dazufügen, Zeugnis, Lebenslauf, masernschutz, damit die Lehrer eine Entscheidung treffen können


















Was würde Sie noch daran hindern die Software in Ihrem Arbeitsalltag einzusetzen?
Nach unserem ersten Eindruck nichts
Mit dem neuen \textit{SchILD} wird die Anmeldung ja sehr einfach


















Gibt es Funktionen, die Sie in ähnlichen bzw. anderen Anwendungen genutzt haben, die Sie hier vermissen?
Ne, ich kann alles eintragen
Farben, reiter oben in grün, wenn vollständig ausgefüllt			





Welche Software sollte man aus Ihrer Sicht in Schüler Online integrieren bzw. eine Schnittstelle schaffen? 
Excel-Export (auch, damit Lehrer die Daten verwenden können) und um Sortierung/Filterung zu ermöglichen




Welche Dokumente würden Sie gerne im Prozess oder am Ende des Prozesses ausdrucken können?		
Stammdaten (für Lehrer)
Übersicht, wer aufgenommen wurde (Mit Telefonnummer)

Welche Dokumente würden Sie gerne einscannen wollen und beim Datensatz hinterlegen?		
Zeugnis, lebenslauf
Eigentlich nur für eine Anmeldung				

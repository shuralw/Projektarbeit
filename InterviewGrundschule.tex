Interview mit Grundschule.txt
In welchem Umfang besitzen Sie Vorerfahrungen mit Schüler Online 1.0? 	
Keine	

In welchem Umfang besitzen Sie Vorerfahrungen mit der neuen Software?	
Keine	









Welche Probleme können bei herkömmlichen, nicht-digitalen Anmeldungen von Schülern auftreten?	
- Fehlende Einträge
- Unleserliche Schrift
- Fehlende Unterschriften	











Wer nutzt das System hauptsächlich an Ihrer Schule?	
- Sekretärin	









Welche fachlichen und technischen Qualifikationen sind zur Bewältigung der Aufgabe erforderlich (Aufgabenbewältigung / Softwarenutzung)? Welche Vorkenntnisse fehlen ggf.?	
- Kenntnisse mit MS-Office Programmen
- Grundwissen im Schulgesetz vorteilhaft, aber keine Vorraussetzung
- Datenschutz sensibilisierung sollte vorliegen	










Beschreiben Sie die Ausgangssituation die vorliegt, bevor Sie die Aufgabe "Anmeldung eines Schülers" durchführen.	
- Eltern füllen Papierbögen aus zur Anmeldung (6-7 Seiten mit Medienarbeit, Datenschutz, etc.)
- Sekretärin pflegt dies ein
- Selten kommt ein vorausgefüllter Bogen der Eltern per E-Mail	
		
















Zu Beginn der Durchführung:	Direkter Klick auf Anmeldung => neue anmeldung erstellen	






Welche Arbeitsschritte sind durchzuführen?	
- Eltern kommen ins Sekreteriat
- Gespräch findet statt
- Daten werden angegeben
- Unterlagen werden eingereicht
- Schülerakte wird angelegt
- (Einladen der Eltern findet bei Onlineanmeldung später statt)	



Welche Hilfsmittel sind erforderlich (für die Aufgabenbewältigung / zur Softwarenutzung)? Welche davon fehlen ggf., welche sind zusätzlich gewünscht?	


















Welche Ergebnisse / Teilergebnisse entstehen und wie werden diese ggf. verwertet / weitergeführt?	
- Ergebnis: Schüler ist Angemeldet
- Daten können herausgezogen werden falls benötigt	










Welche wichtigen Sonderfälle müssen berücksichtigt werden? (bzw. fallen dem Benutzer spontan ein; z. B. zur Arbeitsteilung / Zusammenarbeit)	
- Staatsangehörigkeit (Blick auf Pass)
- Sorgerechtsfälle (getrennte Eltern, etc.)	
		














Während der Durchführung:		
- Schüler Direkt suche, ob der Schüler schon vorhanden ist (wird immer so gemacht laut eigener Aussage)	 (?)
- ID Schlüssel ist unklar























Bildungsgang
- Die Meldung zur \textit{vorzeitigen Schulaufnahme} ist irritiert
- Werte des DropdownsBildungsgang verwirrend/unbekannt
- Beschulungsende wird leergelassen, weil noch nicht bekannt
 	























































Persönliche Daten	
- Unsicherheit, ob Ziffer im Anmeldeformular bei der Telefonnummer eine 4 oder 9 ist => Lösung: vergleich, mit anderen ziffern, ansonsten gucken in Listen
- Anmerkung, dass bei Telefonnummer mehr Möglichkeiten hilfreich wären, mit Anmerkung, wozu diese Nummer gehört	
- Unterbechung durch Telefon => Unterbrechungen passieren nach eigener Aussage häufiger















Sorgeberechtigte	
- Hier würde angerufen werden, um einen Nachweis für Sorgerecht zu erhalten
 - Sekretärin ruft bei fehlenden Daten je nach Wichtigkeit der Daten nocheinmal bei den Eltern an
- Hier gab es eine Verwirrung, da die E-Mail doppelt eingegeben wurde (Es wurde angenommen, dass auch am Anfang die E-Mail der Mutter gewünscht war)
- Der "Erstellen" Button ist eher verwirrend, da erwartet wird, dass ein weiterer Sorgeberechtigter erstellt wird
- Versuch über die Tabs weiter zu Navigieren, kurze Verwirrung da dies nicht möglich ist, dann der klick auf "weiter" button
 - Unterbrechung durch Telefon => unterbechungen sind im regulären Schulaltag häufiger als derzeit (es sind Ferien) 
 - Unterbechungen sind außerhalb der Ferien andauernd, so das es wenig sinn ergibt die arbeit auf später zu verschieben, da auch dort immer wieder unterbechungen passieren




















Notfallkontakte
- Unpraktische Eingabe, da hier die vorherigen Daten erneut angegeben werden müssen 
 - Es wird ein Vergleich zu \textit{SchILD} gezogen, wo diese Daten automatisch übernommen werden
- Kurze Irritierung, da nur ein Notfallkontakt angegeben werden konnte, jedoch schnell den Button für weitere gefunden
	















Migrationshintergrund		










Bemerkungen

- Irritation, was das Schulkind bei den Bemerkungen sehen kann und was nicht
- Erwartung: Interne Notiz sieht nur die Schule und das andere auch das Kind, aber unsicher, da Infotext oben etwas anderes besagt


Letzte Tätigkeit		

Qualifikationen		

Termine		

Aufnahmeberatung
- Annahme, dass man hier einen Termin erstellen kann, eine E-mail versenden oder Brief verschicken kann, wenn hypothetisch nicht am Anfang angegeben worden wäre, dass ein Gespräch bereits stattgefunden hat	

Zusammenfassung
- Information zur verfrühten Anmeldung wird als fehlerhaft wahrgenommen
- Startcode wird interpretiert als Code für Anmeldung des Schülers (Hinweis wird danach gelesen und erklärt dies)
- Unklar, ob dies bereits eine Anmeldung oder nur eine Bewerbung darstellt	
- Sekretärin hat das Gefühl, dass Daten fehlen: Masernschutz, Betreuung (VG oder OGS)
- Unsicher ob die Möglichkeit zur Angabe einer vorherigen Schule, weiterführenden Schule, etc. fehlt
- Die Eingabe für OGS, etc. wäre wichtig
- Zu Aufgabe 2: Klasse würde noch eingetragen werden, Aufnahmeberatungstermin fehlt
- Sonderpädergogischer Fall unklar, was genau die Einschränkung ist (?)

Übersichtsseite
- Blick auf die Übersichts Icons, drei fehlende Identifiziert => sollten noch gemacht werden
- unklar, was exportiert bedeutet/an wen
- unklar, welche pflichtunterlagen fehlen






Nach der Durchführung:		
Konnten Sie die Aufgabe aus Ihrer Sicht erfolgreich und vollständig abschließen? Falls nein was hat Sie daran gehindert?	
- Ja
- Ich habe den Eindruck, dass sich einige Fragen irgendwann von selbst klären, wenn man etwas mit der Anwendung vertraut ist















Wie effektiv unterstützt die Webanwendung Sie bei der Aufgabe?  Gab es positive oder negative Erfahrungen?	
- Positiv: Design ist intuitiv
- Schritt zurück ist immer möglich bei falscher Eingabe (dies geht bei \textit{SchILD} nur mit Tastenkomib)
 - Man kann jederzeit zurück gehen und Daten korrigieren
- Negativ: Umgewöhnung	











Haben Sie sämtliche Inhalte der Aufgabe verstanden? Gab es Stellen, an denen Sie sich mehr Unterstützung gewünscht hätten?	
- Es gab ein paar Rückfragen, aber keine zusätzlichen Probleme	




Gab es Schwierigkeiten oder Verwirrungen bei der Aufgabe? Wenn ja, welche?		
- Nein















Wie verständlich waren die Rückmeldungen der Anwendung?	
- Verständlich außer Fehlermeldung zur \textit{vorzeitigen Einschulung}











Welche Fähigkeiten setzt die Anwendung Ihrer Einschätzung nach voraus?	
- Vorherige Arbeit mit EDV
- Kaufmännischer Beruf sollte erworben sein
- Restliches Wissen kommt aus dem Alltagsgeschäft	

Wie sehr entspricht die Umsetzung in der Software der Realität? 	
- 
Passt zur Realität.	














Was handhaben Sie in Ihrem Arbeitsalltag bei nicht-digitalen Anmeldungen gewöhnlicherweise anders als in der Anwendung?	
- Nichts



















Was würde Sie noch daran hindern die Software in Ihrem Arbeitsalltag einzusetzen?	
- Umgewöhnung und der daraus resultierende Zeitaufwand
- Schnelles Lernen wäre sehr praktisch und angenehm und könnte diese Hürde nehmen		
		
















Gibt es Funktionen, die Sie in ähnlichen bzw. anderen Anwendungen genutzt haben, die Sie hier vermissen?	 
- Reports oder Ähnliches um die Daten aus dem Programm zu ziehen	








Welche Software sollte man aus Ihrer Sicht in Schüler Online integrieren bzw. eine Schnittstelle schaffen? 
- Excel Schnittstelle
- Kunden sollten daten aus Excel oder ähnlichem importieren können	




Welche Dokumente würden Sie gerne im Prozess oder am Ende des Prozesses ausdrucken können?	
- Schülerstammblatt (mit Laufbahn)
- Klassenlisten 
- Buskinder
- Kinder mit Migrationshintergrund
- Förderkinder
- Sortierung nach Nationalitäten
- Religion (Teilnahme an entsprechendem Unterricht) 
- Eine Anmerkung nebenbei: Ein Gesamtüberblick über alle Buskinder + deren bushaltestellen wäre erleichternd

Welche Dokumente würden Sie gerne einscannen wollen und beim Datensatz hinterlegen?	
- Digitale Personalakten
- Sorgerechtsbescheide
- Anmerkungen über Aktuelles (Kuraufenthalt)	


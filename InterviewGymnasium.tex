In welchem Umfang besitzen Sie Vorerfahrungen mit Schüler Online 1.0? 	
- Bisher gab es 2 mal pro Jahr Schulwechsel zum Lüttfeld oder Hanse Berufskolleg. Daten einpflegen läuft gut
- Eine Person hatte Praktikum im Lüttfeld => Daher vorerfahrung, aber nicht weiter geübt.

In welchem Umfang besitzen Sie Vorerfahrungen mit der neuen Software?	
- Keine









Welche Probleme können bei herkömmlichen, nicht-digitalen Anmeldungen von Schülern auftreten?	
- Daten nicht richtig in jedweder Software eingegeben => Eltern verrechnen sich bspw. mit Einschulungsjahr
- Schüler geben falsche Daten an (Nicht aus böswilliger Absicht sondern versehentlich)
- Getrennt lebende Eltern Erzieherdaten sind falsch






Wer nutzt das System hauptsächlich an Ihrer Schule?	
- Sekreteriat









Welche fachlichen und technischen Qualifikationen sind zur Bewältigung der Aufgabe erforderlich (Aufgabenbewältigung / Softwarenutzung)? Welche Vorkenntnisse fehlen ggf.?	
- Keine besonderen fachlichen oder technischen Kenntnisse notwendig
- Alles verständlich. 
- Man muss lesen können - mehr nicht. Man wird Schritt für Schritt durch das Programm durchgeführt
- Tiefgreifendes Wissen über das Schulsystem ist nicht notwendig
 






Beschreiben Sie die Ausgangssituation die vorliegt, bevor Sie die Aufgabe "Anmeldung eines Schülers" durchführen.
- Es liegt ein Anmeldeformular vor 
    - alle Punkte müssen ausgefüllt werden (leserlich). 
    - Bei wichtigen sachen (z.B. Sprachenwahl) müssen Unterlagen eingereicht werden
    - Bei unleserlichen / fehlendem wird (insbesondere telefonisch) nachgefragt. 
- Vor Corona fand alles vor Ort statt (mit Fristen), seit Corona immer mehr digital => Dies hat sich gut bewährt => Wird weiterhin übernommen => Dies bietet den Eltern Ruhe, sie haben die Daten dort zur Hand
- Manche Eltern fangen zuhause an, laden sich das Anmeldeformular herunter und wollen die Anmeldung selber ausfüllen. Dies brechen sie aber teilweise ab und wollen es lieber persönlich machen, da sie entweder aufgeben oder doch lieber persönlich nur ihre Daten abgeben.











Zu Beginn der Durchführung:	
- Es gab hier eine Unterbrechung durch ein Telefonat
- Es war zunächst unklar, wie man in das Formular zum Erfassen von Anmeldungen kommt. Die Sekretärin hatte einen Hauptmenüpunkt ausgewählt und wurde auf eine leere Seite geleitet. Sie hätte noch eine weitere Auswahl treffen müssen und auf einen Untermenüpunkt anschließend klicken müssen. Weiterhin konnte sie nicht differenzieren, wo der Unterschied zwischen den Menüpunkten "Schüler:innen" und "Anmeldungen" bestand. Sie hatte zwar als Aufgabe eine Anmeldung zu erfassen, konnte aber mit dieser Informationslage nicht korrekt navigieren. 
- Die Menüpunkte Schülerinnen und Anmeldungen sollten besser differenziert werden => unklar, welcher Menüpunkt hier für Aufgaben verwendet wird.
- Wenn die Eltern ihr Kind anmelden wollen werden sie tendenziell eher bei der aufnehmenden Schule als bei der abgebenden Schule nach Informationen zur Anmeldung suchen. 
- Es gab in Folge einer weiteren Unterbrechung einen Hinweis der Sekretärin: Es wäre super, wenn auch Pausen im Vorgang möglich wären und die Daten nicht verfallen würden. Unterbrechungen können auch mal eine halbe Stunde oder Stunde andauern. Manchmal fängt man morgens mit einer Aufgabe an, aber kann erst mittags/nachmittags weitermachen.



Welche Arbeitsschritte sind durchzuführen?	
- Schüler suchen, Daten des Schülers und der Anmeldung erfassen und alles zugehörige.
- Es muss dem Schüler noch eine Information anschließend übermittelt werden, das wurde aber von der Anmeldung genannt.	





Welche Hilfsmittel sind erforderlich (für die Aufgabenbewältigung / zur Softwarenutzung)? Welche davon fehlen ggf., welche sind zusätzlich gewünscht?	
- Es ist grundsätzlich praktisch, wenn man direkt aus anderen Stellen Text kopieren kann. Aus E-Mails funktioniert das in der Regel super, Pdf nur teilweise. 
- Es sollte von der Anwendung ein Einfügen der Daten unterstützt werden. 













Welche Ergebnisse / Teilergebnisse entstehen und wie werden diese ggf. verwertet / weitergeführt?	













Welche wichtigen Sonderfälle müssen berücksichtigt werden? (bzw. fallen dem Benutzer spontan ein - z. B. zur Arbeitsteilung / Zusammenarbeit)
- Wechsel von anderer Schule UND Wiederholung des Jahres müssen unterstützt werden
- Es gibt Flüchtlingskinder, wo nicht alle Daten vorhanden sind
- Langfristige Beurlaubung oder ähnliches, bei dem unklar ist, in welche Klasse das Kind gehen soll
    - Fallbeispiel: Kind ist in Therapie, nimmt nicht an schule teil, Eltern ziehen um und müssen es aufgrund der Schulpflicht trotzdem an der Schule anmelden
- Diese Kinder werden in der Regel in der Klinik geschult (Abfrage der Unterichtsinhalte an Schule) mit Personal der Klinik




Während der Durchführung:		

Schüler Suche
- Es war unklar, ob der ID-Schlüssel angegeben werden muss, es wurde vermutet, dass das kein Pflichtfeld ist.
- Der Schüler-Tab wird als Schülersuche nach Namen interpretiert
- Das Feld Schüler-Id-Schlüssel war unverständlich, die Sekretärin dachte laut: "Braucht man den? Wofür ist der da? Kann man den ignorieren?"
- Die Sekretärin dachte laut: "Was bedeutet die Suche? Wo genau wird gesucht?" Laut ihr wäre hier ein Anhaltspunkt wichtig (Hinweistext)










Bildungsgang
- Hier gabe es zwischenzeitlich wieder eine Unterbrechung durch ein Telefonat
- In den Anmeldungsunterlagen muss enthalten sein, zu welchem Schuljahr die Anmeldung sein soll
- Es könnte auch das laufende schuljahr sein
- Es war unklar, warum man kann keine Klasse auswählen kann => Es wäre hilfreich, wenn laut ihr die Klassen in dem Dropdown stehen. 
- Der Jahrgang sollte angegeben werden können, es steht noch gar nicht fest, welche Klassen es geben wird => Es gibt Profilklassen (Sport, Bilingual, etc.)
- Die Klasse wure als Jahrgangsstufe interpretiert.
- Die Eltern nehmen laut der Sekretärin erfahrungsgemäß den Aufnahmestatus "Angemeldet" als Bestätigung / Nachweis, obwohl die Entscheidung über eine Aufnahme an der Schule erst intern später gefällt wird. Dies könnte ein Druckmittel sein. Der Elternwille sei im Allgemeinen sehr stark ausgeprägt.
- Vorschlag: Rubrik "in Bearbeitung" statt Warteliste
- Es müssen die Daten geprüft werden, ob die Anmeldung des Schülers auf "Aufgenommen" gesetzt werden kann
- Es würde 1zu1 durch alle Daten durchgegangen werden müssen um die Anmeldung zu bestätigen
- Wünsche der Eltern können in anderen (Anwendungsfremden) Formularen mit angegeben werden => als Hinweis (?)
- Starke Einschränkungen bezüglich der Eingabe (z.B. nur zwei Namen als Wunschklassenkameraden) im Schülersystem wären hilfreich
- Unklar, warum für die Anwendung wichtig ist, dass eine Aufnahmeberatung erfolgt ist
- Die Schule gibt bei dem Schulbeginn/Ende immer genau die Daten an, das machen allerdings nicht alle Schulen so, was für Nachweise wichtig sein kann, unter anderem für Kindergeld und weitere informationen, wie lange das Kind an der Schule war
    - Es wäre das beste, wenn die Gesetzgebung den Schulbeginn/Ende vorgibt und es so für alle Schulen gleichmäßig ist
    - Es wäre hilfreich, wenn der Schulbeginn/Ende direkt unter dem Schuljahr ist und dann der Beschulungsbeginn/Ende dort individuell angegeben werden kann. Nach Klasse, Datum unklar ist, wie man weiter soll (klick auf weiter, Aufnahmestatus wird rot)
- Die Auswahloptionen beim Aufnahmestatus-Dropdown sind gut. Es wurde "Angemeldet" ausgewählt, da das Wort eine Aufnahme suggerierte. Eine Übereinstimmung der Begrifflichkeiten mit \textit{SchILD} wäre besser
- Eine Person weiß zunächst nicht, wozu die Uhrzeit dient, die andere Person erklärt, dass in der Oberstufe unterschiedliche Startuhrzeiten vorliegen. Sie fanden die Idee eine Uhrzeit mitzunotieren generell gut sobald das Feld interpretiert wurde. 
- Es sollte eine Unterscheidung oder Markierung geben, welche Felder den Eltern sichtbar gemacht werden.
- Die Schulgliederung wird interpretiert als die Schulgliederung, welche vorher besucht wurde, es war unklar was ausgewählt werden sollte
 	


Persönliche Daten
- Intuitiv, die Hausnummer in das Feld "Straße", aber nicht in "Hausnummer" zunächst eingetragen
- Die E-Mail des Sorgeberechtigten wurde in das Feld für die Schulkindmail eingetragen
- Der Tab "Persönliche Daten" sollte vor dem Tab "Bildungsgang" erscheinen, da das auch so in \textit{SchILD} ist.
- Die Suche nach der Postleitzahl-Ort Kombination funktionierte zunächst nicht.
- Die Eingabe eines Staates wird als unnötig erachtet, da es in der Umgebung keine ausländischen Wohnorte gibt. Es wäre sonst ein Vorausfüllen des Feldes mit "Deutschland" gut.
- Die Angabe eines Ortsteiles ist gut, auch dass es keine Pflichtangabe ist
- Telefonnummer (weitere) => Wunsch an wen das geht => wird nicht gesehen, dass oben der Tab für Sorgeberechtige (?) 
- Die Telefonnummer sollte bei der Sek1 den Hinweis haben, dass die Telefonnummer der Eltern benötigt wird. Ab der Sek2 sollte man die des Schülers nehmen.               




Sorgeberechtige
- Aus der Anmeldung ist nicht ersichtlich, ob alleiniges Sorgerecht. Dies wird angenommen, da nur die Mutter angegeben ist
- Es ist wichtig, dass es einen Nachweis gibt, dass eine Person alleiniges Sorgerecht oder das beide Unterschriften vorhanden sind, damit sich nicht eine Person über die andere hinweg setzt
- PLZ und Ort => auch hier konnte zunächst nicht die gesuchte Kombination aus PLZ+Ort gefunden werden.
- Sorgeberechtige Daten sollten die möglichkeit haben, dass die Adressdaten aus den Schülerdaten übernommen werden
- Es wurde versucht bei der Adressart die Straße anzugeben. 
- Es war eher unklar, warum es ein Posfach geben kann
- Der Sorgerechtigten Prozess wurde ausversehen abgebrochen durch klick neben dem Popup. Dies führt zu Unverständnis und macht den Nutzer sauer.
- Es gab eine weitere telefonische Unterbrechung
- Nochmal der Hinweis: Die Daten sollten nicht sofort nach ein paar Minuten verschwinden, auch nach logout und danach Login sollten die vorher erstellten Daten bestehen bleiben. Hier vielleicht eine Frage nach erneutem einloggen. "Sie haben noch ungesicherte Daten, wollen sie diese weiter bearbeiten?"















Notfallkontakte
Notfallkontakte werden vom Sekretariat nicht kontaktiert für Auskunft über die Leistung, etc.	


















Migrationshintergrund
- Bin zufrieden, alles wichtige vorhanden
- Die Erfassungsmaske sollte direkt angezeigt werden, auch wenn noch nicht "liegt vor" ausgewählt wurde
- Es ist unklar wie es weiter geht













Letzte Tätigkeit
- Nach Auffassung der Sekretärin hätte die letzte Tätigkeit noch im Kontext der Bearbeitung der Schüler-Daten kommen sollen, da es sich um ergänzende Informationen zur Person handelt, nicht erst nach den Sorgeberechtigten und Notfallkontakten
- Bei nicht gefundener Schule ist es unklar wie man weiterehen soll, es wird eine Checkbox überdeckt von der Suche
- In der Schulsuche sollte die eigene Umgebung bevorzugt werden bezüglich der Sortierung (also erst diese einträge und dann der Rest von NRW/Deutschland)
- Die Schulen-Suchfunktion ist sehr speziell, hier ist die Suchfunktion von \textit{SchILD} gut, wo auch kleinere details zum Finden der Schule genutzt werden
- Bei dem Dropdown "Letze Schule" sollte nur realistische Auswahlmöglichkeiten angezeigt werden (Bei Anmeldungen für die Sekundarstufe 2 sollten bspw. keine Grundschulen gelistet werden.)
- Die Checkbox unter der Schulsuche wird ignoriert und zunächst nicht verwendet.
    - Sie sollte besser Sichtbar gemacht werden.
- Wenn die Kinder die Informationen selber ins Formular geben, dann geben diese oft nur Klasse "4" an, ohne die Information 'a','b','c','d','e', ...
- Die Daten müssen aus häufig aus Zeugnissen und anderen Dokumenten gelesen werden, da die Kinder/Eltern die Informationen entweder gar nicht oder unpassend angeben	

Bemerkungen
"Bemerkung" tab sollte wieder früher kommen 
 
Qualifikationen		

Termine		

Aufnahmeberatung		

Zusammenfassung	
- Beim Speichern tritt eine Fehlermeldung auf (Die Fehlermeldung ist sehr unklar) => Hier sollte eine Führung kommen, wo genau darauf hingewiesen wird, wo der fehler liegt und wie dieser behoben werden kann.
- am besten sollte der Fokus auf das Feld gesetzt werden, welches Fehlerhaft ist.
Suche nach der möglichkeit die geendete Anmeldung zu öffnen. Klick auf Schüler/Anmeldung häufig ohne den Unterpunkt


Aufgaben 2 und 3 
- Man müsste beim Bearbeiten einer Anmeldung sämtliche Daten nochmal gegenprüfen, damit die Validität gegeben ist. 
- Die Sekretärin hovert lediglich über das Stift Symbol, mit dessen Hilfe Sie den Schülerdatensatz bearbeiten könnte.
- Die Funktion des Buttons "Anmeldungsformular" ist zunächst unklar.
- Das Anmeldungsformular wird so interpretiert, dass es nur für die Akte ist / wird als unnötig empfunden
- Es ist unklar, warum Der Haken bei "Unterlagen" nicht ausgewählt werden kann
- Es ist unklar, wozu die Checkbox "Schüler exportiert" und die Exportfunktion dienen. 
- "Anmeldung Exportieren" wird als Export an neue annehmende Schule interpretiert
- Es ist unklar, warum die blaue formatierten Texte "Dokumente" nicht angeklickt werden können
- Es wird vertrauen in die Anwendung benötigt um eine Anmeldung anzunehmen ohne alle Daten einzeln zu prüfen
- Sollte sich ein Schulkind über logische dinge hinwegsetzen, dann sollte SO2.0 darauf hinweise, dass es Diskrepanzen in den Daten gibt 
    - Beispiel 1: Wenn eine Anmeldung an die Sekundarstufe 1 erfolgt und die letzte Jahrgangsstufe=3 ist
    - Beispiel 2: Anmeldung an Sek2 als Grundschüler
    - Beispiel 3: Datumsangaben widersprechen sich
    - Es sollte dann eine Warnung erscheinen oder sich ein Fenster öffnen, sobald man speichern möchte.

- Validierungsproblemen sollten nur eine Warnung sein, da das z.B. bei Überspringen der Klasse durchaus das richtig sein kann









Nach der Durchführung:		
Konnten Sie die Aufgabe aus Ihrer Sicht erfolgreich und vollständig abschließen? Falls nein was hat Sie daran gehindert?	
- Erfolgreich beim zweiten Anlauf
- Es sind verschiedene Fragen aufgetaucht
- Es war unklar was noch fehlt
- Reiter anordnung hat alles unersichtlich gemacht
- Hilfeoption oder ähnliches wäre praktisch
- Nicht jedes mal nach neuen Anmeldungen gucken, vielleicht sollte es Ziffern oder eine andere Hervorhebung geben, dass eine neue unbearbeitete Anmeldung vorhanden ist
- Ein Handbuch wäre gut um selber Probleme zu korrigieren











Wie effektiv unterstützt die Webanwendung Sie bei der Aufgabe?  Gab es positive oder negative Erfahrungen?	
- Positiv im vergleich zum alten SchülerOnline => bessere Struktur, übersichtlicher	
- Felder sollten direkt bei Validierungsproblemen markiert werden, so das nicht im Nachhinein alle Schritte erneut durchgeführt werden müssen










Haben Sie sämtliche Inhalte der Aufgabe verstanden? Gab es Stellen, an denen Sie sich mehr Unterstützung gewünscht hätten?	
- Nur die reinen Daten genügen nicht 

Gab es Schwierigkeiten oder Verwirrungen bei der Aufgabe? Wenn ja, welche?
- Nein (nur ob Anmeldung oder Schüler Tab, aber dies ist kein großes Problem)	













Wie verständlich waren die Rückmeldungen der Anwendung?
- Ausbaufähig, es sollte mehr Feedback geben, mit konkreten Anweisungen	









Welche Fähigkeiten setzt die Anwendung Ihrer Einschätzung nach voraus?
- Es kann jeder machen, die Begrifflichkeiten wären auch für Laien verständlich	

Wie sehr entspricht die Umsetzung in der Software der Realität? 	

- Das Erfassen der Anmeldung sollte so aufgebaut sein wie die Dokumente der Schule oder vergleichbare Software ( \textit{SchILD} ), damit man beim übertragen von Daten aus einem schriftlichen Formular nicht blättern muss und beim Kopieren aus einer Software wie \textit{SchILD} die Daten in der gleicher Reihenfolge abgefragt werden, was zu weniger Verwirrung führen würde.
- Ansonsten inhaltlich genau wie mit \textit{SchILD} => entspricht der Realität. Keine Auffäligkeiten	





Was handhaben Sie in Ihrem Arbeitsalltag bei nicht-digitalen Anmeldungen gewöhnlicherweise anders als in der Anwendung?	
Die Sekretärinnen gucken bei dem Papierformular nach, ob es irgendwo lücken gibt => Versuchen diese selber zu füllen und nehmen ansonsten Kontakt auf
- Ganz unwichtige Daten (Geburtsort) werden, falls es abgeschlossen werden muss, im notfall ausgedacht
- Wichtige Daten müssen mit Urkunden, etc. ausgefüllt werden	














Was würde Sie noch daran hindern die Software in Ihrem Arbeitsalltag einzusetzen?	
- Noch ein zusätzliches Programm => Mehr Aufwand 
- Dopplung zu \textit{SchILD} 
- Es sollte die Möglichkeit geben eine E-Mail anzugeben, wenn neue dinge (z.B. eine Anmeldung) vorhanden sind, womit dann in einem einstellbaren Intervall Neuigkeiten empfangen werden können.	
		












Gibt es Funktionen, die Sie in ähnlichen bzw. anderen Anwendungen genutzt haben, die Sie hier vermissen?
Die Filterung der Schule wie in  \textit{SchILD} , ansonsten alles bereits zuvor erwähnte.	







Welche Software sollte man aus Ihrer Sicht in Schüler Online integrieren bzw. eine Schnittstelle schaffen? 	
 \textit{SchILD} 	





Welche Dokumente würden Sie gerne im Prozess oder am Ende des Prozesses ausdrucken können?
- Die Sekretärin fand es gut dass das Anmeldeformular ausdruckbar war um es ggf. zu archivieren.
- Schweigepflichtsentbindung, Einverständnis für PKW / Bulli Mitfahrten, Fotoerlaubnis für Webseite / intern, 

Welche Dokumente würden Sie gerne einscannen wollen und beim Datensatz hinterlegen?
Keine.	

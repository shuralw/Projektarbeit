%scrartcl: Für kürzere Ausarbeitungen. Beginnt mit section, es gibt keine chapter
%scrreprt: Für längere Ausarbeitungen (Bacherlo oder Master Thesis). Beginnt chapter
\documentclass[pdftex,a4paper,abstracton,11pt,parskip=half,bibtotocnumbered]{scrartcl} 

% Je nach LaTeX Compiler werden etwas andere Bibliotheken verwendet.
% Das Paket iftex erlaubt es, den Compiler zu überprüfen
\usepackage{iftex}

\usepackage[ngerman]{babel} % Einstellungen für den deutschen Sprachraum, neue deutsche Rechtschreibung
\ifPDFTeX
	\usepackage[utf8]{inputenc} % Umlaute erkennen. Als Option in \glqq [] \glqq  die vom Editor verwendete Zeichencodierung auswählen
	\usepackage[T1]{fontenc}
	\defineshorthand{"<}{\glqq}
	\defineshorthand{">}{\grqq{}}
\fi

% Für kompakter Aufzählungen
\usepackage{paralist}

\usepackage[style=ieee,backend=bibtex]{biblatex}
\addbibresource{literatur/literatur.bib}

% Für Abbildungen
\usepackage{graphicx}
\graphicspath{{./abbildungen/}}

%Für Tabellen
\usepackage{longtable} % For tables that span multiple pages
\usepackage{booktabs} % Für schönere Linien in der Tabelle

% Ändert die Überschrift des Abstracts.
% Falls kein Abstract benötigt wird, kann die Option \glqq abstracton\grqq{} ganz oben in \documentclass entfallen.
\renewcommand{\abstractname}{Zusammenfassung}

% Für Landscape
\usepackage{pdflscape}
\usepackage{adjustbox}
\usepackage{float}


% Kopfzeile
\usepackage{fancyhdr}
\pagestyle{fancy}
\fancyhead{} % Löscht alle Kopfzeileneinstellungen
\lhead{\leftmark} % Aktuelle Sektion in der linken Kopfzeile
\fancyhfoffset[R]{0cm}

\usepackage[a4paper,left=2.5cm,right=2cm,top=3cm,bottom=2cm,foot=1cm]{geometry}

\title{Neugestaltung von \textit{Schüler Online}: Eine Beobachtungs- und Interviewstudie zur Identifikation von Problemstellen und Nutzerbedürfnissen, um die Effektivität sowie die Zufriedenstellung des Schulpersonals beim Erfüllen von Kernaufgaben der Webanwendung zu optimieren}
\author{Lukas Wessel}
\date{\today}

\begin{document}
\pagenumbering{gobble} % Keine Seitenzahlen bis hier

\makeatletter
\begin{titlepage}
	\centering
	{\scshape\LARGE Fachhochschule Südwestfalen \par}
	\vspace{1cm}
%	{\scshape\Large Merkblatt\par}
	\vspace{1.5cm}
	{\huge\bfseries \@title\par}
	\vspace{3cm}
	{\Large \@author\par}
	\vspace{1cm}
	{\Large \@date\par}
	\vfill

	\raggedright
%	{\large Eingereicht bei:\par}
%	{\large Betreuer 1}
\end{titlepage}
\makeatother

\thispagestyle{empty}
\begin{abstract}
%Ein Abstrakt, also eine Kurzzusammenfassung der Arbeit ist bei einer schriftlichen Ausarbeitung nicht unbedingt notwendig.
%Bei umfangreicheren Arbeiten, also z.~B. einer Bachelor- oder Master-Thesis, sollte die Ausarbeitung in jedem Fall mit einem \textit{Abstract} beginnen.
Offene Punkte, die ich noch erledigen muss:


done- umformulieren der Fat Ergebnisse 

done - Problemstellen mit der Literatur vergleichen

done - Schüler Online definieren

done - Interviews anhängen

done - Forschungsfragen beantworten

done - gibt es neue Erkenntnisse im Fazit?

- Tabelle überarbeiten und referenzieren

- Abbildung 1 erläutern, vor allem die Farben

- Diskussion: "Ergebnisse" Singularisieren 

- Bilder aus Anhang referenzieren

- Bei Abbildung 2 erläutern, wofür die Briefe und Akteure z.B. stehen und Schuljahre 

- Meister Schüler Modell: Literatur

- Ergebnis-Nummern: Sollen die lieber hinter oder vor den Punkt?

- Zweite Seite des Bewerbungsformulars ergänzen

- Inhaltsverzeichnis bis level 3?

- Anhangsverzeichnis

- Bezeichnungen im Anhang

- Abbildungsverzeichnis

- Ist bei jeder Abbildung ein Text dabei?

- Sperrvermerk

- Interviews überarbeiten

- wurden alle Ergebnisse verwendet?

- Abstract schreiben


\end{abstract}

\vfill
\tableofcontents
\pagebreak
\textbf{Gender-Hinweis}

Zur besseren Lesbarkeit wird in dieser Projektarbeit das generische Maskulinum verwendet. Die in dieser Arbeit verwendeten Personenbezeichnungen beziehen sich – sofern nicht anders kenntlich gemacht – auf alle Geschlechter.
\pagebreak

\pagenumbering{arabic} % Beginnt die Seitennummerierung ab hier mit arabischen Zahlen
\setcounter{page}{1}
\section{lose Themen}
\subsection{Zielsetzung}
Das Ziel der Forschung ist es, einen Einblick zu gewinnen, welche Problemstellen und Nutzerbedürfnisse das Schulpersonal beim Bewerbungsprozess eines Schülers existieren. Hierzu wird ein semistrukturiertes Leitfadeninterview durchgeführt und die Testpersonen offen beobachtet. Die Ergebnisse werden anschließend qualitativ ausgewertet. Die Forschung wird betrieben, indem Personen aus der genannten Gruppe die Anwendung Schüler Online die Aufgaben "Bewerbung an der Primarstufe erstellen", "Bewerbung an der Sekundarfstufe I erstellen" sowie "Bewerbung an der Sekundarfstufe II erstellen" vollziehen. 

\section{Einleitung}
%In der Einleitung solltest du die Zielsetzung deiner Projektarbeit, die zentralen Fragestellungen, die Vorgehensweise und die Bedeutung deines Themas für die Wissenschaft oder die Praxis erläutern.
%=> Ggf. eine Annahme darlegen: Schüler sitzen dem Schulpersonal gegenüber wenn sie sich bewerben, die Sekretärin füllt die Daten aus die der Schüler ihr nennt.
%Leitfrage: Welche Erwartungen hat der Anwender an die Software?
%Leitfrage: Welche Nutzerbedürfnisse gab es bei den Feldtests?
%Leitfrage: Wie sollte so ein Fragebogen aussehen?
%Leitfrage: Wie sieht der Nutzungskontext bei einer Sekretärin in einer Schule aus?
%Leitfrage: Welcher Nutzerbedürfnisse werden von der Anwendung erfüllt, welche nicht?
%Leitfrage: Welche Problemstellen gab es bei den Feldtests?
%Leitfrage: Welche Gemeinsamkeiten gab es bei den Feldtests?

%Zu Beginn der Ausarbeitung ist es besonders wichtig, dem Leser zu erklären, warum Sie das Thema bearbeiten.
%Dabei ist nicht unbedingt Ihre eigene Motivation gemeint, sondern vielmehr die Frage, warum die Problemstellung Ihrer Arbeit relevant ist.
%Angemessene Motivationsgründe sind etwa:
%\begin{compactitem}
%\item Sie lösen ein Problem, für das es bisher keine, oder keine gute Lösung gibt.
%\item Ein Unternehmen kann Ihre Lösung einsetzen, um damit Profit zu erwirtschaften.
%\item Sie vergleichen verschiedene Produkte oder Methoden, um damit die Entscheidungsfindung bei der Auswahl zu erleichtern.
%\item Sie stellen ein komplexes Thema für eine bestimmte Zielgruppe angemessen dar.
%\end{compactitem}
Die zugrunde liegende Problemstellung ist relevant für den Hersteller der Anwendung "Schüler Online" ("Kommunales Rechenzentrum Minden/Ravensberg-Lippe"), da Unklarheit herrscht, ob die Anwender die Software korrekt bedienen können. Die Korrektheit ist auch für die Anwender wichtig, da die Anwendung das Schulgesetz abbilden soll und eine korrekten Erfüllung ermöglichen soll. Der Nutzer der Anwendung soll zufrieden sein, seine Erwartungen und Bedürfnisse an die Software sollen erfüllt werden. Mit der vorliegenden Studie soll geprüft werden, inwieweit die Software ebenjenen entspricht oder abweicht.

\section{Theoretischer Hintergrund}
In diesem Kapitel solltest du den theoretischen Hintergrund deines Themas erläutern und den aktuellen Forschungsstand darstellen. Hierbei kannst du auf Literatur und Quellen zurückgreifen.

\subsection{Definition von Effektivität im Rahmen dieser Ausarbeitung }
Die ISO 9241-110 definiert Effektivität wie folgt: \glqq Effektivität = Die Genauigkeit und Vollständigkeit mit der Benutzer ein bestimmtes Ziel erreichen\grqq{} \cite{ISO-9241-110}. 
\grqq{}[Sie] bezeichnet das Ausmaß der Übereinstimmung von tatsächlichen und angestrebten Ergebnissen.\glqq \cite{iso-9241-11}. Ein \glqq Mangel an Effektivität kann zu Ergebnissen führen, die nutzungsbedingte Schäden nach sich ziehen könnten.\glqq \cite{iso-9241-11}
In Anlehnung an diese Definition kann also festgelegt werden, dass Effektivität im Sinne dieser Ausarbeitung den Erreichungsgrad Genauigkeit und Vollständigkeit mit dem die Studienteilnehmer die drei an sie gestellten Aufgaben erreichen. 

Erfolgskriterien, die darauf hinweisen, dass eine Aufgabe erfolgreich abgeschlossen wurde, sind: 
\begin{itemize}
    \item Die Navigation zur Aufgabe wurde erfolgreich abgeschlossen (EK1)
    \item Es wurde über die Aufnahme des Kindes bewusst entschieden (EK2)
    \item Die Anmeldung wurde erfolgreich gespeichert (EK3)
    \item Die Rückmeldung \glqq Anmeldung wurde erfolgreich verschickt\grqq{} wird von der Anwendung ausgegeben (EK4)
\end{itemize}

\subsection{Definition von Zufriedenstellung im Rahmen dieser Ausarbeitung}
Die ISO 9241-110 definiert Effektivität wie folgt: Zufriedenstellung: \glqq Das Ausmaß der Übereinstimmung der physischen, kognitiven und emotionalen Reaktionen des Benutzers, die aus der Benutzung eines Systems, eines Produkts oder einer Dienstleistung resultieren, mit den Benutzererfordernissen und Benutzererwartungen.\glqq \cite{ISO-9241-110} In Anlehnung an diese
%todo
\begin{itemize}
    \item Freiheit von langen Wartezeiten
    \item Freiheit von Behinderungen (Beispiel: Nutzer kann ein Dialogfeld nicht ausfüllen, weil es zu klein ist)
    \item Freiheit von unlösbaren Fehlern (Beispiel: Nutzer erhält Fehlermeldung und kann diese nicht beheben)
    \item Freiheit von Eingabefehlern (Beispiel: Person wählt zu einer Postleitzahl einen nicht zugehörigen Ort aus)
    \item Freiheit von Rauswürfen aus dem Prozess (Beispiel: Nutzer klickt versehentlich auf \textit{Abbrechen} und bricht somit unfreiwillig die Eingabeerfassung ab)
    \item Freiheit von un-/missverständlichen Texten oder Eingabefeldern (Beispiel: Nutzer versteht das Eingabefeld \textit{Sorgerechtsgrund} nicht)
\end{itemize}

\subsection{Definition von Problemstellen im Rahmen dieser Ausarbeitung }
Im Rahmen dieser Arbeit bezieht sich der Begriff \textit{Problemstelle} auf spezifische Elemente der 'Schüler Online'-Software, die bei ihrer Nutzung durch das Schulpersonal zu Defiziten in Bezug auf Effektivität und Zufriedenheit führen.

\subsection{Definition von Nutzerbedürfnissen im Rahmen dieser Ausarbeitung }
\cite{stangl} 
Stangl beschreibt den Begriff Bedürfnis als \glqq das Verlangen oder der Wunsch, einen empfundenen oder tatsächlichen Mangel Abhilfe zu schaffen.\glqq \cite{stangl} Dies wirft die Frage auf, was ein Mangel im Kontext einer Anmeldung an einer Schule für den Anwender bedeutet. Wenn man herkömmliche Anmeldungen mittels eines Papierformulars betrachtet, kann man hier argumentieren, dass mehrere Aspekte mangelhaft sind. Beispielsweise gibt es keine Validierung der Daten hinsichtlich Korrektheit oder Plausibilität.  Datenvalidierung könnte allerdings durch eine gute Kommunikation mit der abgebenden Schule abgesichert werden. Das Einlesen der Daten ist möglicherweise problematisch, da handschriftliches Ausfüllen unleserlich geschrieben sein kann. Eine Übertragung in verwendete Schulsoftware, wie \textit{SchILD-NRW} kann nur durch unkomfortables Abtippen erreicht werden.
Eine auf diesen Argumenten basierende modellhafte Definition im Kontext dieser Arbeit kann also lauten: \glqq Nutzerbedürfnisse ist das Verlangen oder der Wunsch, die Datenerfassung der Anmeldung weder unsicher, inkorrekt, unplausibel noch unkomfortabel zu vollziehen\grqq{}.

%\subsection{Typische Problemstellen und Nutzerbedürfnisse in aktuellen Webanwendungen}
\subsection{Typische Problemstellen in Webanwendungen}
%Aktuelle Webanwendungen\footnote{Als Annahme wird hier getroffen, dass Anwendungen, die jünger als 10 Jahre sind als aktuell gelten}
Im folgenden werden einige typische Probleme bei Webanwendungen gelistet und erläutert, die zu Defiziten der Effektivität und Zufriedenstellung führen können.

\begin{itemize}
    \item \textbf{Schlechte Navigationsstrukturen:} Wenn Nutzer Schwierigkeiten haben, sich auf einer Website zurechtzufinden, können sie ihre Ziele nicht effektiv erreichen. Eine klare, konsistente und intuitive Navigationsstruktur ist entscheidend.
    \item \textbf{Nicht erfüllte Erwartungen:} Wenn das Design oder die Funktionalität der Anwendung nicht den Erwartungen der Nutzer entspricht, können diese ihre Ziele nicht effektiv erreichen. Beispielsweise kann eine Schaltfläche, die aussieht, als würde sie eine bestimmte Aktion auslösen, tatsächlich eine ganz andere Aktion auslösen.
    \item \textbf{Mangel an Feedback:} Nutzer müssen wissen, was passiert, wenn sie eine Aktion ausführen. Wenn eine Anwendung nicht angemessen auf Nutzereingaben reagiert, kann dies zu Frustration und Ineffektivität führen.
    \item \textbf{Nicht zugängliches Design:} Webanwendungen sollten für alle Benutzer, einschließlich Menschen mit Behinderungen, nutzbar sein. Eine Anwendung, die nicht die Richtlinien für Barrierefreiheit erfüllt, kann für einige Nutzer ineffektiv sein.
    \item \textbf{Schlechte Leistung:} Langsame Ladezeiten oder technische Probleme können die Effektivität stark beeinträchtigen, da sie Nutzer daran hindern, ihre Ziele in einer angemessenen Zeit zu erreichen.
    \item \textbf{Komplizierte oder überladene Benutzeroberflächen:} Wenn eine Benutzeroberfläche zu viele Optionen, zu viel Text oder zu viele Bilder enthält, kann dies Benutzer verwirren und ihre Fähigkeit, ihre Ziele effektiv zu erreichen, beeinträchtigen.
    \item \textbf{Mangel an Suchfunktion oder ineffektive Suchfunktionen:} Eine effektive Suche ist für viele Webanwendungen entscheidend. Wenn Nutzer nicht finden können, was sie suchen, können sie ihre Ziele nicht effektiv erreichen.

    %Schwierige Bedienung
    %Suchen nach: \glqq Common usability problems in web applications\grqq{} 
\end{itemize}

\subsection{Hawthorne Effekt}
Der Hawthorne-Effekt besagt, dass Personen ihr Handeln verändern, weil sie wissen, dass sie unter Beobachtung stehen. Er kann bei Teilnehmenden an wissenschaftlichen Experimenten vorkommen, deren Verhalten beobachtet wird. So ist ihr Verhalten unnatürlich.

\subsection{ggf. Soziale Erwünschtheit}
\grqq{}Beim Effekt der sozialen Erwünschtheit verändern Teilnehmende ihr Verhalten oder ihre Antworten bei Fragebogen, um ein positives Bild von sich selbst abzugeben.\glqq 

\subsection{DECIDE Ansatz}
\subsection{Beobachtung}
\subsection{Hawthorne Effekt}
\subsection{Loud Thinking}
\subsection{Feldtest / Labortest}

evtl. wie sieht das schulgesetz aus oder der Prozess


\newpage
\section{Material und Methode}
\subsection{Teilnehmerauswahl}

In der vorliegenden wissenschaftlichen Arbeit wurde auf zwei divergierende Pools zur Rekrutierung der Teilnehmer zurückgegriffen. Der erste Rekrutierungspool umfasste diejenigen Individuen oder vermittelnden Kontakte\footnote{Ein vermittelnder Kontakt ist ein Kontakt, der einen potenziellen Studienteilnehmer aus seinem persönlichen Umfeld kontaktiert hat und diesen gefragt hat, ob er an der Studie teilnehmen möchte}, die eine Schulung für \glqq Schüler Online 2.0\grqq{} besucht hatten. In diesen Schulungen wurde am Ende eine Folie präsentiert, die diese Studie vorstellte und um die Beteiligung der Anwesenden bat.

Ein alternativer Weg zur Gewinnung von Teilnehmern war die Kaltakquise. Dabei wurden Schulen in zwei umliegenden Kreisen telefonisch kontaktiert und um ihre Teilnahme an der Studie gebeten. Aus Gründen der Vertraulichkeit und der Einhaltung von Verschwiegenheitsvereinbarungen ist es an dieser Stelle nicht möglich, konkrete Angaben zur Region zu machen.

Die Auswahl der Teilnehmer erfolgte basierend auf einer Reihe von Kriterien. Wesentlich war, dass die Teilnehmer in ihrem beruflichen Alltag das zu untersuchende Produkt sinnvoll einsetzen konnten - dies war ein entscheidendes Inklusionskriterium. Besonders geeignet waren daher Sekretariatsmitarbeiter, Schulverwaltungsassistenten und potenziell auch Schulleitungen. Als weiteres Inklusionskriterium war es erforderlich, dass die Teilnehmer aktiv in den genannten Berufen tätig waren und nicht in den Ruhestand getreten waren.

Im Gegenzug dazu wurden Auszubildende und Lehrer von der Studie ausgeschlossen, da sie nur begrenzte Berührungspunkte und Erfahrungen mit dem Tätigkeitsfeld der Software hatten. Dies stellte ein explizites Exklusionskriterium dar. Darüber hinaus spielten Faktoren wie Alter, Geschlecht, Berufserfahrung, ethnischer Hintergrund und sozioökonomischer Status der Teilnehmer bei der Rekrutierung keine Rolle.

Die Teilnahme an der Studie erforderte nur die Verfügbarkeit und die Bereitschaft der Teilnehmer, sich freiwillig zu engagieren. Es wurde darauf geachtet, dass keine Teilnahmeverpflichtungen, beispielsweise durch Vorgesetzte wie die Schulleitung, entstanden.

\subsection{Durchführung der Beobachtungen und Interviews}

Die Beobachtungen und Interviews wurden im Rahmen eines Feldtests abgehalten und fanden während der regulären Arbeitszeit an den gewohnten Arbeitsplätzen der Studienteilnehmer statt. Die Entscheidung gegen eine Durchführung in einer Laborumgebung basierte auf zwei primären Überlegungen. Einerseits konnte eine solche Umgebung nicht die notwendige Realitätsnähe liefern, die für ein umfassendes Verständnis der Nutzung des Produktes im Alltag der Teilnehmer erforderlich war. Andererseits waren finanzielle Gründe ausschlaggebend für die Auswahl des Feldtests.

Hauptelemente, die diese Entscheidung begünstigten, waren zum einen die technische Ausstattung an den realen Arbeitsplätzen, wie beispielsweise Computer mit einer langsamen Internetverbindung, die einen signifikanten Einfluss auf die Handhabung des Produkts haben könnten. Zum anderen war es für die Studie relevant, die Arbeitsrealität der Teilnehmer zu berücksichtigen. Sekretariatsmitarbeiter beispielsweise unterbrechen ihre Tätigkeit typischerweise für Telefonate, was sich in einer Laborumgebung nicht authentisch hätte abbilden lassen.

Die Durchführung der Interviews wurde während der Sommerferien terminiert, was im Diskussionsteil dieser Arbeit noch näher beleuchtet wird. Es war von Bedeutung, den tatsächlichen Arbeitskontext der Teilnehmer einzubeziehen, um ein möglichst repräsentatives Bild ihrer Erfahrungen und Herausforderungen im Umgang mit dem untersuchten Produkt zu gewinnen.

\subsection{Erstellung des Fragebogens}

Die Formulierung des Fragebogens erfolgte durch eine Expertengruppe, die aufgrund ihrer beruflichen Rolle und Erfahrung eine hohe fachliche Expertise in Bezug auf die Anforderungen der zu untersuchenden Software besaßen. Diese Gruppe setzte sich zusammen aus Individuen, die im beruflichen Kontext die Anforderungen an die Software definierten und dokumentierten. Innerhalb dieser Gruppe gab es keinen Usability Experten, ledigliche einen Studenten, der das Fach \glqq Usability Engineering\grqq{} in seinem Studium belegte.

Um die Qualität und Eignung des Fragebogens zu gewährleisten, wurde dieser nach der Erstellung von einem Professor für Usability Engineering überprüft und auf seine inhaltliche Eignung hin bewertet.

Die Formulierung der Fragen folgte bestimmten Richtlinien. Sie sollten klar und verständlich sein und offen formuliert werden, um eine breite Palette von Antworten zu ermöglichen. In der Reihenfolge der Fragen wurde darauf geachtet, zunächst leicht zu beantwortende Fragen zu stellen, um die Teilnehmer nicht zu früh zu überfordern. Zudem war es wichtig, dass der verwendete Sprachschatz den Kenntnissen eines Mitarbeitenden im Schulsekretariat entsprach \cite{Kruse_2015}. Dies gewährleistete, dass alle Teilnehmer die Fragen ohne zusätzliche Erläuterungen verstehen und beantworten konnten.

\subsection{Entwicklung von Aufgaben und Szenarien}

Für die Durchführung des Feldtests wurden drei spezifische Aufgaben im System entwickelt, die die Teilnehmer während ihrer normalen Arbeitszeit bewältigen sollten. Dabei wurde darauf geachtet, dass die individuellen  Voraussetzungen für die jeweilige Schule gegeben waren, einschließlich  vorhandener Bildungsangebote an der jeweiligen Schule. Die konkreten Aufgaben lauteten wie folgt:

\begin{itemize}
\item Aufgabe 1: \glqq Erstellen Sie bitte für dieses Anmeldeformular von 'Max Müller' eine Bewerbung an Ihrer Schule.\grqq{}  (Nähere Details sind im Anhang zu finden.)
\item Aufgabe 2: \glqq Bearbeiten Sie bitte die Bewerbung von \textit{Lotta Meier} nach eigenem Ermessen.\grqq{}  Dieser Datensatz war so gestaltet, dass die Bewerbung entweder von einer abgebenden Schule oder von der Gemeinde gestellt wurde.
\item Aufgabe 3: \glqq Bearbeiten Sie bitte die Bewerbung von \textit{Konrad Schulz} nach eigenem Ermessen.\grqq{}  Bei diesem Datensatz wurde die Bewerbung von den Eltern des Schülers eingereicht.
\end{itemize}

\includegraphics[width=\textwidth]{konstellationenv3}


Um den Realitätsbezug zu gewährleisten, unterschieden sich die Datensätze je nach Schule, insbesondere in Bezug auf das Geburtsjahr, das an die jeweilige Schulstufe angepasst wurde. So wurde für Bewerbungen an der Primarstufe das Geburtsdatum so gewählt, dass das Schulkind zum Zeitpunkt des ersten Schultages 6 Jahre alt wäre. Im Falle einer Bewerbung für die Sekundarstufe 1 war das Schulkind 10 Jahre und für die Sekundarstufe 2 entsprechend 16 Jahre alt.
Hierzu wurden so genannte Anmeldeformulare der jeweiligen Schule verwendet, die mit Ausnahme des Geburtsdatums und des zu besuchenden Bildungsgangs mit identischen Inhalten ausgefüllt wurden.

Zu beachten ist, dass alle Datensätze fiktiv waren, um etwaige Probleme hinsichtlich der Vertraulichkeit zu vermeiden. Dies wurde den Teilnehmern vor Beginn der Aufgaben explizit mitgeteilt.

\subsection{Bereitstellung des Fragebogens}

Im Vorfeld der Untersuchung wurde der Fragebogen den Studienteilnehmern nicht vorab zur Verfügung gestellt. In den initialen Telefongesprächen wurde jedem Teilnehmer ausdrücklich mitgeteilt, dass keine spezielle Vorbereitung für die Teilnahme an der Studie erforderlich sei. Die Software und der Zweck der Studie wurde jedem Teilnehmer in diesem Zuge ebenfalls kurz dargelegt.

Darüber hinaus wurden in diesen Gesprächen die Software und der Zweck der Studie den Teilnehmern ausführlich erläutert. Dies gewährleistete, dass jeder Teilnehmer über den Kontext und die Ziele der Studie informiert war und eine Vorstellung von dem hatte, was von ihm oder ihr erwartet wurde.

\subsection{Rollenverteilung während der Studie}

Während der Durchführung der Studie gab es zwei Hauptrollen innerhalb des Forschungsteams: den Interviewer und den Schreiber. Der Interviewer, der die Software mehrere Jahre mitentwickelt hat, war mit den funktionalen Aspekten des Systems gut vertraut, hatte jedoch keine umfangreichen praktischen Erfahrungen hinsichtlich Interviewtechniken. Diese Rolle wurde durch eine Person besetzt, deren Aufgabe es war, die Teilnehmer durch die Aufgaben zu leiten und die Diskussion während des Interviews zu lenken.

Die Rolle des Schreibers wurde durch zwei Personen wahrgenommen, die sich abwechselten. Schreiber A nahm an den Interviews 1 und 4 teil, während Schreiber B an den Interviews 2, 3 und 5 präsent war. Beide Schreiber hatten grundlegende Kenntnisse der Software, die sie während des ersten Jahres ihrer Beteiligung an der Entwicklung der Software erworben hatten.

Die Hauptaufgabe der Schreiber war es, während der Interviews Notizen zu machen und die Reaktionen der Teilnehmer sowie relevante Beobachtungen zu dokumentieren. 

\subsection{Durchführung der Studie}

Die Implementierung der Studie wurde vor Ort in den Schulen durchgeführt. Der Interviewer und der Schreiber trafen sich persönlich mit den Teilnehmern. Beide stellten sich kurz vor und erläuterten den Zweck der zu untersuchenden Software.

Für die Interviews suchten sie sich innerhalb des Raums einen Ort aus, von dem aus sie sowohl den Monitor als auch den Probanden gut beobachten konnten. Während des Interviews stellte der Interviewer sowohl die Fragen aus dem Fragebogen als auch zusätzliche klärende Fragen. Der Notierer hingegen konzentrierte sich hauptsächlich darauf, die Antworten und Beobachtungen zu dokumentieren.

Die Beobachtung der Aufgabenbearbeitung und des Verhaltens der Teilnehmer war sowohl dem Interviewer als auch dem Notierer zugeordnet. Um ein realistisches Szenario zu gewährleisten, wurden keine fachlichen Rückfragen der Studienteilnehmer beantwortet - sie mussten sich auf die Dokumentation und ihre eigenen Ressourcen verlassen, ähnlich wie in einer realen Arbeitsumgebung.

Den Teilnehmern wurde versichert, dass das Ziel der Studie war, die Software und nicht den Anwender zu testen. Das Interview und die Beobachtung fanden gleichzeitig statt und dauerten zwischen einer und drei Stunden.

In der ersten Durchführung mit dem Gymnasium gab es die Besonderheit, dass  in Anlehnung an die Empfehlung des Leitfaden Usability ein erfahrener Requirements Engineer (der Product Owner der Software) mittels Anruf zugeschaltet, der \glqq in Form einer Supervision die Gesprächssituation beobachtet, bewertet und anschließend mit dem Beobachteten [besprochen hat]\grqq{} \cite[p.~133]{dakks}.

In drei Fällen (Gymnasium, Realschule und zwitweise bei der Förderschule) waren bei der Durchführung der Interviews unerwarteterweise zwei Mitarbeiter vonseiten der Studienteilnehmer anwesend. Es wurde die Entscheidung getroffen, das Interview nur mit dem vorab rekrutierten Teilnehmer durchzuführen. Kommentare und Diskussionen zwischen den Mitarbeitern waren jedoch zulässig, um ein realistischeres Szenario zu belassen. Die gesammelten Daten wurden ausschließlich aus den Ansichten, Antworten und Beobachtungen des Interviewpartners erfasst, nicht von dem weiteren Mitarbeiter. Die Interviews wurden in einem Sekretariat mit Sekretariatsmitarbeitern durchgeführt und die Daten direkt in Microsoft Word dokumentiert. Es wurden keine Audio- oder Filmaufnahmen erstellt.

\subsection{Nachbereitung und Auswertung der Studie}

Unmittelbar nach Abschluss der Interviews wurde eine erste Nachbearbeitung der gesammelten Daten durchgeführt. Hierbei wurden unklare oder zu kurz formulierte Notizen präzisiert und ausführlicher beschrieben. Es ist wichtig zu betonen, dass diese Nachbearbeitung nur für Einträge vorgenommen wurde, bei denen eine eindeutige Interpretation möglich war. Bei Notizen, bei denen das Potenzial für Fehlinterpretationen bestand oder die im Nachhinein unklar blieben, wurden keine Änderungen vorgenommen. Sie wurden in ihrer ursprünglichen Form beibehalten, um die Möglichkeit von Missverständnissen zu minimieren.

Für die Auswertung der gesammelten Daten wurden keine speziellen Analysetools oder ähnliche Instrumente verwendet. Dies erfolgte lediglich bei den Notizen, die unmissverständlich waren und bei denen man keine Fehlinterpretationen beim ausformulieren machen konnte. Bei Punkten, die potenziell fehlinterpretiert werden konnten oder unklar im Nachhinein waren, wurden keine Ausformulierungen durchgeführt, sondern die Notiz so belassen wie sie mitgeschrieben wurde. Verständnis der Benutzererfahrungen und -bedürfnisse zu gewinnen.

%todo: Die Ergebnisse der unterschiedlichen Schulen wurden jedoch händisch miteinander auf Gemeinsamkeiten und Kontroversen verglichen. 


\subsection{Materialien und Ressourcen}

Für die Durchführung der Studie waren die Anforderungen an die Infrastruktur relativ minimal. Es war lediglich ein Computer mit Internetzugang erforderlich, um den Zugriff auf die Anwendung zu ermöglichen.\\
Darüber hinaus war kein zusätzliches Material notwendig.

%  Ethik: Welche ethischen Gesichtspunkte wurden bei der Durchführung des Interviews berücksichtigt, z.B. in Bezug auf Datenschutz oder Anonymität?
 
 
% In diesem Kapitel solltest du die Methoden und Techniken, die du zur Durchführung deiner Projektarbeit verwendest, beschreiben und begründen. Hier sollten auch die Einzelheiten zur Durchführung des leitfadengestützten Interviews enthalten sein.
% Für das leitfadengestützte Experteninterview solltest du in der Methodik deiner Projektarbeit den Ablauf und die Durchführung des Interviews beschreiben. Hierbei sollst du beispielsweise folgende Punkte erwähnen:
% Teilnehmer: Wer wurde für das Interview ausgewählt und warum? Wie viele Experten wurden befragt? 

% material
% Wie haben wir den Fragebogen erstellt? 
% Schulleiter zeigt nochmal weitere Aspekte auf
% Offene Fragen gemacht => Warum offene Fragen
% induktives vorgehen
% ich habe jedes Sekretärin vorher telefonisch gesprochen

% Fragebogen ist delegiert ans Team und ein Zwischenergebnis. Ist ein Ergebnis eines Expertenteams für dieses Programm
% Experten sind Material


% ggf. der fixierte Nutzungskontex 
\section{Ergebnisse}

In diesem Kapitel werden die Ergebnisse des Fragebogens und der Beobachtungen bei den Studienteilnehmern beschrieben.

Zunächst werden die Antworten auf die Fragen des Fragebogens und den Beobachtungen, welche vor der Durchführung der drei Aufgaben festgestellt wurden, behandelt. 

Anschließend kommen die Ergebnisse, welche zu Beginn der Durchführung der drei Aufgaben erworben wurden.

Darauf folgen die Befunde, welche während der Bearbeitung der Aufgaben entstanden sind. Dies erfolgt für zunächst jeden Tab, der im Rahmen der Aufgabe 1 bearbeitet wurde welche ein Anlegen eines Datensatzes darstellte. Im nächsten Schritt werden die Erkenntnisse aus den Aufgaben 2 und 3, die eine Bearbeitung von existierenden Datensätzen beinhaltete, zusammenzufassen.

Abschließen wird der Ergebnisteil mit den Resultaten aus den Antworten und Beobachtungen, welche nach der Bearbeitung der Aufgaben ermittelt wurden.

Referenzen zu den Aussagen der Sekretärinnen des Gymnasiums (1), der Realschule (2), der Förderschule (3), der Grundschule (4) und des Berufskollegs (5) werden konsequent den Anhängen 1.1 bis 1.5 zugeordnet.

\subsection{Vor der Durchführung}
\textbf{In welchem Umfang besitzen Sie Vorerfahrungen mit\textit{Schüler Online 1.0}?}\\
Die Sekretärin des Gymnasiums (1) verfügte über praktische und indirekte Erfahrungen mit \textit{Schüler Online}, während die Sekretärinnen der Realschule (2), der Förderschule (3), der Grundschule (4) und des Berufskollegs (5) keine Vorerfahrungen mit der Anwendung hatten.

\textbf{In welchem Umfang besitzen Sie Vorerfahrungen mit der neuen Software?}\\
Die Sekretärin des Gymnasiums (1), der Realschule (2), der Grundschule (4) sowie die des Berufskollegs (5) gaben an, keinerlei vorherige Erfahrung mit dem System zu haben.

Anders äußerte sich die Sekretärin der Förderschule (3). Sie hatte bereits einen ersten Blick auf das System geworfen, die Elternseite eingeschlossen. Sie gab an, dass sie sich bereits auf der Seite angemeldet und eingeloggt hatte.

\textbf{Welche Probleme können bei herkömmlichen, nicht-digitalen Bewerbungen von Schülern auftreten?}\\
Die Sekretärin des Gymnasiums (1) stellte fest, dass sich Eltern beim Einschulungsjahr der Kinder oft verrechnen und dass Schüler gelegentlich versehentlich falsche Daten eingeben, ohne dabei jedoch böswillige Absichten zu haben (E1). 

Demgegenüber äußerte die Sekretärin der Realschule (2) Bedenken hinsichtlich verschiedener Problembereiche: Falsch ausgefüllte oder fehlende Angaben sowie inkorrekte Daten wurden als häufige Fehlerquellen identifiziert (E2). Darüber hinaus wies sie darauf hin, dass bestimmte Fachbegriffe wie \textit{Konfession} oder \textit{Schulformempfehlung} für Schüler möglicherweise nicht verständlich seien (E3). Sie stellte fest, dass einige Schüler absichtlich falsche Angaben machen würden (E4). Dies steht im Gegensatz zur Aussage der Gymnasiums-Sekretärin, die lediglich versehentliche Fehler thematisiert hatte. 

Alle Sekretärinnen unterstrichen die Problematik der Lesbarkeit, da die Formulare in der Regel handschriftlich ausgefüllt werden (E5).

Die Sekretärinnen der Grundschule (4) und des Berufskollegs (5) gaben an, dass unvollständige Einträge (E6) und fehlende Unterschriften ebenfalls zu den häufigsten Schwierigkeiten bei der Dateneingabe zählten.


\textbf{Wer nutzt das System hauptsächlich an Ihrer Schule?}\\
Die Sekretariate aller fünf Schulen (1, 2, 3, 4, 5) würden das System nutzen. Darüber hinaus verwendete bei den Sekretärinnen der Förderschule (3) und des Berufskollegs (5) auch die Verwaltung die Software. Zusätzlich zu Sekretariat und Verwaltung wurden bei der Sekretärin des Berufskollegs (5) die Lehrer einbezogen, die über Aufnahmeentscheidungen bestimmten. E7 

% todo: \textbf{Welche (Teil-)Faktoren bestimmen das Abschließen einer Bewerbung, sodass sie diese nicht mehr bearbeiten müssen?}\\

\textbf{Beschreiben Sie die Ausgangssituation die vorliegt, bevor Sie die Aufgabe \glqq Bewerbung eines Schülers\grqq{} durchführen.}\\
Die Sekretärin der Realschule (2) schilderte, dass die Entscheidung über die Aufnahme eines Kindes auf der Schulempfehlung und vorausgehenden Gesprächen mit der Schulleitung basiere (E8). Mehrfachanmeldungen müssen laut ihr vermieden werden und würden in der Regel bei Kennenlernterminen auffallen (E9). Sie erklärte, dass die Schulen der Region eine spezielle Methode anwendeten, um dies sicherzustellen (mittels farbiger Formulare).

Die Sekretärin der Förderschule (3) gab an, dass das Kind im Schuleinzugsgebiet leben müssen. (E10) Eine Information über die Diagnose Autismus könnte laut ihr oft schon als Entscheidungsgrundlage für die Eignung des Kindes dienen, allerdings könnten in einigen Fällen auch Gutachten erforderlich sein.

\textbf{Welche fachlichen und technischen Qualifikationen sind zur Bewältigung der Aufgabe erforderlich (Aufgabenbewältigung / Softwarenutzung)? Welche Vorkenntnisse fehlen ggf.?}\\
Die Sekretärin des Gymnasiums (1) und der Realschule (2) hoben hervor, dass keine spezifischen fachlichen oder technischen Kenntnisse für die Nutzung von \textit{Schüler Online} erforderlich seien, während die Bereitschaft, sich mit der Software vertraut zu machen, essentiell sei. Die Sekretärin der Förderschule (3) und die der Grundschule (4) unterstrichen die Bedeutung von Datenschutzbewusstsein (E11). Grundlegende Kenntnisse im Schulgesetz und MS Office wurden von der Sekretärin der Grundschule (4) als vorteilhaft, jedoch nicht als notwendig erachtet. Die Sekretärin des Berufskollegs (5) betonte die mögliche Notwendigkeit von Schulungen und das Verständnis für spezifische Bildungsangebote der eigenen Schule.

\subsection{Zu Beginn der Durchführung}
Zu Beginn der Durchführung der Aufgaben mit der neuen Software zeigten sich bei mehreren Sekretärinnen Schwierigkeiten in der Navigation. Die Sekretärin des Gymnasiums (1) und die der Förderschule (3) navigierten auf die falschen Seiten. Sie konnten den Sinn und Zweck der Menüpunkte \glqq Schüler:innen\grqq{} und \glqq Anmeldungen\grqq{} nicht korrekt differenzieren. Obwohl ihre Aufgabe darin bestand, eine Anmeldung zu erfassen, konnten sie aufgrund der bereitgestellten Informationen nicht richtig navigieren. (E12)

\textbf{Welche Arbeitsschritte sind durchzuführen?}\\
Die Sekretärin des Gymnasiums (1) beschrieb den Prozess in drei Schritten: Zuerst müsse der Schüler im System gesucht werden, dann würden die Daten des Schülers erfasst. Anschließend müsse dem Schüler eine Information übermittelt werden.

% Die Sekretärin der Realschule (2) betonte die Bedeutung der persönlichen Anwesenheit des Schülers an der Schule.
Für die Sekretärin der Förderschule (3) umfasste der Prozess mehrere Schritte: Erst müsse geklärt werden, ob ein Rundgang durchgeführt wurde. Danach würden die Daten aufgenommen und das Gutachten für die Akten angefordert. Ein kurzes Kennenlernen (E13) und die Entscheidung, in welche Klasse das Kind käme, gehörten ebenso zu den Arbeitsschritten. 

Die Sekretärin der Grundschule (4) erläuterte, dass die Eltern zunächst ins Sekretariat kämen, wo ein Gespräch stattfände (E14). Anschließend würden die Daten erfasst und Unterlagen eingereicht. Letztendlich würde eine Schülerakte angelegt. (E15)

\textbf{Welche Hilfsmittel sind erforderlich (für die Aufgabenbewältigung / zur Softwarenutzung)? Welche davon fehlen ggf., welche sind zusätzlich gewünscht?}\\
Für die Sekretärin des Gymnasiums (1) sind Kopier- und Einfügefunktionalitäten wichtig, um Daten aus anderen Programmen wie E-Mail oder PDF-Reader übertragen zu können. (E16)

Die Sekretärin der Förderschule (3) nutzt Google Maps zur Adressrecherche als unterstützendes Hilfsmittel. (E17)

Die Sekretärin des Berufskollegs (5) spricht den Bedarf und einer Kurzanleitung für den Einstieg in die Softwarenutzung an. (E18)

\textbf{Welche Ergebnisse / Teilergebnisse entstehen und wie werden diese ggf. verwertet / weitergeführt?}\\
Die Sekretärin des Berufskollegs (5) stellt heraus, dass nach Abschluss des Prozesses alle notwendigen Daten für die Bewerbung des Schülers erfasst sein sollten. Sie betont insbesondere die Wichtigkeit der Erfassung der vorherigen Schule, da es sehr aufwändig wäre, diese Information nachträglich herauszufinden.(E19) 

\textbf{Welche wichtigen Sonderfälle müssen berücksichtigt werden? (bzw. fallen dem Benutzer spontan ein; z. B. zur Arbeitsteilung / Zusammenarbeit)}\\
Die Sekretärin des Gymnasiums (1) weist darauf hin, dass Wechsel innerhalb des Schuljahres und das Wiederholen von Stufen berücksichtigt werden müssen. (E20) Bei Flüchtlingskindern sind möglicherweise nicht alle erforderlichen Daten vorhanden. (E21) Sie stellt hebt besondere Situationen hervor, wie langfristige Beurlaubungen oder Fälle, in denen Kinder, die etwa Therapien unterziehen und in der Klinik unterrichtet werden, dennoch schulpflichtig angemeldet werden müssen. (E22)

Die Sekretärin der Grundschule (4) betont, dass besondere Fälle wie unterschiedliche Staatsangehörigkeiten und bestimmte Sorgerechtssituationen die Anmeldung verkomplizieren können. (E23)

Die Sekretärin des Berufskollegs (5) identifiziert Schüler mit Förderbedarf als Sonderfälle. Sie merkt auch an, dass manche Schüler in ihrem Alter unsicher sind, ob sie sich als männlich oder weiblich identifizieren. Des Weiteren stellt sie fest, dass bei minderjährigen Schülern eine Unterschrift der Eltern erforderlich ist. Ein weiterer Sonderfall sind Schüler, die sich zu spät bewerben und möglicherweise nicht die notwendigen Voraussetzungen für bestimmte Bildungsgänge erfüllen. (E24)

\subsection{Während der Durchführung}
%todo überleitungstext \textbf{Für jeden Tab / Seite: Welche Seiteninhalte sind unverständlich?}\\

\textbf{\glqq Schüler\grqq{}-Tab}\\  
Alle Studienteilnehmer hatten Schwierigkeiten zu verstehen, wofür das \textit{ID-Schlüssel} gedacht ist. (E25)

Die Sekretärin des Gymnasiums (1) konnte den Zweck der Schülersuche-Funktion nicht nachvollziehen. (E26)


\textbf{Anmerkungen zum Tab \glqq Bildungsgang\grqq{}}\\
Für die Sekretärin des Gymnasiums (1) war das Dropdown-\textit{Klasse} unklar, da keine Daten vorhanden waren und noch nicht feststand, welche Klassen es geben würde. Sie wählte den Status \glqq Warteliste\grqq{}, um die Bewerbung als textit{in Bearbeitung} zu markieren und konnte nicht nachvollziehen, warum in der Anwendung vermerkt werden soll, dass eine Aufnahmeberatung stattgefunden hat.

Die Sekretärin der Realschule (2) hätte den Begriff textit{Neuaufnahme} in der Auswahl für den Aufnahmestatus erwartet, da diese Terminologie auch bei \textit{SchILD} verwendet wird. Sie hatte auch Schwierigkeiten, das \textit{Beschulungsbeginn}-Datum korrekt einzugeben.

Die Sekretärin der Förderschule (3) hielt die Sortierung der Schuljahre für nicht sinnvoll und stellte fest, dass man mitunter keine medizinischen Informationen erfassen darf.

Die Sekretärin der Grundschule (4) war verwirrt durch die Meldung, dass ein \textit{Antrag auf vorzeitige Einschulung} stattfindet und fand die Bildungsgangbezeichnungen verwirrend.

Die Sekretärin der Berufsschule (5) hatte Diskussionen mit einer Kollegin, wie das \textit{Schuljahr} zu interpretieren ist. Sie hatte eine andere Meinung als die Kollegin, dass es sich dabei um das aktuelle Schuljahr handelt und nicht um das Schuljahr, das der Schüler zuvor besucht hat. Darüber hinaus war der Anmeldestatus für sie nicht klar interpretierbar.

\textbf{Anmerkungen zum Tab \glqq Persönliche Daten\grqq{}}\\
Die Sekretärin des Gymnasiums (1) trug die Hausnummer zunächst intuitiv zusammen mit der Straße ins Feld \textit{Straße} ein, anstatt ins \textit{Hausnummer}. Sie gab auch die E-Mail-Adresse des Sorgeberechtigten in das \textit{E-Mail}-Feld des Schülers ein. Sie merkte an, dass sie es bevorzugen würde, wenn der Tab \textit{Persönliche Daten} vor dem Tab \textit{Bildungsgang} kommen würde, so wie es bei \textit{SchILD}  der Fall ist. Sie schlug auch vor, dass der Eintrag für \textit{Staat} standardmäßig mit \textit{Deutschland} vorbelegt sein sollte. Sie wünschte sich außerdem einen Hinweis, dass bei Sek1-Schülern grundsätzlich die Telefonnummer der Sorgeberechtigten und bei Sek2-Schülern die des Schülers erwartet wird.

% Die Sekretärin der Realschule (2) fand die Erfassung von Ortsteilen interessant.
Die Sekretärin der Förderschule (3) überprüft unbekannte Straßen mithilfe von Google Maps auf ihre Existenz. Sie schlug vor, dass es sinnvoll wäre, wenn Eltern angeben könnten, zu welchen Zeiten sie telefonisch erreichbar sind. Sie fände es auch gut, wenn mehrere Telefonnummern für Notfälle erfasst werden könnten.

Die Sekretärin der Berufsschule (5) wünschte sich eine Anzeige, die darauf hinweist, ob der Schüler volljährig ist. Sie hatte Schwierigkeiten mit der Suche nach der Postleitzahl-Ort-Kombination.

\textbf{Anmerkungen zum Tab \glqq Sorgeberechtigte\grqq{}}\\
Die Sekretärin des Gymnasiums (1) hielt es für wichtig, einen Nachweis dafür zu haben, ob eine Person das alleinige Sorgerecht hat oder ob beide Unterschriften vorhanden sind, um zu verhindern, dass sich eine Person über die andere hinwegsetzt. Sie hätte erwartet, dass es eine Funktion zur Datenübernahme der Adressdaten aus den persönlichen Daten des Schülers gibt. Sie war auch unsicher über den Zweck des Postfachs. Sie war verärgert, als der Sorgerechtsprozess durch einen versehentlichen Klick neben das Popup abgebrochen wurde, und forderte, dass es nicht möglich sein sollte, Daten zu verlieren, sei es durch Danebenklicken oder Ausloggen.

Die Sekretärin der Realschule (2) trug die Hausnummer auch intuitiv zusammen mit der Straße in das Feld \textit{Straße} ein.

Alle Sekretärinnen außer die Sekretärin des Berufskolleg (5) versuchten, die Straße in das Feld \textit{Adressart} einzugeben. Die Sekretärin der Förderschule (3) erklärte auf Nachfrage, dass sie intuitiv die folgende Eingabereihenfolge erwartet hätte: Staat => Postleitzahl/Ort => Straße.

Der \textit{Erstellen}-Button wurde unterschiedlich interpretiert. Die Sekretärin der Förderschule (3) interpretierte ihn im Sinne von \textit{Speichern}, während die Sekretärin der Hauptschule (4) annahm, dass damit ein weiterer Sorgeberechtigter angelegt werden kann, was sie auf das Plus-Symbol zurückführte.

Die Sekretärin der Berufsschule (5) empfand das Erfassungsformular für die Sorgeberechtigten im Vergleich zu den anderen Formularen als unattraktiv und unübersichtlich.

\textbf{Anmerkungen zum Tab \glqq Notfallkontakte\grqq{}}\\
Die Sekretärin der Realschule (2) schlug vor, dass es hilfreich wäre, die Art der Telefonnummer (im Sinne von dienstlich, privat, mobil) für die Notfallkontakte hinterlegen zu können und mehrere Telefonnummern einer Person zuzuordnen. Sie konnte zeitweise nicht erkennen, wie sie den Datensatz speichern kann, da das Autocomplete-Dropdown-Feld den Speichern-Button versteckte. Sie nahm dieses Feld nicht als ein Feld wahr, das man benutzerdefiniert ausfüllen kann, sondern als ein Feld, in dem man zwingend einen der vorgeschlagenen Werte auswählen muss.

Die Sekretärin der Förderschule (3) versuchte, mehrere Nummern in ein Feld einzugeben, was ihr nicht gelang. Sie wies auch darauf hin, dass es wichtig sei, die Rolle des Notfallkontakts zu erfassen, um im Notfall beurteilen zu können, ob sensible Informationen weitergegeben werden dürfen. Eine Rangfolge der Notfallkontakte hielt sie nicht für sinnvoll. Sie hätte erwartet, die Notfallkontakte früher im Prozess zu sehen.

Die Sekretärin der Hauptschule (4) empfand die Eingabe der Notfallkontakte als unpraktisch, da sie die gleichen Daten wie bei den Sorgeberechtigten eingeben musste. In \textit{SchILD} werden diese Daten automatisch übernommen.

\textbf{Anmerkungen zum Tab \glqq Migrationshintergrund\grqq{}}\\
Die Sekretärin des Gymnasiums (1) äußerte Zufriedenheit mit dem Abschnitt zum Migrationshintergrund und betonte, dass alle wichtigen Informationen vorhanden seien. Sie war jedoch der Meinung, dass das Eingabefeld angezeigt werden sollte, selbst wenn die Option \textit{Migrationshintergrund liegt nicht vor\grqq} ausgewählt ist.

Die Sekretärin der Gesamtschule (5) stellte die Definition von \textit{Migrationshintergrund} in Frage. Sie war sich unsicher und spekulierte darüber, ab wann ein Migrationshintergrund vorliegt.

\textbf{Anmerkungen zum Tab \glqq Letzte Tätigkeit\grqq{}}\\
Die Sekretärin des Gymnasiums (1) kritisierte die Reihenfolge der Bereiche und schlug vor, dass die \textit{Letzte Tätigkeit} innerhalb des Abschnitts \textit{Schüler-Daten} erfasst werden sollte, da es sich hierbei um ergänzende Informationen zur Person handelt. Sie hatte Schwierigkeiten mit der Suche nach der letzten Schule, da die Suche eine Checkbox überdeckte und die Sortierung der Ergebnisse ihrer Meinung nach die lokale Umgebung bevorzugen sollte.

Die Sekretärin der Gemeinschaftsschule (3) hielt den Tab \textit{Letzte Tätigkeit} für überflüssig, da ein Kind, das sich für die Grundschule anmeldet, noch keinen Schulabschluss haben kann. Sie war verwirrt und konnte nicht unterscheiden, ob die letzte Tätigkeit des Schülers oder des Sorgeberechtigten erfragt wird. Die angestrebte Schulstufe wurde von ihr als letzte Schulstufe interpretiert.

Die Sekretärin der Gesamtschule (5) hatte Schwierigkeiten, eine einmal ausgewählte Grundschulempfehlung wieder zu entfernen. Sie wies auch darauf hin, dass sich das \textit{Angestrebtestufe} und mit dem Tab \textit{Letzte Tätigkeit} einander zu widersprechen scheinen.

\textbf{Anmerkungen zum Tab \glqq Qualifikationen\grqq{}}\\
Die Sekretärin der Gesamtschule (5) stellte fest, dass in der Liste der Qualifikationen einige Schulabschlüsse, wie beispielsweise der Hauptschulabschluss, fehlten. Sie äußerte Kritik und regte eine Überarbeitung der Auswahlmöglichkeiten an.

\textbf{Anmerkungen zum Tab \glqq Termine\grqq{}}\\
Sekretärin 5 konnte den Zweck der Seite \textit{Termine} nicht erkennen. Sie gab an, dass sie ohnehin keine Zeit hätte, 500 Personen zu Terminen einzuladen. Sie merkte jedoch an, dass einige Bildungsgangleitungen vorab Gespräche mit den Schülern führen möchten.

\textbf{Anmerkungen zum Tab \glqq Bemerkungen\grqq{}}\\
Sekretärin 2 äußerte Unsicherheit darüber, wer Zugriff auf die internen Notizen hat. Sie definierte die Felder \textit{Interne} und \textit{Bemerkung für Schüler*in} als \glqq Interne Notizen sind unsere Meinung, Bemerkungen sind Fakten\grqq{}. Sekretärin 4 war sich ebenfalls unsicher, welche Informationen dem Schüler zugänglich sind. Im Gegensatz zu 2 interpretierte 4, dass die \textit{Interne Notiz} nur für die Schule sichtbar ist und das \textit{Bemerkung}-Feld auch vom Kind gesehen werden kann, fühlte sich aber durch den Infotext oben im Formular verwirrt.

Sekretärin 3 machte darauf aufmerksam, dass sensible Informationen, wie zum Beispiel Behinderungen, möglicherweise nicht erfasst werden dürfen. Sie betonte die Notwendigkeit einer Benutzerführung, die das Bewusstsein für gesetzliche Anforderungen stärkt.

Sekretärin 5 war überrascht, dass dieser Tab erschien, da sie zuvor nicht bemerkt hatte, wie weit sie im Prozess fortgeschritten war.Dieser Tab wurde zuvor aufgrund des Overflow-Verhaltens des Browsers verdeckt.

Sekretärin 1 gab an, dass sie den Tab \textit{Bemerkungen} an einer früheren Stelle im Prozess erwartet hätte.

\textbf{Anmerkungen zum Tab \glqq Zusammenfassung\grqq{}}\\
Sekretärin 1 erlebte eine unklare Fehlermeldung beim Klick auf \textit{Speichern}. Sie wünscht sich klarere Anweisungen zur Fehlerbehebung und einen direkten Fokus auf das fehlerhafte Feld.
Sekretärin 2 stellte eine Verzögerung beim Laden der Inhalte fest. Sie befürchtet, dass die per E-Mail gesendete Bestätigungsmeldung mit der Formulierung \textit{Angemeldet} falsche Erwartungen bei den Schülern wecken könnte. Sie war sich auch unsicher, ob das Schuljahr korrekt erfasst wurde und merkte an, dass noch eine Unterschrift von den Eltern für die Anmeldung erforderlich ist.
Sekretärin 3 interpretierte die Meldung auf der Zusammenfassungsseite als \glqq Die Mutter wurde kontaktiert\grqq{}.
Sekretärin 4 nahm an, dass die Anmeldung des Kindes als \textit{vorzeitige Einschulung} gekennzeichnet wurde, was sie als Fehler ansah. Sie war sich auch unsicher, ob sie nun eine \glqq Anmeldung Bewerbung\grqq{} eingereicht hat. Sie bemerkte, dass Informationen zu wichtigen Themen wie dem Nachweis des Masernschutzes oder der Betreuung für beispielsweise die offene Ganztagsschule fehlten.
Sekretärin 5 kritisierte den Informationsgehalt der Seite. Ihrer Meinung nach sind zu viele Informationen im Panel \textit{Bildungsgang} enthalten, insbesondere das Kürzel des Bildungsgangs, das für sie schwer zu verstehen war.

\subsection{Nach der Durchführung:}

\textbf{Konnten Sie die Aufgabe aus Ihrer Sicht erfolgreich und vollständig abschließen? Falls nein - was hat Sie daran gehindert?}\\
Sekretärin 1 gab an, dass sie die Aufgabe erst beim zweiten Anlauf erfolgreich abschließen konnte. Sie empfand die Aufgabe zwischendurch als unübersichtlich und hätte sich eine Hilfeoption oder ein Handbuch gewünscht. Sie bemerkte auch, dass sie ungern prüfen würde, welche Anmeldungen neu eingegangen sind und wünschte sich Ziffern oder eine andere Hervorhebung, die auf neue, unbearbeitete Anmeldungen hinweist.
Sekretärinnen 2, 3 und 4 konnten die Aufgaben alle erfolgreich speichern und abschließen. Sekretärin 4 meinte, dass sich Fragen zur Software im Laufe der Zeit von selbst klären könnten, sobald sie etwas geübter ist.
Sekretärin 5 konnte die Aufgabe nicht abschließen, da die Aufnahmeentscheidung in der Verantwortung der Abteilungsleiter liegt und sie somit auf diese angewiesen ist. Sie ist jedoch zuversichtlich, dass sie mit etwas Übung genau weiß, worauf sie achten muss.

\textbf{Wie effektiv unterstützt die Webanwendung Sie bei der Aufgabe?  Gab es positive oder negative Erfahrungen?}\\
Sekretärin 1 schätzt die neue Software als besser und übersichtlicher ein als die vorherige. Sie bemängelt jedoch, dass Felder mit Validierungsproblemen erst bei der Speicherung markiert werden und nicht direkt, wenn das Problem auftritt.

Sekretärin 2 hofft, dass die Bewerbungsdaten nun in das SchILD-Programm übertragen werden können. Sie würde gerne \glqq Standardinformationen\grqq{} , die stets zu einer Bewerbung erfasst werden müssen, einfach eingeben können. Beispiele hierfür sind, ob das Kind im Schuljahr fotografiert werden darf oder ob es ein Schwimmabzeichen hat. Sie findet die verwendeten Farben in der Anwendung ansprechend.

Sekretärin 3 bezeichnet die Umsetzung der Aufgaben als \glqq relativ intuitiv\grqq{} .

Sekretärin 4 fand die Umsetzung der Aufgaben \glqq nicht schwierig\grqq{}  und \glqq überschaubar\grqq{} , obwohl es bei der letzten Tätigkeit zu Irritationen kam.

\textbf{Haben Sie sämtliche Inhalte der Aufgabe verstanden? Gab es Stellen, an denen Sie sich mehr Unterstützung gewünscht hätten?}\\
Die Sekretärin einer Förderschule (3) äußerte den Bedarf, den Unterschied zwischen den Menüpunkten \textit{Schüler:innen} und \textit{Anmeldungen} klarer zu gestalten. Die Sekretärin einer Grundschule (4) verzeichnete zwar Fragen, gab jedoch keine weiteren Probleme an. Die Sekretärin eines Berufskollegs (5) bestätigte, dass die Aufgabenstellung für sie verständlich war, merkte jedoch an, dass ihre Erfahrung als Sekretärin möglicherweise dazu beigetragen hat.

\textbf{Gab es Schwierigkeiten oder Verwirrungen bei der Aufgabe? Wenn ja, welche?}\\
Die Sekretärin eines Gymnasiums (1) begegnete keinen besonderen Schwierigkeiten oder Verwirrungen, abgesehen von der bereits angesprochenen Unklarheit hinsichtlich der Menüpunkte, die auch die Sekretärin einer Förderschule (3) wahrgenommen hatte.

Die Sekretärin einer Realschule (2) fand die Felder im Zusammenhang mit dem Export im Update-Prozess verwirrend. Außerdem bemängelte sie die Position des Speichern-Buttons bei den Sorgeberechtigten, welcher ihrer Meinung nach oben platziert hätte sein sollen, ähnlich wie die anderen Speichern-Buttons.

Die Sekretärinnen einer Förderschule (3) und Grundschule (4) gaben an, keine Schwierigkeiten oder Verwirrungen gehabt zu haben.

Die Sekretärin eines Berufskollegs (5) äußerte den Wunsch, die Bezeichnungen innerhalb der Software anzupassen. Ihrer Meinung nach sollte die Bezeichnung \textit{Letzte Tätigkeit}  durch \textit{Bisherige Schullaufbahn}  ersetzt werden.

\textbf{Wie verständlich waren die Rückmeldungen der Anwendung?}\\
Die Sekretärin eines Gymnasiums (1) äußerte, dass es nützlich wäre, wenn die Anwendung Rückmeldung darüber gibt, ob ein Datensatz vollständig und korrekt ausgefüllt ist. Zudem betrachtete sie die Rückmeldungen als \glqq ausbaufähig\grqq{}  und forderte mehr Feedback mit konkreten Anweisungen.

Die Sekretärin einer Förderschule (3) bemängelte das Fehlen eines Hinweises, auf welchem Reiter ein Fehler vorliegt.

Die Sekretärin einer Grundschule (4) fand die Rückmeldungen der Anwendung bis auf den Hinweis zur \glqq vorzeitigen Einschulung\grqq{}  verständlich.

Die Sekretärin eines Berufskollegs (5) sprach den Wunsch aus, dass es neben den Feldern kleine Erklärungen geben sollte, um die Bedienung der Anwendung zu erleichtern.

\textbf{Welche Fähigkeiten setzt die Anwendung Ihrer Einschätzung nach voraus?}\\
Die Sekretärin eines Gymnasiums (1) war der Ansicht, dass die Aufgaben von jedem erledigt werden könnten, da die Begrifflichkeiten auch für Laien verständlich seien.

Die Sekretärin einer Realschule (2) betonte, dass grundlegende Fertigkeiten wie Lesen und Schreiben sowie die Bedienung einer Tastatur und Maus ausreichend seien. Sie fügte hinzu, dass ein Grundverständnis des Schulgesetzes vorteilhaft wäre. Ihrer Meinung nach könnten sogar Praktikanten die Aufgaben erledigen, allerdings sollte eine Überprüfung erfolgen.

Die Sekretärin einer Förderschule (3) sah als Voraussetzung, dass man der deutschen Sprache mächtig sein sollte. Lesefähigkeit allein reiche aus.

Die Sekretärin einer Grundschule (4) betonte, dass Vorkenntnisse in der EDV und eine kaufmännische Ausbildung hilfreich wären. Sie fügte hinzu, dass das weitere Wissen aus dem alltäglichen Arbeitsablauf gewonnen wird.

Die Sekretärin eines Berufskollegs (5) äußerte, dass Lesefähigkeit eine zentrale Voraussetzung darstellt. Darüber hinaus sollten die Nutzer über die eigenen Bildungsgänge der Schule Bescheid wissen und den normalen Schulwerdegang kennen.

\textbf{Wie sehr entspricht die Umsetzung in der Software der Realität?}\\
Die Sekretärin eines Gymnasiums (1) schlug vor, die Erfassung der Bewerbung analog zu den Dokumenten der Schule oder vergleichbaren Software (wie \textit{SchILD}) zu gestalten, um eine effizientere Datenübertragung zu ermöglichen. Sie fand jedoch, dass das Programm insgesamt der Realität entspricht. Sowohl sie (1) als auch die Sekretärin einer Realschule (2) bestätigten, dass die Software inhaltlich \textit{SchILD}  entspricht.

Die Sekretärin einer Realschule (2) bemerkte, dass die Notizen zu den neuen Schülern der fünften Klasse in einer Liste zusammengeführt werden sollten. Sie bewertete es positiv, dass die Kinder bzw. Eltern über die Anmeldung informiert werden und dass es vier Arbeitsschritte in einem sind. Sie regte an, dass das Programm der abgebenden Schule mitteilen sollte, dass der Schüler aufgenommen wurde, und wies auf mögliche Probleme bei Mehrfachanmeldungen hin.

Die Sekretärin einer Förderschule (3) bemerkte, dass in der Regel die Eltern den Schüler anmelden und fand es praktisch, wenn die Kommunen die Anmeldung vorbereiten würden.

Die Sekretärinnen einer Grundschule (4) und eines Berufskollegs (5) waren der Meinung, dass das Programm der Realität entspricht. Die Sekretärin eines Berufskollegs (5) hoffte jedoch, dass Anmeldezeiträume berücksichtigt werden würden.

\textbf{Was handhaben Sie in Ihrem Arbeitsalltag bei nicht-digitalen Bewerbungen gewöhnlicherweise anders als in der Anwendung?}\\
Die Sekretärin eines Gymnasiums (1) berichtete, dass bei fehlenden unwichtigen Daten wie dem Geburtsort im Notfall improvisiert werde, während wichtige Daten stets mit Urkunden ausgefüllt werden müssten.

Die Sekretärin einer Realschule (2) gab an, dass sie Postleitzahlen und Orte manuell validiere und Kilometerangaben händisch berechne. Sie äußerte den Wunsch nach einer automatischen Berechnung dieser Angaben. Zudem teilte sie mit, dass die Lehrkräfte die Daten der Schülerinnen und Schüler ausgedruckt erhalten möchten, weshalb sie die Möglichkeit zum Ausdruck in der Anwendung vermisste.

Für die Sekretärin einer Förderschule (3) würde die Anwendung lediglich ihre Excel-Tabelle ersetzen.

Die Sekretärin einer Grundschule (4) erklärte, dass sie in ihrem Arbeitsalltag nichts anders handhabe als in der Anwendung.

Die Sekretärin eines Berufskollegs (5) fügte hinzu, dass sie normalerweise noch Dokumente wie Zeugnisse, Lebensläufe und Nachweise über den Masernschutz hinzufüge, um den Lehrkräften eine Entscheidung über die Aufnahme zu ermöglichen.

\textbf{Was würde Sie noch daran hindern die Software in Ihrem Arbeitsalltag einzusetzen?}\\
Die Gymnasiumssekretärin (1) äußerte Bedenken hinsichtlich des zusätzlichen Aufwands, der durch das Erlernen einer weiteren Software entstehen würde. Sie betrachtete die Anwendung als eine Art Dopplung zu \textit{SchILD} und wünschte sich die Möglichkeit, eine E-Mail-Adresse angeben zu können, um in einem wählbaren Intervall Neuigkeiten über die Anwendung zu erhalten.

Die Realschulsekretärin (2) gab an, dass noch weitere, derzeit nicht erfasste Informationen benötigt würden, darunter Angaben zu iPad-Mietverträgen und Kaufoptionen. Sie merkte zudem an, dass die E-Mail-Adresse korrekt sein und überprüft werden müsse. Die Integration von Schulbuchbestellungen und Materiallisten in die versendete E-Mail wäre aus ihrer Sicht ebenso wünschenswert, wie die Erfassung von Noten ausgewählter Fächer. Des Weiteren merkte sie an, dass eine Geburtsurkunde vorliegen müsse und die Datenschutzerklärung der Schule vom Schüler bzw. Elternteil gelesen und akzeptiert werden sollte.

Die Förderschulsekretärin (3) betonte, dass die Anwendung einfacher zu bedienen sein müsse als ihre aktuelle Excel-Tabelle und dass die Eltern sich am Prozess beteiligen müssten.

Die Grundschulsekretärin (4) nannte die notwendige Umstellung und den damit verbundenen Zeitaufwand als Hürden. Eine schnelle Erlernbarkeit der Anwendung würde diesen Aufwand aus ihrer Sicht jedoch reduzieren.

Für die Berufskollegssekretärin (5) gab es keine Hindernisse für den Einsatz der Software im Alltag. Sie war der Ansicht, dass der Erfassungsprozess in Kombination mit der neuen \textit{SchILD}-Software \glqq sehr einfach\grqq{}  sein würde.

%\textbf{Wie sehr erleichtert Ihnen die Anwendung ihre Arbeit?}\\

\textbf{Gibt es Funktionen, die Sie in ähnlichen bzw. anderen Anwendungen genutzt haben, die Sie hier vermissen?}\\
Die Gymnasiumssekretärin (1) vermisst eine Funktion, die sie aus der \textit{SchILD}-Software kennt - die Filterung von Schulen.

Die Grundschulsekretärin (4) würde sich eine Funktion wünschen, mit der sie Berichte oder ähnliches erstellen kann.

Die anderen befragten Sekretärinnen (2, 3, und 5) haben keine fehlenden Funktionen aus ähnlichen oder anderen Anwendungen erwähnt.

\textbf{Welche Software sollte man aus Ihrer Sicht in Schüler Online integrieren bzw. eine Schnittstelle schaffen?}\\
Alle Sekretärinnen (1,2,3) sprachen sich einstimmig für eine Integration mit Schulverwaltungsprogrammen wie \textit{SchILD} aus. Des Weiteren betonte die Sekretärin der Realschule (2) die Wichtigkeit einer Verbindung zu ISERV.

Ein weiterer Punkt, der von den Sekretärinnen der Förderschule, der Grundschule und des Berufskollegs (3,4,5) hervorgehoben wurde, ist die Möglichkeit, einen Export nach Excel durchführen zu können.

\textbf{Welche Dokumente würden Sie gerne im Prozess ausdrucken können?}\\
Die Sekretärin des Gymnasiums (1) lobte die Möglichkeit, das Anmeldeformular ausdrucken zu können für Archivierungszwecke. Sie wünschte jedoch zusätzlich die Druckfunktionalität für Formulare wie Schweigepflichtsentbindungen, Einverständniserklärungen für PKW/Bulli Mitfahrten und Fotoerlaubnisse für die Website.

Die Realschulsekretärin (2) stellte die Forderung auf, ein Schülerstammblatt sowie jährliche Notenübersichten drucken zu können für die Schülerakte.

Für die Sekretärin der Förderschule (3) war es wichtig, zahlreiche Dokumente wie Bewerbungsformulare, Formulare für Essensverpflegung, Informationen zum Schülerspezialverkehr, Notfallkontaktbögen und Fotoerlaubnisse ausdrucken zu können.

Die Sekretärin der Grundschule (4) hätte gerne die Option, verschiedene Listen und Formulare auszudrucken. Darunter fielen ein Schülerstammblatt, Klassenlisten, Listen von Buskindern, eine Auflistung von Kindern mit Migrationshintergrund und Förderbedarf, eine Liste sortiert nach Nationalitäten, ein Formular über den Religionsunterricht und einen Gesamtüberblick über die mit dem Bus fahrenden Kinder einschließlich ihrer Bushaltestellen.

Die Sekretärin des Berufskollegs (5) wünschte sich die Druckfähigkeit für Schülerstammdatenblätter und eine Übersicht über die aufgenommenen Schülerinnen und Schüler.

\textbf{Welche Dokumente würden Sie gerne einscannen wollen und beim Datensatz hinterlegen?}\\
Die Sekretärin des Gymnasiums (1) äußerte keinen Bedarf an der Funktion, Dokumente zu scannen und im System zu hinterlegen.

Im Gegensatz dazu wünschte sich die Realschulsekretärin (2) die Möglichkeit, alle für die Anmeldung relevanten Dokumente, insbesondere das Anmeldeformular, digitalisieren und speichern zu können.

Die Förderschulsekretärin (3) äußerte ebenfalls den Wunsch, das Anmeldeformular digitalisieren und hinterlegen zu können.

Die Grundschulsekretärin (4) fand es wichtig, digitale Personalakten, Sorgerechtsbescheide und Anmerkungen zu aktuellen Gegebenheiten wie Kuraufenthalte scannen und speichern zu können.

Die Berufskollegsekretärin (5) äußerte den Wunsch, Zeugnisse und Lebensläufe scannen und im System hinterlegen zu können.
\input{kapitel/diskussionv2}
\newpage
\section{Eigenständigkeitserklärung}
% Hiermit erkläre ich, Lukas Wessel, dass ich die vorliegende Projektarbeit mit dem Titel \glqq Neugestaltung von \textit{Schüler Online}: Eine Beobachtungs- und Interviewstudie zur Identifikation von Problemstellen und Nutzerbedürfnissen, um die Effektivität sowie die Zufriedenstellung des Schulpersonals beim Erfüllen von Kernaufgaben der Webanwendung zu optimieren\grqq{} selbstständig und ohne fremde Hilfe verfasst habe. Alle verwendeten Quellen und Hilfsmittel sind im Literaturverzeichnis vollständig und korrekt angegeben.

% Sämtliche Textstellen, die wörtlich oder sinngemäß aus veröffentlichten oder nicht veröffentlichten Schriften entnommen wurden, sind als solche kenntlich gemacht. Die Arbeit hat in gleicher oder ähnlicher Form noch keiner Prüfungsbehörde vorgelegen.

% Ich bin mir bewusst, dass eine falsche Erklärung rechtliche Konsequenzen haben wird.
% \\\\

% Ort, Datum: 
% \\\\

% Unterschrift: 
\begin{figure}[H]
    \centering
    \begin{adjustbox}{width=\linewidth, center}
        \includegraphics{eigenstaendigkeitserklaerung}
    \end{adjustbox}
\end{figure}



%\appendix
%
\newpage
\section{Anhang}

\subsection{Musteranmeldeformular}
\label{section-Musteranmeldeformular}
\begin{figure}[H]
    \centering
    \caption{Musteranmeldeformular von der fiktiven Person Max Müller}
    \begin{adjustbox}{height=0.85\textheight, center}
        \includegraphics{bewerbungsformular1}
    \end{adjustbox}
    \label{fig:anmeldeformular}
\end{figure}

\subsection{fragebogen}
\label{section-fragebogen}
\textbf{Vor der Durchführung}
\begin{itemize}
    \item In welchem Umfang besitzen Sie Vorerfahrungen mit\textit{Schüler Online 1.0}?
    \item In welchem Umfang besitzen Sie Vorerfahrungen mit der neuen Software?
    \item Welche Probleme können bei herkömmlichen, nicht-digitalen Anmeldungen von Schülern auftreten?
    \item Wer nutzt das System hauptsächlich an Ihrer Schule?
    \item Beschreiben Sie die Ausgangssituation die vorliegt, bevor Sie die Aufgabe \glqq Anmeldung eines Schülers\grqq{} durchführen.
    \item Welche fachlichen und technischen Qualifikationen sind zur Bewältigung der Aufgabe erforderlich (Aufgabenbewältigung / Softwarenutzung)? Welche Vorkenntnisse fehl
\end{itemize}
\textbf{Unmittelbar vor der Durchführung}
\begin{itemize}
    \item Welche Arbeitsschritte sind durchzuführen?
    \item Welche Hilfsmittel sind erforderlich (für die Aufgabenbewältigung / zur Softwarenutzung)? Welche davon fehlen ggf., welche sind zusätzlich gewünscht?
    \item Welche Ergebnisse / Teilergebnisse entstehen und wie werden diese ggf. verwertet / weitergeführt?
    \item Welche wichtigen Sonderfälle müssen berücksichtigt werden? (bzw. fallen dem Benutzer spontan ein; z. B. zur Arbeitsteilung / Zusammenarbeit)
\end{itemize}
\textbf{Nach der Durchführung}
\begin{itemize}
    \item Konnten Sie die Aufgabe aus Ihrer Sicht erfolgreich und vollständig abschließen? Falls nein - was hat Sie daran gehindert?
    \item Wie effektiv unterstützt die Webanwendung Sie bei der Aufgabe?  Gab es positive oder negative Erfahrungen?
    \item Haben Sie sämtliche Inhalte der Aufgabe verstanden? Gab es Stellen, an denen Sie sich mehr Unterstützung gewünscht hätten?
    \item Gab es Schwierigkeiten oder Verwirrungen bei der Aufgabe? Wenn ja, welche?
    \item Wie verständlich waren die Rückmeldungen der Anwendung?
    \item Welche Fähigkeiten setzt die Anwendung Ihrer Einschätzung nach voraus?
    \item Wie sehr entspricht die Umsetzung in der Software der Realität?
    \item Was handhaben Sie in Ihrem Arbeitsalltag bei nicht-digitalen Anmeldungen gewöhnlicherweise anders als in der Anwendung?
    \item Was würde Sie noch daran hindern die Software in Ihrem Arbeitsalltag einzusetzen?
    \item Wie sehr erleichtert Ihnen die Anwendung ihre Arbeit?
    \item Gibt es Funktionen, die Sie in ähnlichen bzw. anderen Anwendungen genutzt haben, die Sie hier vermissen?
    \item Welche Software sollte man aus Ihrer Sicht in Schüler Online integrieren bzw. eine Schnittstelle schaffen?
    \item Welche Dokumente würden Sie gerne im Prozess ausdrucken können?
    \item Welche Dokumente würden Sie gerne einscannen wollen und beim Datensatz hinterlegen?
\end{itemize}


\subsection{InterviewGymnasium}
\label{section-InterviewGymnasium}
\input{InterviewGymnasium}

\subsection{InterviewRealschule}
\label{section-InterviewRealschule}
\input{InterviewRealschule}

\subsection{InterviewFoerderschule}
\label{section-InterviewFoerderschule}
\input{InterviewFoerderschule}

\subsection{InterviewGrundschule}
\label{section-InterviewGrundschule}
\input{InterviewGrundschule}

\subsection{InterviewBerufskolleg}
\label{section-InterviewBerufskolleg}
\input{InterviewBerufskolleg}

\begin{landscape}

    \begin{longtable}{p{15cm}cc}
        \caption{Your Table} \label{tab:mytable} \\
        \toprule
        Erfordernis: Der Benutzer muss... & Zugehörige Ergebnisse \\
        \midrule
            ... inkorrekte Daten identifizieren und korrigieren können. & E1, E2 \\
            ... die an ihn eingereichten Formulare korrekt übertragen können. & E5, E6 \\
            ... unzulässige Bewerbungen identifizieren können. & E9, E10 \\
            ... die Daten datenschutzkonform in die Anwendung eintragen können und über mögliche Verstöße informiert werden. & E11 \\
            ... erkennen können, wie er zum korrekten Prozess gelangt. & E12 \\
            ... bezüglich Aufnahmeentscheidungen mit den Entscheidungsträgern zusammen arbeiten können. & E7, E8 \\
            ... die Software auch bei fehlenden Daten bedienen können.  & \\
            ... Aufnahmekriterien berücksichtigen können. & \\
            ... Termine für Aufnahmeberatungsgespräche hinterlegen können. & b \\
            ... Schülerakten erzeugen können. & b \\
            ... Daten aus anderen Programmen übernehmen können. & b \\
            ... Adressrecherchen durchführen können. & b \\
            ... eine Kurzanleitung für den Einstieg abrufen können. & b \\
            ... die Software auch bei fehlenden Daten bedienen können. & b \\
            ... innerjährige Wechsel und Stufenwiederholungen erfassen können. & b \\
            ... langfristige Beurlaubungen vermerken können. & b \\
            ... auch komplizierte Bewerbungen und Sonderfälle bearbeiten können. & b \\
            ... seinen bisherigen Jargon verwenden können. & b \\
            ... die Aufgaben und Prozesse intuitiv bedienen können. & b \\
            ... eine Adressvalidierung vornehmen können. & b \\
            ... Erreichbarkeiten von Notfallkontakten erfassen können. & b \\
            ... unterschiedliche Arten von Notfallkontakten erfassen können. & b \\
            ... erkennen können, ob ein Schüler volljährig ist. & b \\
        \endfirsthead
        \toprule
        Erfordernis & Beispiel & Zugehörige Ergebnisse \\
        \midrule
        \endhead
        \bottomrule
        \endfoot
            ... Nachweise über das Sorgerecht hinterlegen können. & b \\
            ... Daten auch bei Dialogabbrüchen wiederherstellen können. & b \\
            ... ähnliche Datenabfragen aus vorherigen Formularen übernehmen können. & b \\
            ... bei Bedarf Handbücher heranziehen können. & b \\
            ... bei Bedarf Fachterminologie nachschlagen oder verstehen können. & b \\
            ... jederzeit darüber Bescheid wissen, welche Daten den Eltern und Schülern angezeigt werden. & b \\
            ... den Systemstatus jederzeit einsehen können. & b \\
            ... Bewerbungslisten in geeigneter Form exportieren können. & b \\
            ... Daten in Schulverwaltungsprogramme wie Schild exportieren können. & b \\
            ... Daten nach Excel exportieren können. & b \\
            ... Berichte anfertigen können. & b \\
            ... Anmeldezeiträume berücksichtigen. & b \\
            ... mit anderen Behörnden zusammenarbeiten können. & b \\
            ... über Änderungen an Bewerbungen regelmäßig informiert werden. & b \\
            ... zusätzliche Informationen zu einer Bewerbung hinterlegen können. & b \\
            ... Dokumente ausdrücken können. & b \\
            ... Dokumente digitalisieren und beim Datensatz hinterlegen können. & b \\
    \label{tab:anmeldeformular}
\end{longtable}

\subsection{bildungsgang}
\label{section-bildungsgang}
\begin{figure}[H]
    \centering
    \caption{Testüberschrift}
    \begin{adjustbox}{width=\linewidth, center}
        \includegraphics{bildungsgang}
    \end{adjustbox}
\end{figure}

\subsection{sorgeberechtigte-liste}
\label{section-sorgeberechtigte-liste}
\begin{figure}[H]
    \centering
    \caption{Testüberschrift}
    \begin{adjustbox}{width=\linewidth, center}
        \includegraphics{sorgeberechtigte-liste}
    \end{adjustbox}
\end{figure}

\subsection{sorgeberechtigter-person}
\label{section-sorgeberechtigter-person}
\begin{figure}[H]
    \centering
    \caption{Testüberschrift}
    \begin{adjustbox}{width=0.5\linewidth, center}
        \includegraphics{sorgeberechtigter-person}
    \end{adjustbox}
\end{figure}

\subsection{sorgeberechtigter-anschrift}
\label{section-sorgeberechtigter-anschrift}
\begin{figure}[H]
    \centering
    \caption{Testüberschrift}
    \begin{adjustbox}{width=0.85\linewidth, center}
        \includegraphics{sorgeberechtigter-anschrift}
    \end{adjustbox}
\end{figure}

\subsection{sorgeberechtigter-kontakt}
\label{section-sorgeberechtigter-kontakt}
\begin{figure}[H]
    \centering
    \caption{Testüberschrift}
    \begin{adjustbox}{width=0.6\linewidth, center}
        \includegraphics{sorgeberechtigter-kontakt}
    \end{adjustbox}
\end{figure}

\subsection{notfallkontakt-daten}
\label{section-notfallkontakt-daten}
\begin{figure}[H]
    \centering
    \caption{Testüberschrift}
    \begin{adjustbox}{width=0.6\linewidth, center}
        \includegraphics{notfallkontakt-daten}
    \end{adjustbox}
\end{figure}

\subsection{notfallkontakt-liste}
\label{section-notfallkontakt-liste}
\begin{figure}[H]
    \centering
    \caption{Testüberschrift}
    \begin{adjustbox}{width=\linewidth, center}
        \includegraphics{notfallkontakt-liste}
    \end{adjustbox}
\end{figure}

\subsection{migrationshintergrund-liegtvor}
\label{section-migrationshintergrund-liegtvor}
\begin{figure}[H]
    \centering
    \caption{Testüberschrift}
    \begin{adjustbox}{width=\linewidth, center}
        \includegraphics{migrationshintergrund-liegtvor}
    \end{adjustbox}
\end{figure}

\subsection{migrationshintergrund-liegtnichtvor}
\label{section-migrationshintergrund-liegtnichtvor}
\begin{figure}[H]
    \centering
    \caption{Testüberschrift}
    \begin{adjustbox}{width=\linewidth, center}
        \includegraphics{migrationshintergrund-liegtnichtvor}
    \end{adjustbox}
\end{figure}

\subsection{qualifikation}
\label{section-qualifikation}
\begin{figure}[H]
    \centering
    \caption{Testüberschrift}
    \begin{adjustbox}{width=\linewidth, center}
        \includegraphics{qualifikation}
    \end{adjustbox}
\end{figure}

\subsection{letztetaetigkeit}
\label{section-letztetaetigkeit}
\begin{figure}[H]
    \centering
    \caption{Testüberschrift}
    \begin{adjustbox}{width=\linewidth, center}
        \includegraphics{letztetaetigkeit}
    \end{adjustbox}
\end{figure}

\subsection{aufnahmeberatung}
\label{section-aufnahmeberatung}
\begin{figure}[H]
    \centering
    \caption{Testüberschrift}
    \begin{adjustbox}{width=\linewidth, center}
        \includegraphics{aufnahmeberatung}
    \end{adjustbox}
\end{figure}

\subsection{bemerkungen}
\label{section-bemerkungen}
\begin{figure}[H]
    \centering
    \caption{Testüberschrift}
    \begin{adjustbox}{width=\linewidth, center}
        \includegraphics{bemerkungen}
    \end{adjustbox}
\end{figure}

\subsection{zusammenfassung}
\label{section-zusammenfassung}
\begin{figure}[H]
    \centering
    \caption{Testüberschrift}
    \begin{adjustbox}{width=\linewidth, center}
        \includegraphics{zusammenfassung}
    \end{adjustbox}
\end{figure}

\subsection{bestaetigung}
\label{section-bestaetigung}
\begin{figure}[H]
    \centering
    \caption{Testüberschrift}
    \begin{adjustbox}{width=\linewidth, center}
        \includegraphics{bestaetigung}
    \end{adjustbox}
\end{figure}

\subsection{update-bewerbung}
\label{section-update-bewerbung}
\begin{figure}[H]
    \centering
    \caption{Testüberschrift}
    \begin{adjustbox}{width=\linewidth, center}
        \includegraphics{update-bewerbung}
    \end{adjustbox}
\end{figure}

\end{landscape}


% Bibliographie
\pagebreak
\printbibliography

\pagebreak

\newpage
\section{Anhang}

\subsection{Musteranmeldeformular}
\label{section-Musteranmeldeformular}
\begin{figure}[H]
    \centering
    \caption{Musteranmeldeformular von der fiktiven Person Max Müller}
    \begin{adjustbox}{height=0.85\textheight, center}
        \includegraphics{bewerbungsformular1}
    \end{adjustbox}
    \label{fig:anmeldeformular}
\end{figure}

\subsection{fragebogen}
\label{section-fragebogen}
\textbf{Vor der Durchführung}
\begin{itemize}
    \item In welchem Umfang besitzen Sie Vorerfahrungen mit\textit{Schüler Online 1.0}?
    \item In welchem Umfang besitzen Sie Vorerfahrungen mit der neuen Software?
    \item Welche Probleme können bei herkömmlichen, nicht-digitalen Anmeldungen von Schülern auftreten?
    \item Wer nutzt das System hauptsächlich an Ihrer Schule?
    \item Beschreiben Sie die Ausgangssituation die vorliegt, bevor Sie die Aufgabe \glqq Anmeldung eines Schülers\grqq{} durchführen.
    \item Welche fachlichen und technischen Qualifikationen sind zur Bewältigung der Aufgabe erforderlich (Aufgabenbewältigung / Softwarenutzung)? Welche Vorkenntnisse fehl
\end{itemize}
\textbf{Unmittelbar vor der Durchführung}
\begin{itemize}
    \item Welche Arbeitsschritte sind durchzuführen?
    \item Welche Hilfsmittel sind erforderlich (für die Aufgabenbewältigung / zur Softwarenutzung)? Welche davon fehlen ggf., welche sind zusätzlich gewünscht?
    \item Welche Ergebnisse / Teilergebnisse entstehen und wie werden diese ggf. verwertet / weitergeführt?
    \item Welche wichtigen Sonderfälle müssen berücksichtigt werden? (bzw. fallen dem Benutzer spontan ein; z. B. zur Arbeitsteilung / Zusammenarbeit)
\end{itemize}
\textbf{Nach der Durchführung}
\begin{itemize}
    \item Konnten Sie die Aufgabe aus Ihrer Sicht erfolgreich und vollständig abschließen? Falls nein - was hat Sie daran gehindert?
    \item Wie effektiv unterstützt die Webanwendung Sie bei der Aufgabe?  Gab es positive oder negative Erfahrungen?
    \item Haben Sie sämtliche Inhalte der Aufgabe verstanden? Gab es Stellen, an denen Sie sich mehr Unterstützung gewünscht hätten?
    \item Gab es Schwierigkeiten oder Verwirrungen bei der Aufgabe? Wenn ja, welche?
    \item Wie verständlich waren die Rückmeldungen der Anwendung?
    \item Welche Fähigkeiten setzt die Anwendung Ihrer Einschätzung nach voraus?
    \item Wie sehr entspricht die Umsetzung in der Software der Realität?
    \item Was handhaben Sie in Ihrem Arbeitsalltag bei nicht-digitalen Anmeldungen gewöhnlicherweise anders als in der Anwendung?
    \item Was würde Sie noch daran hindern die Software in Ihrem Arbeitsalltag einzusetzen?
    \item Wie sehr erleichtert Ihnen die Anwendung ihre Arbeit?
    \item Gibt es Funktionen, die Sie in ähnlichen bzw. anderen Anwendungen genutzt haben, die Sie hier vermissen?
    \item Welche Software sollte man aus Ihrer Sicht in Schüler Online integrieren bzw. eine Schnittstelle schaffen?
    \item Welche Dokumente würden Sie gerne im Prozess ausdrucken können?
    \item Welche Dokumente würden Sie gerne einscannen wollen und beim Datensatz hinterlegen?
\end{itemize}


\subsection{InterviewGymnasium}
\label{section-InterviewGymnasium}
\input{InterviewGymnasium}

\subsection{InterviewRealschule}
\label{section-InterviewRealschule}
\input{InterviewRealschule}

\subsection{InterviewFoerderschule}
\label{section-InterviewFoerderschule}
\input{InterviewFoerderschule}

\subsection{InterviewGrundschule}
\label{section-InterviewGrundschule}
\input{InterviewGrundschule}

\subsection{InterviewBerufskolleg}
\label{section-InterviewBerufskolleg}
\input{InterviewBerufskolleg}

\begin{landscape}

    \begin{longtable}{p{15cm}cc}
        \caption{Your Table} \label{tab:mytable} \\
        \toprule
        Erfordernis: Der Benutzer muss... & Zugehörige Ergebnisse \\
        \midrule
            ... inkorrekte Daten identifizieren und korrigieren können. & E1, E2 \\
            ... die an ihn eingereichten Formulare korrekt übertragen können. & E5, E6 \\
            ... unzulässige Bewerbungen identifizieren können. & E9, E10 \\
            ... die Daten datenschutzkonform in die Anwendung eintragen können und über mögliche Verstöße informiert werden. & E11 \\
            ... erkennen können, wie er zum korrekten Prozess gelangt. & E12 \\
            ... bezüglich Aufnahmeentscheidungen mit den Entscheidungsträgern zusammen arbeiten können. & E7, E8 \\
            ... die Software auch bei fehlenden Daten bedienen können.  & \\
            ... Aufnahmekriterien berücksichtigen können. & \\
            ... Termine für Aufnahmeberatungsgespräche hinterlegen können. & b \\
            ... Schülerakten erzeugen können. & b \\
            ... Daten aus anderen Programmen übernehmen können. & b \\
            ... Adressrecherchen durchführen können. & b \\
            ... eine Kurzanleitung für den Einstieg abrufen können. & b \\
            ... die Software auch bei fehlenden Daten bedienen können. & b \\
            ... innerjährige Wechsel und Stufenwiederholungen erfassen können. & b \\
            ... langfristige Beurlaubungen vermerken können. & b \\
            ... auch komplizierte Bewerbungen und Sonderfälle bearbeiten können. & b \\
            ... seinen bisherigen Jargon verwenden können. & b \\
            ... die Aufgaben und Prozesse intuitiv bedienen können. & b \\
            ... eine Adressvalidierung vornehmen können. & b \\
            ... Erreichbarkeiten von Notfallkontakten erfassen können. & b \\
            ... unterschiedliche Arten von Notfallkontakten erfassen können. & b \\
            ... erkennen können, ob ein Schüler volljährig ist. & b \\
        \endfirsthead
        \toprule
        Erfordernis & Beispiel & Zugehörige Ergebnisse \\
        \midrule
        \endhead
        \bottomrule
        \endfoot
            ... Nachweise über das Sorgerecht hinterlegen können. & b \\
            ... Daten auch bei Dialogabbrüchen wiederherstellen können. & b \\
            ... ähnliche Datenabfragen aus vorherigen Formularen übernehmen können. & b \\
            ... bei Bedarf Handbücher heranziehen können. & b \\
            ... bei Bedarf Fachterminologie nachschlagen oder verstehen können. & b \\
            ... jederzeit darüber Bescheid wissen, welche Daten den Eltern und Schülern angezeigt werden. & b \\
            ... den Systemstatus jederzeit einsehen können. & b \\
            ... Bewerbungslisten in geeigneter Form exportieren können. & b \\
            ... Daten in Schulverwaltungsprogramme wie Schild exportieren können. & b \\
            ... Daten nach Excel exportieren können. & b \\
            ... Berichte anfertigen können. & b \\
            ... Anmeldezeiträume berücksichtigen. & b \\
            ... mit anderen Behörnden zusammenarbeiten können. & b \\
            ... über Änderungen an Bewerbungen regelmäßig informiert werden. & b \\
            ... zusätzliche Informationen zu einer Bewerbung hinterlegen können. & b \\
            ... Dokumente ausdrücken können. & b \\
            ... Dokumente digitalisieren und beim Datensatz hinterlegen können. & b \\
    \label{tab:anmeldeformular}
\end{longtable}

\subsection{bildungsgang}
\label{section-bildungsgang}
\begin{figure}[H]
    \centering
    \caption{Testüberschrift}
    \begin{adjustbox}{width=\linewidth, center}
        \includegraphics{bildungsgang}
    \end{adjustbox}
\end{figure}

\subsection{sorgeberechtigte-liste}
\label{section-sorgeberechtigte-liste}
\begin{figure}[H]
    \centering
    \caption{Testüberschrift}
    \begin{adjustbox}{width=\linewidth, center}
        \includegraphics{sorgeberechtigte-liste}
    \end{adjustbox}
\end{figure}

\subsection{sorgeberechtigter-person}
\label{section-sorgeberechtigter-person}
\begin{figure}[H]
    \centering
    \caption{Testüberschrift}
    \begin{adjustbox}{width=0.5\linewidth, center}
        \includegraphics{sorgeberechtigter-person}
    \end{adjustbox}
\end{figure}

\subsection{sorgeberechtigter-anschrift}
\label{section-sorgeberechtigter-anschrift}
\begin{figure}[H]
    \centering
    \caption{Testüberschrift}
    \begin{adjustbox}{width=0.85\linewidth, center}
        \includegraphics{sorgeberechtigter-anschrift}
    \end{adjustbox}
\end{figure}

\subsection{sorgeberechtigter-kontakt}
\label{section-sorgeberechtigter-kontakt}
\begin{figure}[H]
    \centering
    \caption{Testüberschrift}
    \begin{adjustbox}{width=0.6\linewidth, center}
        \includegraphics{sorgeberechtigter-kontakt}
    \end{adjustbox}
\end{figure}

\subsection{notfallkontakt-daten}
\label{section-notfallkontakt-daten}
\begin{figure}[H]
    \centering
    \caption{Testüberschrift}
    \begin{adjustbox}{width=0.6\linewidth, center}
        \includegraphics{notfallkontakt-daten}
    \end{adjustbox}
\end{figure}

\subsection{notfallkontakt-liste}
\label{section-notfallkontakt-liste}
\begin{figure}[H]
    \centering
    \caption{Testüberschrift}
    \begin{adjustbox}{width=\linewidth, center}
        \includegraphics{notfallkontakt-liste}
    \end{adjustbox}
\end{figure}

\subsection{migrationshintergrund-liegtvor}
\label{section-migrationshintergrund-liegtvor}
\begin{figure}[H]
    \centering
    \caption{Testüberschrift}
    \begin{adjustbox}{width=\linewidth, center}
        \includegraphics{migrationshintergrund-liegtvor}
    \end{adjustbox}
\end{figure}

\subsection{migrationshintergrund-liegtnichtvor}
\label{section-migrationshintergrund-liegtnichtvor}
\begin{figure}[H]
    \centering
    \caption{Testüberschrift}
    \begin{adjustbox}{width=\linewidth, center}
        \includegraphics{migrationshintergrund-liegtnichtvor}
    \end{adjustbox}
\end{figure}

\subsection{qualifikation}
\label{section-qualifikation}
\begin{figure}[H]
    \centering
    \caption{Testüberschrift}
    \begin{adjustbox}{width=\linewidth, center}
        \includegraphics{qualifikation}
    \end{adjustbox}
\end{figure}

\subsection{letztetaetigkeit}
\label{section-letztetaetigkeit}
\begin{figure}[H]
    \centering
    \caption{Testüberschrift}
    \begin{adjustbox}{width=\linewidth, center}
        \includegraphics{letztetaetigkeit}
    \end{adjustbox}
\end{figure}

\subsection{aufnahmeberatung}
\label{section-aufnahmeberatung}
\begin{figure}[H]
    \centering
    \caption{Testüberschrift}
    \begin{adjustbox}{width=\linewidth, center}
        \includegraphics{aufnahmeberatung}
    \end{adjustbox}
\end{figure}

\subsection{bemerkungen}
\label{section-bemerkungen}
\begin{figure}[H]
    \centering
    \caption{Testüberschrift}
    \begin{adjustbox}{width=\linewidth, center}
        \includegraphics{bemerkungen}
    \end{adjustbox}
\end{figure}

\subsection{zusammenfassung}
\label{section-zusammenfassung}
\begin{figure}[H]
    \centering
    \caption{Testüberschrift}
    \begin{adjustbox}{width=\linewidth, center}
        \includegraphics{zusammenfassung}
    \end{adjustbox}
\end{figure}

\subsection{bestaetigung}
\label{section-bestaetigung}
\begin{figure}[H]
    \centering
    \caption{Testüberschrift}
    \begin{adjustbox}{width=\linewidth, center}
        \includegraphics{bestaetigung}
    \end{adjustbox}
\end{figure}

\subsection{update-bewerbung}
\label{section-update-bewerbung}
\begin{figure}[H]
    \centering
    \caption{Testüberschrift}
    \begin{adjustbox}{width=\linewidth, center}
        \includegraphics{update-bewerbung}
    \end{adjustbox}
\end{figure}

\end{landscape}


\end{document}

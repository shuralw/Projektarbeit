%scrartcl: Für kürzere Ausarbeitungen. Beginnt mit section, es gibt keine chapter
%scrreprt: Für längere Ausarbeitungen (Bacherlo oder Master Thesis). Beginnt chapter
\documentclass[pdftex,a4paper,abstracton,11pt,parskip=half,bibtotocnumbered]{scrartcl} 

% Je nach LaTeX Compiler werden etwas andere Bibliotheken verwendet.
% Das Paket iftex erlaubt es, den Compiler zu überprüfen
\usepackage{iftex}

\usepackage[ngerman]{babel} % Einstellungen für den deutschen Sprachraum, neue deutsche Rechtschreibung
\ifPDFTeX
  \usepackage[utf8]{inputenc} % Umlaute erkennen. Als Option in "[] " die vom Editor verwendete Zeichencodierung auswählen
  \usepackage[T1]{fontenc}
\fi

% Für kompakter Aufzählungen
\usepackage{paralist}

\usepackage[style=ieee,backend=bibtex]{biblatex}
\addbibresource{literatur/literatur.bib}

% Für Abbildungen
\usepackage{graphicx}
\graphicspath{{./abbildungen/}}

% Ändert die Überschrift des Abstracts.
% Falls kein Abstract benötigt wird, kann die Option "abstracton" ganz oben in \documentclass entfallen.
\renewcommand{\abstractname}{Zusammenfassung}



\title{Neugestaltung von Schüler Online: Eine Beobachtungs- und Interviewstudie zur Identifikation von Problemstellen und Nutzerbedürfnissen, um die Effektivität sowie die Zufriedenstellung des Schulpersonals beim Erfüllen von Kernaufgaben der Webanwendung zu optimieren}
\author{Lukas Wessel}
\date{\today}



\begin{document}

\makeatletter
\begin{titlepage}
	\centering
	{\scshape\LARGE Fachhochschule Südwestfalen \par}
	\vspace{1cm}
%	{\scshape\Large Merkblatt\par}
	\vspace{1.5cm}
	{\huge\bfseries \@title\par}
	\vspace{3cm}
	{\Large \@author\par}
	\vspace{1cm}
	{\Large \@date\par}
	\vfill

	\raggedright
%	{\large Eingereicht bei:\par}
%	{\large Betreuer 1}
\end{titlepage}
\makeatother

\thispagestyle{empty}
\begin{abstract}
%Ein Abstrakt, also eine Kurzzusammenfassung der Arbeit ist bei einer schriftlichen Ausarbeitung nicht unbedingt notwendig.
%Bei umfangreicheren Arbeiten, also z.~B. einer Bachelor- oder Master-Thesis, sollte die Ausarbeitung in jedem Fall mit einem \textit{Abstract} beginnen.
Schriftliche Ausarbeitungen sind  wissenschaftliche Texte, die in ihrem formalen Aufbau bestimmten Richtlinien entsprechen müssen.
Dies gilt im Besonderen für Abschlussarbeiten (Bachelor- oder Masterarbeiten), aber prinzipiell auch für kürzere Aufsätze, Hausarbeiten und Projektberichte.
In diesem Leitfaden soll es darum gehen, wie Sie Ihre Ausarbeitung strukturell aufbauen sollten und welche Qualitätskriterien für die äußere und sprachliche Form gelten.
Bei einer typischen Projekt-, Bachelor- oder Masterarbeit macht die schriftliche Ausarbeitung nur einen Teil der Arbeitslast aus, ist aber gleichzeitig das wichtigste Kriterium für die Bewertung.
Daher ist es ratsam, sich möglichst frühzeitig mit den inhaltlichen und formalen Anforderungen wissenschaftlicher Texte vertraut zu machen und diese bei der Anfertigung eigener Ausarbeitungen zu berücksichtigen.
\end{abstract}

\vfill
\tableofcontents
\pagebreak

\setcounter{page}{1}
\input{kapitel/leitfaden}

%\appendix
%
\newpage
\section{Anhang}

\subsection{Musteranmeldeformular}
\label{section-Musteranmeldeformular}
\begin{figure}[H]
    \centering
    \caption{Musteranmeldeformular von der fiktiven Person Max Müller}
    \begin{adjustbox}{height=0.85\textheight, center}
        \includegraphics{bewerbungsformular1}
    \end{adjustbox}
    \label{fig:anmeldeformular}
\end{figure}

\subsection{fragebogen}
\label{section-fragebogen}
\textbf{Vor der Durchführung}
\begin{itemize}
    \item In welchem Umfang besitzen Sie Vorerfahrungen mit\textit{Schüler Online 1.0}?
    \item In welchem Umfang besitzen Sie Vorerfahrungen mit der neuen Software?
    \item Welche Probleme können bei herkömmlichen, nicht-digitalen Anmeldungen von Schülern auftreten?
    \item Wer nutzt das System hauptsächlich an Ihrer Schule?
    \item Beschreiben Sie die Ausgangssituation die vorliegt, bevor Sie die Aufgabe \glqq Anmeldung eines Schülers\grqq{} durchführen.
    \item Welche fachlichen und technischen Qualifikationen sind zur Bewältigung der Aufgabe erforderlich (Aufgabenbewältigung / Softwarenutzung)? Welche Vorkenntnisse fehl
\end{itemize}
\textbf{Unmittelbar vor der Durchführung}
\begin{itemize}
    \item Welche Arbeitsschritte sind durchzuführen?
    \item Welche Hilfsmittel sind erforderlich (für die Aufgabenbewältigung / zur Softwarenutzung)? Welche davon fehlen ggf., welche sind zusätzlich gewünscht?
    \item Welche Ergebnisse / Teilergebnisse entstehen und wie werden diese ggf. verwertet / weitergeführt?
    \item Welche wichtigen Sonderfälle müssen berücksichtigt werden? (bzw. fallen dem Benutzer spontan ein; z. B. zur Arbeitsteilung / Zusammenarbeit)
\end{itemize}
\textbf{Nach der Durchführung}
\begin{itemize}
    \item Konnten Sie die Aufgabe aus Ihrer Sicht erfolgreich und vollständig abschließen? Falls nein - was hat Sie daran gehindert?
    \item Wie effektiv unterstützt die Webanwendung Sie bei der Aufgabe?  Gab es positive oder negative Erfahrungen?
    \item Haben Sie sämtliche Inhalte der Aufgabe verstanden? Gab es Stellen, an denen Sie sich mehr Unterstützung gewünscht hätten?
    \item Gab es Schwierigkeiten oder Verwirrungen bei der Aufgabe? Wenn ja, welche?
    \item Wie verständlich waren die Rückmeldungen der Anwendung?
    \item Welche Fähigkeiten setzt die Anwendung Ihrer Einschätzung nach voraus?
    \item Wie sehr entspricht die Umsetzung in der Software der Realität?
    \item Was handhaben Sie in Ihrem Arbeitsalltag bei nicht-digitalen Anmeldungen gewöhnlicherweise anders als in der Anwendung?
    \item Was würde Sie noch daran hindern die Software in Ihrem Arbeitsalltag einzusetzen?
    \item Wie sehr erleichtert Ihnen die Anwendung ihre Arbeit?
    \item Gibt es Funktionen, die Sie in ähnlichen bzw. anderen Anwendungen genutzt haben, die Sie hier vermissen?
    \item Welche Software sollte man aus Ihrer Sicht in Schüler Online integrieren bzw. eine Schnittstelle schaffen?
    \item Welche Dokumente würden Sie gerne im Prozess ausdrucken können?
    \item Welche Dokumente würden Sie gerne einscannen wollen und beim Datensatz hinterlegen?
\end{itemize}


\subsection{InterviewGymnasium}
\label{section-InterviewGymnasium}
\input{InterviewGymnasium}

\subsection{InterviewRealschule}
\label{section-InterviewRealschule}
\input{InterviewRealschule}

\subsection{InterviewFoerderschule}
\label{section-InterviewFoerderschule}
\input{InterviewFoerderschule}

\subsection{InterviewGrundschule}
\label{section-InterviewGrundschule}
\input{InterviewGrundschule}

\subsection{InterviewBerufskolleg}
\label{section-InterviewBerufskolleg}
\input{InterviewBerufskolleg}

\begin{landscape}

    \begin{longtable}{p{15cm}cc}
        \caption{Your Table} \label{tab:mytable} \\
        \toprule
        Erfordernis: Der Benutzer muss... & Zugehörige Ergebnisse \\
        \midrule
            ... inkorrekte Daten identifizieren und korrigieren können. & E1, E2 \\
            ... die an ihn eingereichten Formulare korrekt übertragen können. & E5, E6 \\
            ... unzulässige Bewerbungen identifizieren können. & E9, E10 \\
            ... die Daten datenschutzkonform in die Anwendung eintragen können und über mögliche Verstöße informiert werden. & E11 \\
            ... erkennen können, wie er zum korrekten Prozess gelangt. & E12 \\
            ... bezüglich Aufnahmeentscheidungen mit den Entscheidungsträgern zusammen arbeiten können. & E7, E8 \\
            ... die Software auch bei fehlenden Daten bedienen können.  & \\
            ... Aufnahmekriterien berücksichtigen können. & \\
            ... Termine für Aufnahmeberatungsgespräche hinterlegen können. & b \\
            ... Schülerakten erzeugen können. & b \\
            ... Daten aus anderen Programmen übernehmen können. & b \\
            ... Adressrecherchen durchführen können. & b \\
            ... eine Kurzanleitung für den Einstieg abrufen können. & b \\
            ... die Software auch bei fehlenden Daten bedienen können. & b \\
            ... innerjährige Wechsel und Stufenwiederholungen erfassen können. & b \\
            ... langfristige Beurlaubungen vermerken können. & b \\
            ... auch komplizierte Bewerbungen und Sonderfälle bearbeiten können. & b \\
            ... seinen bisherigen Jargon verwenden können. & b \\
            ... die Aufgaben und Prozesse intuitiv bedienen können. & b \\
            ... eine Adressvalidierung vornehmen können. & b \\
            ... Erreichbarkeiten von Notfallkontakten erfassen können. & b \\
            ... unterschiedliche Arten von Notfallkontakten erfassen können. & b \\
            ... erkennen können, ob ein Schüler volljährig ist. & b \\
        \endfirsthead
        \toprule
        Erfordernis & Beispiel & Zugehörige Ergebnisse \\
        \midrule
        \endhead
        \bottomrule
        \endfoot
            ... Nachweise über das Sorgerecht hinterlegen können. & b \\
            ... Daten auch bei Dialogabbrüchen wiederherstellen können. & b \\
            ... ähnliche Datenabfragen aus vorherigen Formularen übernehmen können. & b \\
            ... bei Bedarf Handbücher heranziehen können. & b \\
            ... bei Bedarf Fachterminologie nachschlagen oder verstehen können. & b \\
            ... jederzeit darüber Bescheid wissen, welche Daten den Eltern und Schülern angezeigt werden. & b \\
            ... den Systemstatus jederzeit einsehen können. & b \\
            ... Bewerbungslisten in geeigneter Form exportieren können. & b \\
            ... Daten in Schulverwaltungsprogramme wie Schild exportieren können. & b \\
            ... Daten nach Excel exportieren können. & b \\
            ... Berichte anfertigen können. & b \\
            ... Anmeldezeiträume berücksichtigen. & b \\
            ... mit anderen Behörnden zusammenarbeiten können. & b \\
            ... über Änderungen an Bewerbungen regelmäßig informiert werden. & b \\
            ... zusätzliche Informationen zu einer Bewerbung hinterlegen können. & b \\
            ... Dokumente ausdrücken können. & b \\
            ... Dokumente digitalisieren und beim Datensatz hinterlegen können. & b \\
    \label{tab:anmeldeformular}
\end{longtable}

\subsection{bildungsgang}
\label{section-bildungsgang}
\begin{figure}[H]
    \centering
    \caption{Testüberschrift}
    \begin{adjustbox}{width=\linewidth, center}
        \includegraphics{bildungsgang}
    \end{adjustbox}
\end{figure}

\subsection{sorgeberechtigte-liste}
\label{section-sorgeberechtigte-liste}
\begin{figure}[H]
    \centering
    \caption{Testüberschrift}
    \begin{adjustbox}{width=\linewidth, center}
        \includegraphics{sorgeberechtigte-liste}
    \end{adjustbox}
\end{figure}

\subsection{sorgeberechtigter-person}
\label{section-sorgeberechtigter-person}
\begin{figure}[H]
    \centering
    \caption{Testüberschrift}
    \begin{adjustbox}{width=0.5\linewidth, center}
        \includegraphics{sorgeberechtigter-person}
    \end{adjustbox}
\end{figure}

\subsection{sorgeberechtigter-anschrift}
\label{section-sorgeberechtigter-anschrift}
\begin{figure}[H]
    \centering
    \caption{Testüberschrift}
    \begin{adjustbox}{width=0.85\linewidth, center}
        \includegraphics{sorgeberechtigter-anschrift}
    \end{adjustbox}
\end{figure}

\subsection{sorgeberechtigter-kontakt}
\label{section-sorgeberechtigter-kontakt}
\begin{figure}[H]
    \centering
    \caption{Testüberschrift}
    \begin{adjustbox}{width=0.6\linewidth, center}
        \includegraphics{sorgeberechtigter-kontakt}
    \end{adjustbox}
\end{figure}

\subsection{notfallkontakt-daten}
\label{section-notfallkontakt-daten}
\begin{figure}[H]
    \centering
    \caption{Testüberschrift}
    \begin{adjustbox}{width=0.6\linewidth, center}
        \includegraphics{notfallkontakt-daten}
    \end{adjustbox}
\end{figure}

\subsection{notfallkontakt-liste}
\label{section-notfallkontakt-liste}
\begin{figure}[H]
    \centering
    \caption{Testüberschrift}
    \begin{adjustbox}{width=\linewidth, center}
        \includegraphics{notfallkontakt-liste}
    \end{adjustbox}
\end{figure}

\subsection{migrationshintergrund-liegtvor}
\label{section-migrationshintergrund-liegtvor}
\begin{figure}[H]
    \centering
    \caption{Testüberschrift}
    \begin{adjustbox}{width=\linewidth, center}
        \includegraphics{migrationshintergrund-liegtvor}
    \end{adjustbox}
\end{figure}

\subsection{migrationshintergrund-liegtnichtvor}
\label{section-migrationshintergrund-liegtnichtvor}
\begin{figure}[H]
    \centering
    \caption{Testüberschrift}
    \begin{adjustbox}{width=\linewidth, center}
        \includegraphics{migrationshintergrund-liegtnichtvor}
    \end{adjustbox}
\end{figure}

\subsection{qualifikation}
\label{section-qualifikation}
\begin{figure}[H]
    \centering
    \caption{Testüberschrift}
    \begin{adjustbox}{width=\linewidth, center}
        \includegraphics{qualifikation}
    \end{adjustbox}
\end{figure}

\subsection{letztetaetigkeit}
\label{section-letztetaetigkeit}
\begin{figure}[H]
    \centering
    \caption{Testüberschrift}
    \begin{adjustbox}{width=\linewidth, center}
        \includegraphics{letztetaetigkeit}
    \end{adjustbox}
\end{figure}

\subsection{aufnahmeberatung}
\label{section-aufnahmeberatung}
\begin{figure}[H]
    \centering
    \caption{Testüberschrift}
    \begin{adjustbox}{width=\linewidth, center}
        \includegraphics{aufnahmeberatung}
    \end{adjustbox}
\end{figure}

\subsection{bemerkungen}
\label{section-bemerkungen}
\begin{figure}[H]
    \centering
    \caption{Testüberschrift}
    \begin{adjustbox}{width=\linewidth, center}
        \includegraphics{bemerkungen}
    \end{adjustbox}
\end{figure}

\subsection{zusammenfassung}
\label{section-zusammenfassung}
\begin{figure}[H]
    \centering
    \caption{Testüberschrift}
    \begin{adjustbox}{width=\linewidth, center}
        \includegraphics{zusammenfassung}
    \end{adjustbox}
\end{figure}

\subsection{bestaetigung}
\label{section-bestaetigung}
\begin{figure}[H]
    \centering
    \caption{Testüberschrift}
    \begin{adjustbox}{width=\linewidth, center}
        \includegraphics{bestaetigung}
    \end{adjustbox}
\end{figure}

\subsection{update-bewerbung}
\label{section-update-bewerbung}
\begin{figure}[H]
    \centering
    \caption{Testüberschrift}
    \begin{adjustbox}{width=\linewidth, center}
        \includegraphics{update-bewerbung}
    \end{adjustbox}
\end{figure}

\end{landscape}


% Bibliographie
\printbibliography

\end{document}

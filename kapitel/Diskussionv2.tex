\section{Diskussion}

\subsection{Zusammenfassung}
Diese Studie zeigt, dass die untersuchten Bereiche der Anwendung \textit{Schüler Online 2.0} laut der befragten Sekretärinnen zwar insgesamt intuitiv (s. E83) und im Vergleich zur alten Anwendung \textit{Schüler Online 1.0}, \glqq besser\grqq{} sowie \glqq übersichtlicher\grqq{} seien (s. E81), allerdings ließen sich auch diverse Problemstellen in den Beobachtungen und Interviews feststellen. 

%Die Entscheidung gegen eine Durchführung in einer Laborumgebung basierte auf zwei primären Überlegungen. Einerseits konnte eine solche Umgebung nicht die notwendige Realitätsnähe liefern, die für ein umfassendes Verständnis der Nutzung des Produktes im Alltag der Teilnehmer erforderlich war. Andererseits waren finanzielle Gründe ausschlaggebend für die Auswahl des Feldtests.
%

\subsection{Interpretation der Ergebnisse}

\subsubsection{Problemstellen bezüglich Verständlichkeit}

Die Studie zeigte anhand der Ergebnisse (E61, E89), dass beispielsweise durch eine Verwendung der Begriffe \textit{Angestrebte Schulstufe} im Kontext des Tabs \textit{Letzte Tätigkeit} \textbf{das Programm Widersprüche beinhaltet}. Norman spricht diesbezüglich von "<konzeptuellen Modellen"> - also wie Benutzer Erwartungen darüber entwickeln, wie etwas funktionieren sollte, basierend auf ihrer bisherigen Erfahrung und den Hinweisen, die das Design bietet. Wenn ein Design Elemente enthält, die widersprüchlich oder verwirrend sind, kann es zu einer Diskrepanz zwischen dem konzeptuellen Modell des Benutzers und dem tatsächlichen Verhalten des Systems kommen. 

Die Studie zeigte anhand der Ergebnisse (E35, E45, E46), dass stellenweise \textbf{keine intuitive Bedienung} in der Anwendung vorliegt. Demgegenüber steht das Ergebnis E83, in dem die Anwendung abschließend als "<relativ intuitiv"> eingeschätzt wurde.

Die Studie zeigte anhand der Ergebnisse (E12, E85, E86), dass eine \textbf{unverständliche Navigation} wie etwa bei der Navigation zur Aufgabe 1 vorliegt. Dieses Phänomen wird auch von Steve Krug beleuchtet, der Navigationsprobleme anhand eines alltäglichen Beispiels beschreibt: den Kauf einer Kettensäge in einem Geschäft. Genau wie ein Kunde im Laden sich zunächst in den falschen Abteilungen verirren kann, so kann sich auch die Sekretärin auf den falschen Seiten der Software verlieren. Krug betont, dass solche Entscheidungen auf Variablen basieren, wie der Vertrautheit mit dem Geschäft und der Dringlichkeit der Aufgabe\footnote{Im Original: "<how familiar are you with the store"> und "<how much of a hurry you're in">.}.\cite{krug} In ähnlicher Weise kann die Navigation innerhalb der Software auch von der Vertrautheit des Benutzers mit der Anwendung und dem Zeitrahmen, in dem die Aufgaben erledigt werden müssen, beeinflusst werden. Insbesondere könnte die Notwendigkeit, während der Hauptanmeldezeiten zahlreiche Anmeldungen zu bearbeiten, den Druck auf die Sekretärin erhöhen und somit als signifikanter Katalysator für Frustration wirken.

Die Studie zeigte anhand der Ergebnisse (E77, E92), dass \textbf{Hilfestellungen wie Handbücher oder Erklärtexte fehlen}. Die ISO 9241-11:2018 empfiehlt, dass Benutzer grundsätzlich eine geeignete Dokumentation und Unterstützung erhalten sollten.\cite{iso9241-11} Nielsen betont die Bedeutung der Bereitstellung klarer Hilfestellungen und Dokumentationen, welche präzise formuliert sein sollten.\cite{Nielsen10}

Die Ergebnisse E93, E94, E95, E96 zeigten im Gegensatz zu den vorher genannten Problematiken, dass die Erfüllung der Aufgaben in der Anwendung insgesamt verständlich sei. Die Problematiken scheinen also benutzerabhängig zu sein.

\subsubsection{Problemstellen bezüglich Fehlerhandhabung und Nutzerführung}
Die Studie zeigte anhand der Ergebnisse (E71, E73, E91), dass \textbf{Fehler auftreten}. Einerseits stelle die Falschmeldung der \textit{vorzeitige Einschulung} als auch die Fehlermeldung beim Klick auf \textit{Speichern} einen Fehler dar. Um ein fehlerfreies System zu erreichen, sollten laut Nielsen "<error-prone conditions [eliminiert]"> oder diese rechtzeitig überprüft werden und der Nutzer um Fortsetzungsbestätigung gebeten werden. \cite{Nielsen10} 

Die Studie zeigte anhand der Ergebnisse (E71, E81, E91), dass \textbf{Führungen bei Fehlern fehlen}. Nielsen fordert diesbezüglich, dass Hilfestellungen beispielsweise zum Beheben von Fehlern erhalten sollte ausgegeben werden sollten. Diese sollten auch "<concrete Steps"> enthalten. \cite{Nielsen10}

Die Studie zeigte anhand des Ergebnisses (E44), dass \textbf{Dialogabbrüche} auftreten. Dies hatte eine Sekretärin "<verärgert">. Tullis und Albert sprechen in diesem Zusammenhang von einer "<Expression of Frustration">, welche einen "<Usability Issue"> darstellt.\cite{TullisAlbert}

\subsubsection{Problemstellen von Inhaltlichen Fehler}
Die Studie zeigte anhand der Ergebnisse (E75, E82, E105), dass die \textbf{Daten nicht vollständig erfasst} werden. Laut Nielsen sollte das System der realen Welt entsprechen.\cite{Nielsen10} Es sollte dementsprechend alle notwendigen Daten erfassen.

Die Studie zeigte anhand des Ergebnisses (E6), dass \textbf{nicht vollständig vorliegende Daten} eine Problemstelle darstellen. Dies ist allerdings mitunter kein Problem der Software an sich, sondern eine Prärequisite, die vorhanden sein muss, damit die Anwendung bedient werden kann.

Die Studie zeigte anhand der Ergebnisse (E25, E27, E29, E33, E47, E49, E51, E56, E66, E67, E71, E74, E76, E87), dass die Problemstellen \textbf{Datenfelder und Schaltflächen sind unverständlich} sowie \textbf{Texte oder Bezeichner werden missinterpretiert} zu Verwirrungen führen. Der Nutzer sollte sich die Fachterminologie nicht jederzeit merken müssen und bei Bedarf auf Glossare, Hilfe und Dokumentationen zurückgreifen können. Dies ist vergleichbar mit Nielsen's Ausführungen bezüglich \glqq Selbsterklärung vor Erinnerung\grqq{} sowie \glqq Hilfe und Dokumentation\grqq{} \cite{Nielsen10}

Die Studie zeigte anhand des Ergebnisses (E106) dass \textbf{inkorrekte Daten} in den Formularen vorliegen können. Es sollte sichergestellt werden, dass Daten wie \textit{E-Mail-Adresse} nicht unverifiziert oder unvalidiert in das System gelangen.

\subsubsection{Problemstellen bezüglich der Anordnung und Präsentation von Inhalten}

Die Studie zeigte anhand der Ergebnisse (E48, E76, E84), dass \textbf{unattraktive}, \textbf{unübersichtliche} oder \textbf{überladene Seiten} vorhanden sind. Nielsen fordert diesbezüglich, dass sich die Inhalte von Benutzeroberflächen auf das Essentiellste konzentrieren. Das Design sollte laut ihm ästhetisch und minimalistisch sein. \cite{Nielsen10}

Die Studie zeigte anhand der Ergebnisse (E31, E58), dass an zwei Stellen \textbf{Werte unlogisch sortiert} vorliegen.

Die Studie zeigte anhand der Ergebnisse (E59, E70, E88), dass stellenweise \textbf{keine logische Reihenfolge der Inhalte} vorliegen. Nielsen schreibt hierzu, dass "<Industry conventions"> verwendet werden sollten.\cite{Nielsen10} Dies kann insofern für \textit{Schüler Online} auslegen werden, alsdass die Reihenfolge der Tabs und Felder denen von \textit{SchILD} oder allgemeinen Industriestandards ähneln sollte, da bei Benutzern durch Konsistenzverlust die kognitive Last aufgrund des Neulernens erhöht wird.\cite{Nielsen10}. Auch Cooper et al. betonen die Wichtigkeit von Konsistenzen und Standards in Anwendungen, indem sie darauf hinweisen, dass die Verwendung eines einheitlichen Standards die Benutzerfähigkeit verbessert, Schnittstellen schneller erlernbar macht und die Produktivität erhöht, während gleichzeitig Fehler reduziert werden. Sie warnen jedoch davor, Standards rigide anzuwenden, wo sie nicht angebracht sind und betonen, dass die "<Spirit of the law"> und nicht die "<letter of the law"> der Leitfaden sein sollte \cite{CooperReimannEtAl14}.

Die Studie zeigte anhand des Ergebnisses (E97), dass \textbf{die Software nicht aufgebaut ist wie ähnliche Software}. Menschen bilden laut Norman Erwartungen basierend auf früheren Erfahrungen. Wenn sie eine Tür in einem Zugwaggon sehen, erwarten sie, dass sie sich so verhält wie andere Zugtüren, die sie in der Vergangenheit benutzt haben. Wenn eine Software anders aufgebaut ist als ähnliche Software, die der Benutzer zuvor verwendet hat, kann dies zu Verwirrung und Fehlern führen, ähnlich wie im Beispiel welches Norman nennt, in dem aufgrund der unerwarteten Funktionsweise der Waggontür der Zugreisende seinen Zug verpasst. Diese führt dazu, dass der Benutzer den Zug verpasst.\cite{Norman} In ähnlicher Weise kann eine Software, die sich nicht wie ähnliche Software verhält, dazu führen, dass Benutzer Fehler machen oder wichtige Funktionen übersehen. Sie könnten sogar aufhören, das Produkt zu verwenden, weil es nicht ihren Erwartungen entspricht. Wie E98 allerdings aufzeigt, ist die Software inhaltlich mit Schild vergleichbar.

Die Studie zeigte anhand der Ergebnisse (E50, E57, E69), dass \textbf{nicht sichtbare Felder} vorhanden sind.

Zusammengefasst weisen die Ergebnisse der Analyse die meisten Schwierigkeiten im Bereich der Verständlichkeit und Intuitivität auf. Besonders problematisch sind die unklare Navigation, die den Benutzer daran hindern kann, die Aufgaben überhaupt zu beginnen, sowie Dialogabbrüche, die bei den Nutzern Frustration hervorrufen können.

\subsubsection{Nutzerbedürfnisse}
\label{section-beduerfnisse}
\subsubsection{Erfordernisse}
Die Studie zeigte viele Erfordernisse auf. Eine detaillierte Aufzählung ist Tabelle 1 zu entnehmen. Die wichtigsten werden jedoch im Folgenden benannt.

Der Benutzer muss in der Lage sein, inkorrekte Daten zu identifizieren und zu korrigieren. Des Weiteren muss er Daten nach Excel exportieren können und bei Bedarf Handbücher heranziehen können. Die Fähigkeit, unzulässige Bewerbungen zu identifizieren ist unerlässlich, ebenso wie die Möglichkeit, innerjährige Wechsel und Stufenwiederholungen zu erfassen. Eine datenschutzkonforme Dateneingabe ist erforderlich, wobei der Benutzer über mögliche Verstöße informiert werden muss. Darüber hinaus muss der Benutzer die Software auch bei fehlenden Daten bedienen können und erkennen können, wie er zum korrekten Prozess gelangt.

\subsubsection{Erwartungen}





\subsection{Kritische Analyse}
Der qualitative Charakter der Studie ermöglichte umfangreiche Einblicke in die Problemstellen der Anwendung und der Nutzerbedürfnisse, die die Zufriedenheit der Nutzer schmälern oder sie daran hindern, ihre dienstlichen Aufgaben effektiv zu bewältigen. 

Trotz der aufschlussreichen Ergebnisse der Studie gibt es Kritikpunkte. Obwohl laut Nielsen ein Stichprobenumfang von fünf Teilnehmern ausreicht, um bereits 85\% der Usability-Probleme zu identifizieren\cite{Nielsen5Teilnehmer}, sind die Ergebnisse nicht zwangsläufig repräsentativ für ganz Nordrhein-Westfalen. Eine quantitativ angelegte Untersuchung mit einer umfangreicheren Stichprobe wäre erforderlich, um die gewonnenen Erkenntnisse zu bestätigen oder zu widerlegen.

Auch die abweichende Realitätsnähe, da keine Echtdaten verwendet und nicht im echten Arbeitskontext gearbeitet wurde, könnte die Ergebnisse beeinflusst haben. Diese Entscheidung wurde jedoch bewusst getroffen, um zu große Unterbrechungen und datenschutzrechtliche Herausforderungen zu vermeiden.

Weitere kritische Aspekte sind der mögliche Einfluss des Hawthorne-Effekts, welcher besagt dass Studienteilnehmer ihr Verhalten ändern wenn ihnen bekannt ist, dass sie beobachtet werden\cite{mayo1933human,landsberger1958hawthorne,adair1984hawthorne} sowie der Effekt der sozialen Erwünschtheit, welcher beschreibt dass Leute sich beispielsweise bei Beobachtungen so verhalten wie sie es für gesellschaftlich erwartet und akzeptabel empfinden.\cite{crowne1960scale, fisher1993social,paulhus2002socially} Die offene Beobachtung könnte dazu geführt haben, dass die Studienteilnehmer ihr Verhalten veränderten, weil sie wussten, dass sie beobachtet wurden. Zudem könnten sie versucht haben, ein positives Bild von sich abzugeben, was ebenfalls das Verhalten verfälscht haben könnte.

Der zeitliche Kontext der Studie, der außerhalb der regulären Aufnahmephase lag, könnte auch die Ergebnisse der Aufgaben 2 und 3 beeinflusst haben. Da die Studienteilnehmer keinen akuten intrinsischen Drang hatten, über Aufnahmeentscheidungen zu urteilen, wäre ein anderer zeitlicher Kontext möglicherweise aussagekräftiger gewesen.

Die Reliabilität der Studie wurde durch die Verwendung identischer Fragebögen (mit Ausnahme der Geburtstage und der Bildungsgänge) weitestgehend sichergestellt.

\subsection{Fazit}

\textit{Schüler Online} ist ein Online-Portal, das Schülerinnen und Schülern die Möglichkeit bietet, Bewerbungen an Schulen digital einzureichen. Im Rahmen der betrachteten Untersuchung lag der Schwerpunkt auf der Erfassung und Bearbeitung von Anmeldungen durch Sekretariatsmitarbeiter an fünf verschiedenen Schulen.

Die Effektivität im Kontext dieser Untersuchung wurde als das Ausmaß der Genauigkeit und Vollständigkeit definiert, mit dem die Studienteilnehmer die ihnen gestellten Aufgaben erfüllen konnten. Zufriedenheit wurde beschrieben als das Ausmaß der Übereinstimmung zwischen den Reaktionen des Schulpersonals, die aus der Benutzung von \textit{Schüler Online} entstanden und den Benutzeranforderungen sowie Benutzererwartungen.

Der in der Studie verwendete Fragebogen, der auf die Bestimmung der Effektivität sowie der Erfordernisse und Erwartungen abzielte, umfasste Fragen die vor-, unmittelbar vor- und nach der Durchführung der Aufgaben gestellt wurden. Der vollständige Fragebogen kann Anhang \ref{section-fragebogen} entnommen werden.

Die Studie ergab, dass diverse Problemstellen identifiziert wurden, die weitestgehend auch in der Literatur nachgewiesen werden konnten. Die Teilnehmer konnten ihre Ziele nur teilweise effektiv erreichen. Insbesondere stellten sich Herausforderungen in der Navigation zur ersten Aufgabe für zwei der fünf Sekretärinnen dar, was Erfolgskriterium 1 (Navigation erfolgreich) in diesen Fällen verwirft. Von drei Sekretärinnen kam der Wunsch auf, weitere wichtige Daten erfassen zu können, was Erfolgskriterium 2 (Pflichtangaben erfasst) als nur teilweise erfüllt gelten lässt. Bezüglich Erfolgskriterium 3 (Aufnahmestatus ist Aufgenommen / Abgelehnt / Warteliste) wurde festgestellt, dass keine der Sekretärinnen den Anmeldestatus auf \textit{Aufgenommen}, \textit{Abgelehnt} oder \textit{Warteliste} gesetzt hat, stattdessen verblieb der Anmeldestatus auf \textit{Angemeldet}, was dieses Kriterium vollständig verwirft. Dies könnte möglicherweise auf missverständliche Terminologie im Dropdown des Aufnahmestatus zurückzuführen sein. 
Vier der Sekretärinnen gaben an, dass sie die Aufgaben erfolgreich speichern konnten, während eine Sekretärin angab, dass die Aufnahmeentscheidung noch aussteht. In vier von fünf Fällen wurde somit Erfolgskriterium 4 (Anmeldung erfolgreich gespeichert) erfüllt.

Um die Zufriedenheit zu steigern sollten insbesondere aus den in Kapitel \ref{section-beduerfnisse} identifizierten Erfordernisse Anforderungen an die Anwendung abgeleitet und Prüfkriterien entworfen werden, um in einem weiteren Schritt Verbesserungen zu implementieren.   

Die Ergebnisse der Untersuchung weisen insgesamt auf einen deutlichen Verbesserungsbedarf hin, damit Sekretariatsmitarbeiter die Bewerbungen von Schülern mit Schüler Online effektiv und zufriedenstellend erfassen sowie bearbeiten können.

\subsection{Ausblick}
Es kann eine Folgeuntersuchung durchgeführt werden, die konkrete Maßnahmen anhand der Ergebnisse herleitet. Hierfür bietet sich beispielsweise das Konzept einer Fokusgruppe an. Diese Änderungen sollten anschließend entsprechend eines iterativen Vorgehens auf Wirksamkeit überprüft und reanalysiert werden. 
Auch kann eine Beobachtung erfolgen, wie Schulen im Arbeitsalltag die Software mit Echtdaten in reellen Situationen verwenden. Insbesondere wäre es interessant zu wissen, wie die Software verwendet wird, wenn viele Anmeldungen eingereicht werden und viele Schüler die Schule anrufen.
Damit festgestellt werden kann, wie repräsentativ die Ergebnisse sind, sollte nachfolgend untersucht werden, wie die Ergebnisse bei einem großen Stichprobenumfang ausfallen. Dies könnte mit einer quantitativen Studie erforscht werden.

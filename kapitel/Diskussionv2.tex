\subsection
diese studie soll zeigen, dass 


\textit{Schüler Online} ist insgesamt eine Software, die das kann..


\subsection{Diskussion der Erfordernisse}
Damit das System den Benutzer zufriedenstellen kann, muss es die folgenden Erfordernisse der Benutzer erfüllen.
     Da geäußert wurde, dass ... muss Benutzer das System {ohne spezielle Vorkenntnisse oder Schulungen nutzen können.}  Dies wird auch in derr Literatur von ... gefordert.
ODER Da die Beobachtung gezeigt hat, dass viele Unterbrechungen auftreten, dass ... muss Benutzer das System {ohne spezielle Vorkenntnisse oder Schulungen nutzen können.}  Dies wird auch in derr Literatur von ... gefordert.
Der Benutzer muss in der Lage sein, {inkorrekte oder fehlende Angaben zu identifizieren und zu korrigieren.} 
Der Benutzer muss {handschriftlich ausgefüllte Formulare digitalisieren oder per E-Mail einreichen können.}
Der Benutzer muss die geltenden {Datenschutzgesetze einhalten und sensibel mit persönlichen Informationen umgehen} können.
     => Datenschutzerklärung
Der Benutzer muss in der Lage sein, {Mechanismen zur Erkennung und Vermeidung von Mehrfachanmeldungen zu nutzen.}
Der Benutzer muss in der Lage sein, {innerhalb des Systems mit anderen Beteiligten wie Verwaltung, Lehrern und Eltern zu kommunizieren.}
Der Benutzer muss {Fachbegriffe verstehen oder Zugang zu Erklärungen für Fachbegriffe haben.}
Der Benutzer muss das System {unabhängig von seinem individuellen Erfahrungslevel mit der Anwendung nutzen können.}
Der Benutzer muss in der Lage sein, {Daten in andere Anwendungen übertragen zu können}
Der Benutzer muss {problemlos durch die Anwendung navigieren können.}
Der Benutzer muss die Möglichkeit haben, {Kopier- und Einfügefunktionalitäten zu nutzen, um Daten aus anderen Programmen übertragen zu können.}
Der Benutzer muss in der Lage sein, {besondere Situationen wie Wechsel innerhalb des Schuljahres und Fälle langfristiger Beurlaubung zu behandeln.}
Der Benutzer muss in der Lage sein, unterschiedliche Anforderungen wie Unterricht in der Herkunftssprache, Unterschiede in Schulstufen, spezifische Sorgerechtssituationen und Förderbedarf zu berücksichtigen.
Der Benutzer muss {vor Datenverlust geschützt sein, etwa durch versehentliches Klicken neben ein Popup oder Ausloggen.}
Der Benutzer muss {klare Richtlinien und Funktionen für das Erfassen von sensiblen Informationen wie medizinischen Daten haben.}
Der Benutzer muss {Berichte über die Schülerdaten anfertigen können.}








E1: Der Benutzer muss falsche Daten erkennen und korrigieren können
E2: Der Benutzer muss falsche Daten erkennen und korrigieren können. Der Benutzer muss die Software auch bei fehlenden Daten bedienen können. Unvollständige Daten.
E3: Der Benutzer muss die verwendeten Begrifflichkeiten aus der Praxis verstehen können.
E4: Der Benutzer muss falsche Daten erkennen und korrigieren können
E5: Der Benutzer muss die an ihn eingereichten Formulare korrekt übertragen können.
E6: Unvollständige Daten.
E7: Der Benutzer muss bezüglich Aufnahmeentscheidungen mit den Entscheidungsträgern zusammen arbeiten können.
E8: Der Benutzer muss bezüglich Aufnahmeentscheidungen mit den Entscheidungsträgern zusammen arbeiten können.
E9: Der Benutzer muss unzulässige Bewerbungen erkennen können. (Mehrfachanmeldungen)
E10: Der Benutzer muss Aufnahmekriterien berücksichtigen können.
E11: Der Benutzer muss die Daten datenschutzkonform in die Anwendung eintragen können und über mögliche Verstöße informiert werden.
E12: Der Benutzer muss erkennen können, wie er zum korrekten Prozess gelangt. Navigationsschwierigkeiten.
E13: Der Benutzer muss Termine für Aufnahmeberatungsgespräche hinterlegen können.
E14: Der Benutzer muss Termine für Aufnahmeberatungsgespräche hinterlegen können.
E15: Der Benutzer muss Schülerakten erzeugen können.
E16: Der Benutzer muss Daten aus anderen Programmen übernehmen können.
E17: Der Benutzer muss Adressrecherchen durchführen können.
E18: Der Benutzer muss eine Kurzanleitung für den Einstieg abrufen können.
E19: Der Benutzer muss die Software auch bei fehlenden Daten bedienen können. Unvollständige Daten.
E20: Der Benutzer muss innerjährige Wechsel und Stufenwiederholungen erfassen können.
E21: Der Benutzer muss die Software auch bei fehlenden Daten bedienen können. Unvollständige Daten.
E22: Der Benutzer muss langfristige Beurlaubungen vermerken können.
E23: Der Benutzer muss 
E24: Interpretationsfehler von Texten und Bezeichnern. Anwender kennt Fachterminologie nicht.
E25:
E26:
E27:
E28:
E29:
E30:
E31:
E32:
E33:
E34:
E35:
E36:

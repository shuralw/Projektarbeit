\section{Diskussion}

\textbf{DIESES KAPITEL MUSS UNBEDINGT NOCH ÜBERARBEITET WERDEN}

\subsection{Zusammenfassung}
Diese Studie zeigt, dass die untersuchten Bereiche der Anwendung \textit{Schüler Online 2.0} laut der Sekretärinnen zwar insgesamt intuitiv (s. E83) und im Vergleich zur alten Anwendung \textit{Schüler Online 1.0}, \glqq besser\grqq{} sowie \glqq übersichtlicher \grqq{} als seien (s. E81), allerdings ließen sich auch diverse Problemstellen in den Beobachtungen und Interviews feststellen. 

%Die Entscheidung gegen eine Durchführung in einer Laborumgebung basierte auf zwei primären Überlegungen. Einerseits konnte eine solche Umgebung nicht die notwendige Realitätsnähe liefern, die für ein umfassendes Verständnis der Nutzung des Produktes im Alltag der Teilnehmer erforderlich war. Andererseits waren finanzielle Gründe ausschlaggebend für die Auswahl des Feldtests.
%

\subsection{Interpretation der Ergebnisse}

\textbf{Problemstellen}:

\textbf{Keine intuitive Bedienung}

\textbf{Die Daten liegen nicht vollständig vor}

\textbf{Fehler treten auf} und \textbf{Programmabbrüche} wie das Verlassen von Dialogfeldern: Das System sollte fehlerfrei sein. Dies wird am ehesten erreicht, wenn man laut Nielsen fehleranfällige Bedingungen  vermeidet oder sie rechtzeitig überprüft und den Nutzer um Bestätigung bittet \cite{Nielsen10}

\textbf{Fehlende Führung bei Fehlern}: Nielsen fordert hierzu, dass man Hilfestellungen beispielsweise zum Beheben von Fehlern erhalten sollte.  \cite{Nielsen10}
sollten ebenfalls vermieden werden. 

\textbf{Unverständliche Navigation}

Für die Problemstellen \textbf{Datenfelder sind unverständlich}, \textbf{Schaltflächen sind unverständlich} sowie \textbf{Texte oder Bezeichner werden missinterpretiert} lässt sich interpretieren, dass man sich entweder nicht die Fachterminologie jederzeit merken muss oder bei Bedarf auf Glossare, Hilfe und Dokumentationen zurückgreifen können sollte. Dies beschreibt Nielsen mit den Punkten \glqq Selbsterklärung vor Erinnerung\grqq{} sowie \glqq Hilfe und Dokumentation\grqq{} \cite{Nielsen10}

\textbf{Das Programm beinhaltet Widersprüche}

\textbf{Unattraktiv} und \textbf{Unübersichtlich}: Nielsen fordert, dass sich die Inhalte von Benutzeroberflächen auf das essentiellste konzentrieren. Das Design sollte laut ihm ästhetisch und minimalistisch sein. \cite{Nielsen10}

\textbf{Felder sind nicht sichtbar}

\textbf{Unlogische Sortierung von Werten}

\textbf{Die Software ist nicht aufgebaut wie ähnliche Software} \cite{Nielsen10}

\textbf{Unvollständige Datenabfrage}

\textbf{Logische Reihenfolge der Inhalte}

\textbf{Nutzerbedürfnisse}

Ich finde das hier könnte man als Tabelle in den Anhang auslagern.


\subsection{Kritische Analyse}
Der qualitative Charakter der Studie ermöglichte umfangreiche Einblicke in die Problemstellen der Anwendung und der Nutzerbedürfnisse, die die Zufriedenheit der Nutzer schmälern oder sie daran hindern, ihre Aufgaben effektiv zu bewältigen. 

Dennoch gab es in der Studie auch einige Kritikpunkte. Der Stichprobenumfang von nur fünf bis sechs Studienteilnehmern war zwar für die Erforschung erster Eindrücke ausreichend, lässt jedoch keine repräsentativen Aussagen für ganz Nordrhein-Westfalen zu. Eine quantitative Forschung mit einem größeren Stichprobenumfang wäre notwendig, um die Ergebnisse zu validieren oder zu falsifizieren.

Auch die abweichende Realitätsnähe, da keine Echtdaten verwendet und nicht im echten Arbeitskontext gearbeitet wurde, könnte die Ergebnisse beeinflusst haben. Diese Entscheidung wurde jedoch bewusst getroffen, um zu große Unterbrechungen und datenschutzrechtliche Herausforderungen zu vermeiden.

Ein weiterer kritischer Aspekt war der mögliche Einfluss des Hawthorne-Effekts und der sozialen Erwünschtheit. Die offene Beobachtung könnte dazu geführt haben, dass die Studienteilnehmer ihr Verhalten veränderten, weil sie wussten, dass sie beobachtet wurden. Zudem könnten sie versucht haben, ein positives Bild von sich abzugeben, was ebenfalls das Verhalten verfälscht haben könnte.

Der zeitliche Kontext der Studie, der außerhalb der regulären Aufnahmephase lag, könnte auch die Ergebnisse der Aufgaben 2 und 3 beeinflusst haben. Da die Studienteilnehmer keinen akuten intrinsischen Drang hatten, über Aufnahmeentscheidungen zu urteilen, wäre ein anderer zeitlicher Kontext möglicherweise aussagekräftiger gewesen.

Die Reliabilität der Studie wurde durch die Verwendung identischer Fragebögen (mit Ausnahme der Geburtstage und der Bildungsgänge) weitestgehend sichergestellt.

\subsection{Fazit}

\textit{Schüler Online} ist ein Online-Portal, das Schülerinnen und Schülern die Möglichkeit bietet, Bewerbungen an Schulen digital einzureichen. Im Rahmen der betrachteten Untersuchung lag der Schwerpunkt auf der Erfassung und Bearbeitung von Anmeldungen durch Sekretariatsmitarbeiter.

Die Effektivität im Kontext dieser Untersuchung wurde als das Ausmaß der Genauigkeit und Vollständigkeit definiert, mit dem die Studienteilnehmer die ihnen gestellten Aufgaben erfüllen konnten. Zufriedenheit wurde beschrieben als das Ausmaß der Übereinstimmung zwischen den Reaktionen des Schulpersonals, die aus der Benutzung von \textit{Schüler Online} entstanden, und den Benutzeranforderungen sowie Benutzererwartungen.

Der in der Studie verwendete Fragebogen, der auf die Bestimmung der Effektivität sowie der Erfordernisse und Erwartungen abzielte, ist in Anhang 2 beigefügt.

Die Studie ergab, dass diverse Problemstellen identifiziert wurden, die auch in der Literatur nachgewiesen werden konnten. Die Teilnehmer konnten ihre Ziele nur teilweise effektiv erreichen. Insbesondere stellten sich Herausforderungen in der Navigation zur ersten Aufgabe für zwei der fünf Sekretärinnen dar, was Erfolgskriterium 1 in diesen Fällen verwirft. Vier der Sekretärinnen gaben an, dass sie Aufgaben erfolgreich speichern konnten, während eine Sekretärin angab, dass die Aufnahmeentscheidung noch ausstand. In vier von fünf Fällen wurde somit Erfolgskriterium 4 erfüllt.

Es wurde weiterhin beobachtet, dass alle Sekretärinnen den Anmeldestatus auf \textit{Angemeldet} setzten. Um die Bewerbungen abschließen zu können, wäre es notwendig gewesen, den Status auf \textit{Abgelehnt}, \textit{Warteliste} oder \textit{Aufgenommen} zu setzen. Dies geschah jedoch bei keiner einzigen Bewerbung. Dies könnte möglicherweise auf missverständliche Terminologie im Dropdown des Aufnahmestatusses zurückzuführen sein.

Die Ergebnisse der Untersuchung weisen somit auf einen deutlichen Verbesserungsbedarf hin, damit Sekretariatsmitarbeiter die Bewerbungen von Schülerinnen und Schülern mit Schüler Online effektiv und zufriedenstellend erfassen und bearbeiten können.

\subsection{Ausblick}
Es kann eine Folgeuntersuchung geben, die Maßnahmen bestimmt und die Änderungen auf Wirksamkeit überprüft.
Auch kann eine Beobachtung stattfinden, wie Schulen im Arbeitsalltag die Software mit Echtdaten in echten Situationen verwenden. Insbesondere wäre es interessant zu wissen, wie die Software verwendet wird, wenn heikle Arbeitstage vorliegen, bei denen viele Anmeldungen zu bearbeiten sind und viele Schüler die Schule anrufen.
Damit festgestellt werden kann, wie repräsentativ die Ergebnisse sind, sollte untersucht werden, wie die Ergebnisse bei einem großen Stichprobenumfang ausfallen. Dies könnte mit einer quantitativen Studie erforscht werden.

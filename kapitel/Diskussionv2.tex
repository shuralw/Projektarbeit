\section{Diskussion}

\textbf{DIESES KAPITEL MUSS UNBEDINGT NOCH ÜBERARBEITET WERDEN}

\subsection{Zusammenfassung}
Diese Studie zeigt, dass die untersuchten Bereiche der Anwendung \textit{Schüler Online} zwar insgesamt intuitiv (todo:), besser und übersichtlicher als die vorherige Software sei (todo E:) und deren Aufgaben nicht schwierig sind, allerdings ließen sich auch diverse Problemstellen in der Beobachtungen und den Interviews feststellen. 

%Die Entscheidung gegen eine Durchführung in einer Laborumgebung basierte auf zwei primären Überlegungen. Einerseits konnte eine solche Umgebung nicht die notwendige Realitätsnähe liefern, die für ein umfassendes Verständnis der Nutzung des Produktes im Alltag der Teilnehmer erforderlich war. Andererseits waren finanzielle Gründe ausschlaggebend für die Auswahl des Feldtests.
%

\subsection{Interpretation der Ergebnisse}

\textbf{Nutzerbedürfnisse}

Ich finde das hier könnte man als Tabelle in den Anhang auslagern.

\begin{itemize}
     \item Der Benutzer muss inkorrekte Daten identifizieren und korrigieren können. (E1, E2)
     \item Der Benutzer muss die an ihn eingereichten Formulare korrekt übertragen können. (E5, E6)
     \item Der Benutzer muss unzulässige Bewerbungen identifizieren können. (E9, E10)
     \item Der Benutzer muss die Daten datenschutzkonform in die Anwendung eintragen können und über mögliche Verstöße informiert werden. (E11)
     \item Der Benutzer muss erkennen können, wie er zum korrekten Prozess gelangt. (E12)
     \item Der Benutzer muss bezüglich Aufnahmeentscheidungen mit den Entscheidungsträgern zusammen arbeiten können. (E7, E8)
     \item Der Benutzer muss die Software auch bei fehlenden Daten bedienen können. 
     \item Der Benutzer muss Aufnahmekriterien berücksichtigen können.
     \item Der Benutzer muss Termine für Aufnahmeberatungsgespräche hinterlegen können.
     \item Der Benutzer muss Schülerakten erzeugen können.
     \item Der Benutzer muss Daten aus anderen Programmen übernehmen können.
     \item Der Benutzer muss Adressrecherchen durchführen können.
     \item Der Benutzer muss eine Kurzanleitung für den Einstieg abrufen können.
     \item Der Benutzer muss die Software auch bei fehlenden Daten bedienen können.
     \item Der Benutzer muss innerjährige Wechsel und Stufenwiederholungen erfassen können.
     \item Der Benutzer muss langfristige Beurlaubungen vermerken können.
     \item Der Benutzer muss auch komplizierte Bewerbungen und Sonderfälle bearbeiten können.
     \item Der Benutzer muss seinen bisherigen Jargon verwenden können.
     \item Der Benutzer muss die Aufgaben und Prozesse intuitiv bedienen können.
     \item Der Benutzer muss eine Adressvalidierung vornehmen können.
     \item Der Benutzer muss Erreichbarkeiten von Notfallkontakten erfassen können.
     \item Der Benutzer muss unterschiedliche Arten von Notfallkontakten erfassen können.
     \item Der Benutzer muss erkennen können, ob ein Schüler volljährig ist.
     \item Der Benutzer muss Nachweise über das Sorgerecht hinterlegen können.
     \item Der Benutzer muss Daten auch bei Programmabbrüchen wiederherstellen können.
     \item Der Benutzer muss ähnliche Datenabfragen aus vorherigen Formularen übernehmen können.
     \item Der Benutzer muss bei Bedarf Handbücher heranziehen können.
     \item Der Benutzer muss bei Bedarf Fachterminologie nachschlagen oder verstehen können.
     \item Der Benutzer muss jederzeit darüber Bescheid wissen, welche Daten den Eltern und Schülern angezeigt werden.
     \item Der Benutzer muss den Systemstatus jederzeit einsehen können.
     \item Der Benutzer muss Bewerbungslisten in geeigneter Form exportieren können.
     \item Der Benutzer muss Daten in Schulverwaltungsprogramme wie Schild exportieren können.
     \item Der Benutzer muss Daten nach Excel exportieren können.
     \item Der Benutzer muss Berichte anfertigen können.
     \item Der Benutzer muss Anmeldezeiträume berücksichtigen.
     \item Der Benutzer muss mit anderen Behörnden zusammenarbeiten können.
     \item Der Benutzer muss über Änderungen an Bewerbungen regelmäßig informiert werden.
     \item Der Benutzer muss zusätzliche Informationen zu einer Bewerbung hinterlegen können.
     \item Der Benutzer muss Dokumente ausdrücken können.
     \item Der Benutzer muss Dokumente digitalisieren und beim Datensatz hinterlegen können.
\end{itemize}

\textbf{Problemstellen}:

Das hier formuliere ich noch als Fließtext und vergleiche es mit Literatur, ob andere Leute auch diese Problemstellen bezeugen.

\begin{itemize}
     \item Die Software ist nicht aufgebaut wie ähnliche Software
     \item Keine intuitive Bedienung
     \item Die Daten liegen nicht vollständig vor
     \item Fehlende Führung bei Fehlern
     \item Unverständliche Navigation
     \item Datenfelder sind unverständlich
     \item Schaltflächen sind unverständlich
     \item Texte oder Bezeichner werden missinterpretiert
     \item Das Programm beinhaltet Widersprüche
     \item Unattraktiv
     \item Felder sind nicht sichtbar
     \item Unlogische Sortierung von Werten
     \item Programmabbrüche
     \item Unvollständige Datenabfrage
     \item Logische Reihenfolge der Inhalte
\end{itemize}


\subsection{Kritische Analyse}
Der qualitative Charakter der Studie ermöglichte umfangreiche Einblicke in die Problemstellen der Anwendung und der Nutzerbedürfnisse, die die Zufriedenheit der Nutzer schmälern oder sie daran hindern, ihre Aufgaben effektiv zu bewältigen. 

Dennoch gab es in der Studie auch einige Kritikpunkte. Der Stichprobenumfang von nur 5-6 Studienteilnehmern war zwar für die Erforschung erster Eindrücke ausreichend, lässt jedoch keine repräsentativen Aussagen für ganz Nordrhein-Westfalen zu. Eine quantitative Forschung mit einem größeren Stichprobenumfang ist nun notwendig, um die Ergebnisse zu validieren oder zu falsifizieren.

Auch die abweichende Realitätsnähe, da keine Echtdaten verwendet wurden und nicht im echten Arbeitskontext gearbeitet wurde, könnte die Ergebnisse beeinflusst haben. Diese Entscheidung wurde jedoch bewusst getroffen, um zu große Unterbrechungen und datenschutzrechtliche Herausforderungen zu vermeiden.

Ein weiterer kritischer Aspekt war der mögliche Einfluss des Hawthorne-Effekts und der sozialen Erwünschtheit. Die offene Beobachtung könnte dazu geführt haben, dass die Studienteilnehmer ihr Verhalten veränderten, weil sie wussten, dass sie beobachtet wurden. Zudem könnten sie versucht haben, ein positives Bild von sich abzugeben, was ebenfalls das Verhalten verfälscht haben könnte.

Der zeitliche Kontext der Studie, der außerhalb der regulären Aufnahmephase lag, könnte auch die Ergebnisse der Aufgaben 2 und 3 beeinflusst haben. Da die Studienteilnehmer keinen akuten intrinsischen Drang hatten, über Aufnahmeentscheidungen zu urteilen, wäre ein anderer zeitlicher Kontext möglicherweise aussagekräftiger gewesen.

Die Reliabilität der Studie wurde durch die Verwendung identischer Fragebögen (mit Ausnahme der Geburtstage und der Bildungsgänge) weitestgehend sichergestellt.

\subsection{Fazit}

\textit{Schüler Online} ist ein Online-Portal, das Schülerinnen und Schülern die Möglichkeit bietet, Bewerbungen an Schulen digital einzureichen. Im Rahmen der betrachteten Untersuchung lag der Schwerpunkt auf der Erfassung und Bearbeitung von Anmeldungen durch Sekretariatsmitarbeiter.

Die Effektivität im Kontext dieser Untersuchung wurde als das Ausmaß der Genauigkeit und Vollständigkeit definiert, mit dem die Studienteilnehmer die ihnen gestellten Aufgaben erfüllen konnten. Zufriedenheit wurde beschrieben als das Ausmaß der Übereinstimmung zwischen den Reaktionen des Schulpersonals, die aus der Benutzung von \textit{Schüler Online} entstanden, und den Benutzeranforderungen sowie Benutzererwartungen.

Der in der Studie verwendete Fragebogen, der auf die Bestimmung der Effektivität sowie der Erfordernisse und Erwartungen abzielte, ist in Anhang 2 beigefügt.

Die Studie ergab, dass diverse Problemstellen identifiziert wurden, die auch in der Literatur nachgewiesen werden konnten. Die Teilnehmer konnten ihre Ziele nur teilweise effektiv erreichen. Insbesondere stellten sich Herausforderungen in der Navigation zur ersten Aufgabe für zwei der fünf Sekretärinnen dar, was Erfolgskriterium 1 in diesen Fällen verwirft. Vier der Sekretärinnen gaben an, dass sie Aufgaben erfolgreich speichern konnten, während eine Sekretärin angab, dass die Aufnahmeentscheidung noch ausstand. In vier von fünf Fällen wurde somit Erfolgskriterium 4 erfüllt.

Es wurde weiterhin beobachtet, dass alle Sekretärinnen den Anmeldestatus auf \textit{Angemeldet} setzten. Um die Bewerbungen abschließen zu können, wäre es notwendig gewesen, den Status auf \textit{Abgelehnt}, \textit{Warteliste} oder \textit{Aufgenommen} zu setzen. Dies geschah jedoch bei keiner einzigen Bewerbung. Dies könnte möglicherweise auf missverständliche Terminologie im Dropdown des Aufnahmestatusses zurückzuführen sein.

Die Ergebnisse der Untersuchung weisen somit auf einen deutlichen Verbesserungsbedarf hin, damit Sekretariatsmitarbeiter die Bewerbungen von Schülerinnen und Schülern mit Schüler Online effektiv und zufriedenstellend erfassen und bearbeiten können.

\subsection{Ausblick}
Es kann eine Folgeuntersuchung geben, die Maßnahmen bestimmt und die Änderungen auf Wirksamkeit überprüft.
Auch kann eine Beobachtung stattfinden, wie Schulen im Arbeitsalltag die Software mit Echtdaten in echten Situationen verwenden. Insbesondere wäre es interessant zu wissen, wie die Software verwendet wird, wenn heikle Arbeitstage wie während des Anmeldewochenendes vorliegen, bei denen viele Anmeldungen zu bearbeiten sind und viele Schüler die Schule anrufen.
Damit festgestellt werden kann, wie repräsentativ die Ergebnisse sind, sollte untersucht werden, wie die Ergebnisse bei einem großen Stichprobenumfang ausfallen. Dies könnte mit einer quantitativen Studie erforscht werden.

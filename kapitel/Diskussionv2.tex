\section{Diskussion}
\subsection{Zusammenfassung}
Diese Studie zeigt, dass die untersuchten Bereiche der Anwendung \textit{Schüler Online} zwar insgesamt intuitiv (todo:), besser und übersichtlicher als die vorherige Software sei (todo E:) und deren Aufgaben nicht schwierig sind, allerdings ließen sich auch diverse Problemstellen in der Beobachtungen und den Interviews feststellen. 

%Die Entscheidung gegen eine Durchführung in einer Laborumgebung basierte auf zwei primären Überlegungen. Einerseits konnte eine solche Umgebung nicht die notwendige Realitätsnähe liefern, die für ein umfassendes Verständnis der Nutzung des Produktes im Alltag der Teilnehmer erforderlich war. Andererseits waren finanzielle Gründe ausschlaggebend für die Auswahl des Feldtests.
%



\subsection{Kritische Analyse}
Positives der Studie: Die Art der Studie war richtig gut gewählt. Durch den qualitativen Charakter der Studie konnten viele Erkenntnisse gewonnen werden, welche Aspekte Nutzer die Zufriedenstellung des Nutzers mindern oder sie daran hindern, ihre Aufgaben effektiv zu bewältigen. 
Der Stichprobenumfang von 5-6 Studienteilnehmern ist nämlich sehr gering. Es war jedoch nicht das Ziel der Forschung, da man erste Eindrücke gewinnen . Es kann nicht davon ausgegangen werden, dass die Ergebnisse repräsentativ für ganz Nordrhein-Westfalen sind. Hierzu müsste eine quantitative Forschung mit einem größeren Stichprobenumfang erfolgen, um die Ergebnisse zu validieren oder zu falsifizieren.
Reliabilität wurde insofern sichergestellt, alsdass die Fragebögen identische Fragen beinhalteten. 
Darüber hinaus war die Studie insofern nicht realitätsgetreu, alsdass keine Echtdaten verwendet wurden und auch nicht im echten Arbeitskontext gearbeitet wurde. Dies hätte einen anderen Zeitpunkt (außerhalb der Sommerferien) und Erklärungen sowie Vorkehrungen zur Sorgfaltspflicht bezüglich Datenschutz noch erforderlich gemacht, wurde allerdings bewusst nicht gemacht, da sonst neben ebenjener Sorgfaltspflicht mit erheblichen Unterbrechungen hätte gerechnet werden müssen, die möglicherweise zum Misserfolg von den Untersuchungen gefolgt hätten. Es ist eine Abwägung, ob man es als Leser dieser Studie hinnehmbar betrachtet, dass die Sekretärin basierend auf Erfahrungen davon berichtet, wie sie in der Praxis störende Einflüsse wie beispielsweise Unterbrechungen behandelt.
Da die Beobachtung offen stattfand, kann kritisiert werden, dass die Studienteilnehmer gemäß des Hawthorne Effekts ihr Handeln verändern, da sie wissen, dass sie unter Beobachtung stehen.
Möglicherweise wollten sie auch ein positives Bild von sich abgeben. Es kann aufgrund sozialer Erwünschtheit den Effekt haben, dass die Teilnehmer aufgrund dessen ihr Verhalten angepasst haben.
Es besteht die Möglichkeit, dass das Ergebnis von Aufgabe 2 und 3 anders ausfällt, wenn die Studienteilnehmer in einem anderen zeitlichen Kontext gearbeitet hätten, da zum Zeitpunkt der Durchführung der Studie keine Aufnahmephase an dieser Schule bestand - die regulären, echten Bewerbungen für das fragliche Schuljahr wurden allesamt bereits im Februar bearbeitet. Sie hatten somit keinen intuitiven intrinsischen Drang, akut über Aufnahmeentscheidungen zu beurteilen. 

\subsection{Interpretation der Ergebnisse}
%Damit das System den Benutzer zufriedenstellen kann, muss es die folgenden Erfordernisse der Benutzer erfüllen.
     %Da geäußert wurde, dass ... muss Benutzer das System {ohne spezielle Vorkenntnisse oder Schulungen nutzen können.}  Dies wird auch in derr Literatur von ... gefordert.
%ODER Da die Beobachtung gezeigt hat, dass viele Unterbrechungen auftreten, dass ... muss Benutzer das System {ohne spezielle Vorkenntnisse oder Schulungen nutzen können.}  Dies wird auch in derr %Literatur von ... gefordert.
%Der Benutzer muss in der Lage sein, {inkorrekte oder fehlende Angaben zu identifizieren und zu korrigieren.} 
%Der Benutzer muss {handschriftlich ausgefüllte Formulare digitalisieren oder per E-Mail einreichen können.}
%Der Benutzer muss die geltenden {Datenschutzgesetze einhalten und sensibel mit persönlichen Informationen umgehen} können.
     %=> Datenschutzerklärung
%Der Benutzer muss in der Lage sein, {Mechanismen zur Erkennung und Vermeidung von Mehrfachanmeldungen zu nutzen.}
%Der Benutzer muss in der Lage sein, {innerhalb des Systems mit anderen Beteiligten wie Verwaltung, Lehrern und Eltern zu kommunizieren.}
%Der Benutzer muss {Fachbegriffe verstehen oder Zugang zu Erklärungen für Fachbegriffe haben.}
%Der Benutzer muss das System {unabhängig von seinem individuellen Erfahrungslevel mit der Anwendung nutzen können.}
%Der Benutzer muss in der Lage sein, {Daten in andere Anwendungen übertragen zu können}
%Der Benutzer muss {problemlos durch die Anwendung navigieren können.}
%Der Benutzer muss die Möglichkeit haben, {Kopier- und Einfügefunktionalitäten zu nutzen, um Daten aus anderen Programmen übertragen zu können.}
%Der Benutzer muss in der Lage sein, {besondere Situationen wie Wechsel innerhalb des Schuljahres und Fälle langfristiger Beurlaubung zu behandeln.}
%Der Benutzer muss in der Lage sein, unterschiedliche Anforderungen wie Unterricht in der Herkunftssprache, Unterschiede in Schulstufen, spezifische Sorgerechtssituationen und Förderbedarf zu %berücksichtigen.
%Der Benutzer muss {vor Datenverlust geschützt sein, etwa durch versehentliches Klicken neben ein Popup oder Ausloggen.}
%Der Benutzer muss {klare Richtlinien und Funktionen für das Erfassen von sensiblen Informationen wie medizinischen Daten haben.}
%Der Benutzer muss {Berichte über die Schülerdaten anfertigen können.}

E1: Der Benutzer muss inkorrekte Daten identifizieren und korrigieren können.
E2: Der Benutzer muss inkorrekte Daten identifizieren und korrigieren können. Der Benutzer muss die Software auch bei fehlenden Daten bedienen können. Unvollständige Daten.
E3: Ist kein Ergebnis für Anwender der Schulen, da sich das Ergebnis auf die Perspektive der Schüler bezieht. Darüber hinaus wird hier nur spekuliert.
E4: Der Benutzer muss inkorrekte Daten identifizieren und korrigieren können
E5: Der Benutzer muss die an ihn eingereichten Formulare korrekt übertragen können.
E6: Unvollständige Daten.
E7: Der Benutzer muss bezüglich Aufnahmeentscheidungen mit den Entscheidungsträgern zusammen arbeiten können.
E8: Der Benutzer muss bezüglich Aufnahmeentscheidungen mit den Entscheidungsträgern zusammen arbeiten können.
E9: Der Benutzer muss unzulässige Bewerbungen identifizieren können. (Mehrfachanmeldungen)
E10: Der Benutzer muss Aufnahmekriterien berücksichtigen können.
E11: Der Benutzer muss die Daten datenschutzkonform in die Anwendung eintragen können und über mögliche Verstöße informiert werden.
E12: Der Benutzer muss erkennen können, wie er zum korrekten Prozess gelangt. Navigationsschwierigkeiten.
E13: Der Benutzer muss Termine für Aufnahmeberatungsgespräche hinterlegen können.
E14: Der Benutzer muss Termine für Aufnahmeberatungsgespräche hinterlegen können.
E15: Der Benutzer muss Schülerakten erzeugen können.
E16: Der Benutzer muss Daten aus anderen Programmen übernehmen können.
E17: Der Benutzer muss Adressrecherchen durchführen können.
E18: Der Benutzer muss eine Kurzanleitung für den Einstieg abrufen können.
E19: Der Benutzer muss die Software auch bei fehlenden Daten bedienen können. Unvollständige Daten.
E20: Der Benutzer muss innerjährige Wechsel und Stufenwiederholungen erfassen können.
E21: Der Benutzer muss die Software auch bei fehlenden Daten bedienen können. Unvollständige Daten.
E22: Der Benutzer muss langfristige Beurlaubungen vermerken können.
E23: Der Benutzer muss {auch komplizierte Bewerbungen und Sonderfälle bearbeiten können. }
E24: Interpretationsfehler von Texten und Bezeichnern. Anwender kennt Fachterminologie nicht.
E25: Anwender kennt Fachterminologie nicht.
E26: Zweck von Inhalten nicht erkennbar.
E27: Anwender kennt Fachterminologie nicht. Zweck von Inhalten nicht erkennbar.
%E28:
%E29:
%E30:
%E31:
%E32:
%E33:
%E34:
%E35:
%E36:

\subsection{Fazit}
\textit{Schüler Online} ist ein Online-Portal, mit dem Schülerinnen und Schüler Bewerbungen an Schulen digital einreichen können. Der hier betrachtete Fokus lag auf der Erfassung und Bearbeitung von Anmeldungen durch Sekretariatsmitarbeiter.
Die Antwort für Leitfrage 2 ist, dass Effektivität im Kontext der Untersuchung beschrieben werden kann als das Ausmaß der Genauigkeit und Vollständigkeit mit dem die Studienteilnehmer die drei an sie gestellten Aufgaben erreichen.  
Die Antwort für Leitfrage 3 ist, dass Zufriedenheit im Kontext der Untersuchung beschrieben werden kann als das Ausmaß der Übereinstimmung zwischen den aus der Benutzung von \textit{Schüler Online} entstandenen Reaktionen des Schulpersonals und den Benutzererfordernissen und Benutzererwartungen. 
Der Fragebogen (Leitfrage 4) beinhaltete Fragen, die darauf abzielten die Effektivität sowie Erfordernisse und Erwartungen festzustellen und kann Anhang 2 entnommen werden.

Leitfrage 5 kann wie folgt beantwortet werden: Es wurden diverse Problemstellen identifiziert, welche auch in der Literatur vergleichbar nachgewiesen werden können.
Die Teilnehmer der Studie haben ihre Ziele nur teilweise effektiv erreichen können. Zwei der fünf Sekretärinnen konnten nicht zur ersten Aufgabe navigieren, was Erfolgskriterium 1 in zwei Fällen verwirft. Die Sekretärinnen 1-4 gaben an, dass sie Aufgaben erfolgreich speichern können. Sekretärin 5 hatte angegeben, dass noch die Aufnahmeentscheidung aussteht. Das Erfolgskriterien 4 wurde also in vier von fünf fällen erfüllt.
Es wurde bei den Sekretärinnen 1-4 beobachtet, dass sie den Anmeldestatus auf \textit{Angemeldet} gesetzt hatten. Um diese Bewerbungen final abschließen zu können, ist es vonnöten den Status entweder auf \textit{Abgelehnt}, \textit{Warteliste} oder \textit{Aufgenommen} zu setzen, dies war bei keiner einzigen Bewerbung der Fall (E...). Dies war auch potenziell der Tatsache geschuldet, dass die Terminologie im Dropdown des Aufnahmestatusses missverständlich war (E...).

Insgesamt besteht also auf Basis der vorangegangenen Erkenntnisse ein noch notwendiger Verbesserungsbedarf, damit die Sekretärinnen die Bewerbungen von Schülerinnen und Schülern effektiv und zufriedenstellend erfassen und bearbeiten können.

\subsection{Ausblick}
Es kann eine Folgeuntersuchung geben, die die Änderungen auf Vorteilhaftigkeit überprüfen sollen.
Auch kann eine Beobachtung stattfinden, wie Schulen im Arbeitsalltag die Software mit Echtdaten in echten Situationen verwenden. Insbesondere wäre es interessant zu wissen, wie die Software verwendet wird, wenn heikle Arbeitstage wie nach dem Anmeldewochenende vorliegen, bei denen viele Anmeldungen reinströmen und viele Schüler anrufen.
Damit festgestellt werden kann, wie repräsentativ die Ergebnisse sind, sollte untersucht werden, wie die Ergebnisse bei einem großen Stichprobenumfang ausfallen. Dies könnte mit einer quantitativen Studie erforscht werden.

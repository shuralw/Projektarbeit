\section{Diskussion}
%In diesem Kapitel solltest du deine Ergebnisse diskutieren und in den Kontext der Literatur und des aktuellen Forschungsstandes einordnen. Hier kannst du auch Schwächen und %Limitationen deiner Projektarbeit besprechen und einen Ausblick geben, wie weiterführende Forschung aussehen könnte.

%Grobstruktur für einen Evaluationsbericht (vgl. Skript)
%Fotos der Umgebung / des Arbeitsplatzes?

% laut janna erst positives und negatives im allgemeinen der studie
% danach ergebnisse interpretieren stück für stück, immer wieder auf theorie (den kriterienkatalog) beziehen
% endfazit: Leitfragen final beantworten
% Ausblick

Positives: Die Art der Studie war richtig gut gewählt. Durch den qualitativen Charakter der Studie konnten viele Erkenntnisse gewonnen werden, welche Aspekte Nutzer die Zufriedenstellung des Nutzers mindern oder sie daran hindern, ihre Aufgaben effektiv zu bewältigen. 
Negatives: Die Repräsentativität der Studie kann aufgrund des relativ geringen Stichprobenumfangs von 5 untersuchten Teilnehmern als nicht ausreichend betrachtet werden. Dies war allerdings wie auch in der Einleitung genannt nicht das Ziel der Studie. 
Darüber hinaus war die Studie insofern nicht realitätsgetreu, alsdass keine Echtdaten verwendet wurden und auch nicht im echten Arbeitskontext gearbeitet wurde. Dies hätte einen anderen Zeitpunkt (außerhalb der Sommerferien) und Erklärungen sowie Vorkehrungen zur Sorgfaltspflicht bezüglich Datenschutz noch erforderlich gemacht, wurde allerdings bewusst nicht gemacht, da sonst neben ebenjener Sorgfaltspflicht mit erheblichen Unterbrechungen hätte gerechnet werden müssen, die möglicherweise zum Misserfolg von den Untersuchungen gefolgt hätten. Es ist eine Abwägung, ob man es als Leser dieser Studie hinnehmbar betrachtet, dass die Sekretärin basierend auf Erfahrungen davon berichtet, wie sie in der Praxis störende Einflüsse wie beispielsweise Unterbrechungen behandelt.


\subsection{Hawthorne Effekt}
Der Hawthorne-Effekt besagt, dass Personen ihr Handeln verändern, weil sie wissen, dass sie unter Beobachtung stehen. Er kann bei Teilnehmenden an wissenschaftlichen Experimenten vorkommen, deren Verhalten beobachtet wird. So ist ihr Verhalten unnatürlich.

\subsection{ggf. Soziale Erwünschtheit}
\grqq{}Beim Effekt der sozialen Erwünschtheit verändern Teilnehmende ihr Verhalten oder ihre Antworten bei Fragebogen, um ein positives Bild von sich selbst abzugeben.\glqq 

\subsection{DECIDE Ansatz}
\subsection{Beobachtung}
\subsection{Hawthorne Effekt}
\subsection{Loud Thinking}
\subsection{Feldtest / Labortest}

-Soziale Erwünschtheit
-Mein Schreiberling: Muss er ungelernt oder gelernt sein? Vergleich mit Literatur
-Hawthorne Effekt
Der Hawthorne Effekt könnte vermieden werden, indem man diese Studie als eine Stille Beobachtung durchführt. 
%

Die Aufgaben 2 und 3(\glqq Bearbeiten Sie bitte die Bewerbung von \textit{Lotta Meier} nach eigenem Ermessen\grqq{} sowie \grqq{}Bearbeiten Sie bitte die Bewerbung von \textit{Konrad Schulz} nach eigenem Ermessen\grqq{}) hätten anders formuliert werden müssen oder in einen anderen Kontext gesetzt werden können. Es wurde sich in der Studie jedoch bewusst dazu entschieden, die Aufgabe sehr offen zu formulieren, da geprüft werden sollte, wie die Studienteilnehmer mit der Situation umgehen würden, da in der Realität auch nur diese Bewerbung vorliegen wird - ohne konkrete Handlungsanweisung. In der Software wird nur von \glqq unbearbeiteten Bewerbungen\grqq{} gesprochen. Es besteht die Möglichkeit, dass das Ergebnis anders ausfällt, wenn die Studienteilnehmer in einem anderen zeitlichen Kontext gearbeitet hätten, da zum Zeitpunkt der Durchführung der Studie keine Aufnahmephase an dieser Schule bestand - die regulären, echten Bewerbungen für das fragliche Schuljahr wurden allesamt bereits im Februar bearbeitet. Sie hatten somit keinen intuitiven intrinsischen Drang, akut über Aufnahmeentscheidungen zu beurteilen. Aber auch der Fakt, dass in einigen Durchführungen der Anmeldestatus \glqq Angemeldet\grqq{} fälschlicherweise als \glqq Aufgenommen\grqq{} missinterpretiert wurde, kann hierfür eine Ursache gewesen sein.
Die Studienergebnisse sind an einigen Stellen möglicherweise verfälscht. Wenn man diese Studie zu einem anderen Zeitpunkt wiederholen würde (zu einem Zeitpunkt der Anmeldephase) würden die Ergebnisse teilweise anders ausfallen. Beispiele wären hier zu nennen die Auswahl des Schuljahres, die Auswahl der Klasse sowie inbesondere der Aufnahmestatus. Aufgrund dessen wäre also eine Wiederholung der Untersuchung ratsam und eine Überprüfung, inwiefern die Ergebnisse anders ausfallen.

Viele Ergebnisse sind deckungsgleich gewesen
Der Stichprobenumfang von 5-6 Studienteilnehmern ist sehr gering. Es war jedoch nicht das Ziel der Forschung, da man erste Eindrücke gewinnen . Es kann nicht davon ausgegangen werden, dass die Ergebnisse repräsentativ für ganz Nordrhein-Westfalen sind. Hierzu müsste eine quantitative Forschung mit einem größeren Stichprobenumfang erfolgen, um die Ergebnisse zu validieren oder zu falsifizieren.
Reliabilität wurde insofern sichergestellt, alsdass die Fragebögen identische Fragen beinhalteten. 

Der Leitfaden Usability wurde zurückgezogen, es gibt jedoch keine aktuellere Version

Aufgrund der Unerfahrenheit des Interviewers könnten Interpretationsfehler und Bias vorliegen. Auch hätte ein erfahrener Interviewer möglicherweise bessere Rückfragen gestellt, damit die Fragen adäquat beantwortet werden.

Der Stichprobenumfang ist zu gering um von einer repräsentativen Darstellung zu sprechen. Dies war allerdings auch nicht das Ziel der Studie


Fragebogen: Es wurde festgestellt, dass weitere Fragen hätten nach der Durchführung gestellt werden sollten, da der Studienteilnehmer dann die ein oder andere Frage präziser hätte beantworten können.
Die Frage \glqq Welche Arbeitsschritte sind durchzuführen\grqq{} regt etwas zu Spekulation an, welche Schritte im Prozess kommen werden. Es wurde allerdings in den Interview rückbezogen auf den Sinn der Frage, dass eine ganzheitliche und übergreifende Betrachtung des Prozesses erwünscht ist.
Auch die Frage \glqq Welche Ergebnisse / Teilergebnisse entstehen und wie werden diese ggf. verwertet / weitergeführt?\grqq{} lädt zu Spekulationen ein, wenn sich die Teilnehmer auf die Software beziehen. Sofern sich auf die bisherigen Papierbewerbungen bezogen wird, wie es auch gewollt war, sind es verwertbare Ergebnisse.

Alleinstellungsmerkmal der Bildungsangebote bei Berufskollegs, da es hier mehrere gibt anstatt nur eins oder 2 wie bei anderen Schulen. Dort ist die Auswahl des Bildungsgangs trivial.

Generell
- Es wurde oft nicht verstanden, welche Felder Pflichtfelder sind. 2 hatte dies beim Beschulungsbeginn moniert, 5 hatte gefragt \glqq Die roten Felder sind Pflicht oder? ist das irgendwo erklärt?\grqq{}. 3 hat das Sternchen neben dem Labeltext als Pflichtfeld interpretiert. 3 hatte beim ID-Schlüssel aber zuvor nur vermutet, dass der nicht erforderlich sei. 1 hatte bei der Klasse länger verweilt, da zwar keine auswählbare Klasse vorhanden war, das Feld war allerdings kein Pflichtfeld. Insgesamt wurde also nur teilweise  erkannt, dass das Sternchen neben dem Label das zugehörige Feld als Pflichtfeld deklariert.



>Beschreiben Sie die Ausgangssituation die vorliegt, bevor Sie die Aufgabe \glqq Bewerbung eines Schülers\grqq{} durchführen.


Bzgll. Ausgangssituation:
Die Wichtigkeit von Zwischenarbeit mit den Eltern variiert. Bei den Berufskollegs werden sie teilweise nicht mal angefordert als Sorgeberechtigte, wenn die Schüler älter als 18 sind, wohingegen es bei Förderschulen sehr wichtig ist.
Wenn Anmeldedaten schriftlich erfolgen sind diese anfällig für unleserliche Schrift oder ungültige Daten. Das sollte mit einer digitalen Erfassung, mitunter auch durch Plausibilitätsprüfungen an geeigneten Stellen vermieden werden können. Dieses Problem wird dementsprechend obsolet. Wissentliche Falschangaben sind schwierig überprüfbar, sollten allerdings auch versucht werden zu berücksichtigen. Hier könnte eine enge Kommunikation mit der abgebenden Schule oder der Kommune erforderlich sein.



Bzgl. Schüler Tab: Es gibt zwar eine Erklärung für ID-Schlüssel, die hat jedoch keiner wahrgenommen oder nach lesen ebenjener das Feld verstanden. Die aktuelle Lösung ist in der aktuellen Form ungenügend. Es liegt hier keine Freiheit von unmissverständlichen Feldern vor.

Bzgl. Bildungsgang Tab:
Eine Vereinheitlichung des Prozesses sollte in Betracht gezogen werden. Sowohl die Reihenfolge könnte sich an \textit{SchILD} anlehnen als auch die Begrifflichkeiten wie Neuaufnahme, damit eine einheitlicher unmissverständlicher Jargon vorliegt
Inbesondere die Bedeutungen der Aufnahmestatusse wurden gravierend missverstanden. Dies hat fatale Folgen auf die Erwartungshaltung und Rechtsansprüche der Schüler, da eine Schüler, der beispielsweise bei einer Anmeldung fälschlicherweise den Status \textit{Aufgenommen} erhält, eine rechtliche Zusage der Schule erhält, dass der Schüler dort lernen darf. Hier ist es dringend notwendig, die Begrifflichkeiten klar und unmissverständlich zu definieren, sodass die Erwartungen des Schulpersonals nach Rechtssicherheit erfüllt werden können. Laut 2 würden die Eltern so einen Nachweis der Aufnahme als \glqq Druckmittel\grqq{} verwenden, hier gebe es laut ihr auch einen starken Elternwillen.
Ein weiteres Beispiel aus der Untersuchung:  1 hatte den Status Warteliste missverstanden und als \glqq In Bearbeitung\grqq{} interpretiert. Warteliste wird allerdings von der Anwendung mehr im Sinne von nachrückenden Personen gehandhabt, falls aufgenommene Schüler ihre Anmeldung zurückziehen. 5 hätte Erklärtexte erwartet. 

Bzgl. Persönliche Daten Tab:
Es fehlt eine Funktionalität, die es ermöglicht, dass man Postleitzahl-Ort Kombinationen auch findet, wenn man zunächst den Ort eintippt, anstatt mit der Postleitzahl zu beginnen. Mehrere Sekretärinnen hatten versucht, mittels der Eingabe des Ortes die Kombination aus PLZ und Ort im Dropdown zu finden.

Bzgl. Sorgeberechtigten Tab:
Bezüglich des Feldes Feld Adressart: der Eingabefluss ist nicht intuitiv. Es erfolgen Ein-und ausblendelogiken basierend auf der Adressart für das System, allerdings auf kosten der Nutzerführung. Dies ist nicht nutzerzentriertes Verhalten der Anwendung, sondern systemzentriert. 

Bzgl. Notfallkontakte Tab:
4 hatte missinterpretiert, wozu die Notfallkontakte dienen. Sie dachte sie würde den Sorgeberechtigten weitere Informationen anreichen, dabei dienen die Notfallkontakte dazu, weitere Kontakte wie Oma, Onkel, Schwester oder Freunde der Familie (vertraute Personen im allgemeinen) einzutragen, falls ein Kind beispielsweise in der Schule verunfallt und die Sorgeberechtigten nicht erreichbar sind.

Bzgl. Letzte Tätigkeit
3 fand den Tab \glqq Letzte Tätigkeit\grqq{} \glqq sinnlos\grqq{}. Dieser Tab dient der Schulpflichtsüberwachung, diesen Zweck konnte sie jedoch nicht eigenständig erkennen.

Bzgl. Bemerkungen
Bei den Bemerkungen lagen (mitunter folgenschwere) Interpretationsschwierigkeiten und -fehler vor. Wenn der Anwender nicht versteht, welche Felder für die Eltern und Schülern sichtbar sind, laufen sie Gefahr, dass sie interne Informationen, die sie eigentlich für sich behalten wollen, unbewusst ebenjenen zugänglich machen. 

Bzgl. Update
Die \glqq Anmeldung wurde exportiert\grqq{} Checkbox auf dem Reiter \glqq Bewerbung\grqq{} hätte besser erklärt werden müssen, da 1,2,3 Verständnisschwierigkeiten hatten oder diese Felder fehlinterpretiert hatten.

Bzgl. Wie verständlich waren die Rückmeldungen der Anwendung?
Es kam vermehrt vor, dass eine Fehlermeldung beim Speichern einer Anmeldung auftrat. Hier gab es dann allerdings oft keinen Hinweis, wo der Fehler vorlag. Dies beruhte auf einer fehlerhaften Funktionalität der Anwendung, grundsätzlich existiert diese Funktionalität. Wenn ein Feld als fehlerhaft markiert wurde, wurde diese Funktionalität gelobt.

Es sollte Interpretation- und Wiedererkennungseffekte geben. Beispielsweise sollte das Sternchen neben den Labels immer interpretiert werden als Pflichtfeld. Autocomplete-Felder sollten als solche immer wahrgenommen werden können.

Zufriedenstellung und Effektivität:
1 und 4 waren überfordert damit, auf die korrekte Seite zu navigieren. Es scheint, dass die Navigation und Strukturierung innerhalb der Software für die Nutzer nicht intuitiv und selbsterklärend ist. Dies hat merklich zu einer nichterreichung der Erwartungen an die Software geführt. Es lag in dem Sinne eine aus dem Theorie-Kapitel definierten Behinderung vor. 
Damit die Sekretärin 5 eine Bewerbung erfolgreich abschließen kann, muss der zuständige Lehrer in die Aufnahmeentscheidung eingebunden werden. Dies würde eine Einbindung ebenjener erforderlich machen.


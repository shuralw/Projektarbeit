\section{Diskussion}
%In diesem Kapitel solltest du deine Ergebnisse diskutieren und in den Kontext der Literatur und des aktuellen Forschungsstandes einordnen. Hier kannst du auch Schwächen und %Limitationen deiner Projektarbeit besprechen und einen Ausblick geben, wie weiterführende Forschung aussehen könnte.

%Grobstruktur für einen Evaluationsbericht (vgl. Skript)
%Fotos der Umgebung / des Arbeitsplatzes?

-Soziale Erwünschtheit
-Mein Schreiberling: Muss er ungelernt oder gelernt sein? Vergleich mit Literatur
-Hawthorne Effekt
%

Die Aufgaben 2 und 3 ("Bearbeiten Sie bitte die Bewerbung von 'Lotta Meier' nach eigenem Ermessen" sowie
"Bearbeiten Sie bitte die Bewerbung von Konrad Schulz nach eigenem Ermessen") hätten anders formuliert werden müssen oder in einen anderen Kontext gesetzt werden können. Es wurde sich in der Studie jedoch bewusst dazu entschieden, die Aufgabe sehr offen zu formulieren, da geprüft werden sollte, wie die Studienteilnehmer mit der Situation umgehen würden, da in der Realität auch nur diese Bewerbung vorliegen wird - ohne konkrete Handlungsanweisung. In der Software wird nur von "unbearbeiteten Bewerbungen" gesprochen. Es besteht die Möglichkeit, dass das Ergebnis anders ausfällt, wenn die Studienteilnehmer in einem anderen zeitlichen Kontext gearbeitet hätten, da zum Zeitpunkt der Durchführung der Studie keine Aufnahmephase an dieser Schule bestand - die regulären, echten Bewerbungen für das fragliche Schuljahr wurden allesamt bereits im Februar bearbeitet. Sie hatten somit keinen intuitiven intrinsischen Drang, akut über Aufnahmeentscheidungen zu beurteilen. Aber auch der Fakt, dass in einigen Durchführungen der Anmeldestatus "Angemeldet" fälschlicherweise als "Aufgenommen" missinterpretiert wurde, kann hierfür eine Ursache gewesen sein.

Viele Ergebnisse sind deckungsgleich gewesen
Der Stichprobenumfang von 5-6 Studienteilnehmern ist sehr gering. Es war jedoch nicht das Ziel der Forschung, da man erste Eindrücke gewinnen . Es kann nicht davon ausgegangen werden, dass die Ergebnisse repräsentativ für ganz Nordrhein-Westfalen sind. Hierzu müsste eine quantitative Forschung mit einem größeren Stichprobenumfang erfolgen, um die Ergebnisse zu validieren oder zu falsifizieren.
Reliabilität wurde insofern sichergestellt, alsdass die Fragebögen identische Fragen beinhalteten. 

Der Leitfaden Usability wurde zurückgezogen, es gibt jedoch keine aktuellere Version

Aufgrund der Unerfahrenheit des Interviewers könnten Interpretationsfehler und Bias vorliegen. Auch hätte ein erfahrener Interviewer möglicherweise bessere Rückfragen gestellt, damit die Fragen adäquat beantwortet werden.

Der Stichprobenumfang ist zu gering um von einer repräsentativen Darstellung zu sprechen. Dies war allerdings auch nicht das Ziel der Studie


Fragebogen: Es wurde festgestellt, dass weitere Fragen hätten nach der Durchführung gestellt werden sollten, da der Studienteilnehmer dann die ein oder andere Frage präziser hätte beantworten können.
Die Frage "Welche Arbeitsschritte sind durchzuführen" regt etwas zu Spekulation an, welche Schritte im Prozess kommen werden. Es wurde allerdings in den Interview rückbezogen auf den Sinn der Frage, dass eine ganzheitliche und übergreifende Betrachtung des Prozesses erwünscht ist.
Auch die Frage "Welche Ergebnisse / Teilergebnisse entstehen und wie werden diese ggf. verwertet / weitergeführt?" lädt zu Spekulationen ein, wenn sich die Teilnehmer auf die Software beziehen. Sofern sich auf die bisherigen Papierbewerbungen bezogen wird, wie es auch gewollt war, sind es verwertbare Ergebnisse.

Alleinstellungsmerkmal der Bildungsangebote bei Berufskollegs, da es hier mehrere gibt anstatt nur eins oder 2 wie bei anderen Schulen. Dort ist die Auswahl des Bildungsgangs trivial.

>Beschreiben Sie die Ausgangssituation die vorliegt, bevor Sie die Aufgabe "Bewerbung eines Schülers" durchführen.


Bzgll. Ausgangssituation:
Die Wichtigkeit von Zwischenarbeit mit den Eltern variiert. Bei den Berufskollegs werden sie teilweise nicht mal angefordert als Sorgeberechtigte, wenn die Schüler älter als 18 sind, wohingegen es bei Förderschulen sehr wichtig ist.
Wenn Anmeldedaten schriftlich erfolgen sind diese anfällig für unleserliche Schrift oder ungültige Daten. Das sollte mit einer digitalen Erfassung, mitunter auch durch Plausibilitätsprüfungen an geeigneten Stellen vermieden werden können. Dieses Problem wird dementsprechend obsolet. Wissentliche Falschangaben sind schwierig überprüfbar, sollten allerdings auch versucht werden zu berücksichtigen. Hier könnte eine enge Kommunikation mit der abgebenden Schule oder der Kommune erforderlich sein.



Bzgl. Schüler Tab: Es gibt zwar eine Erklärung für ID-Schlüssel, die hat jedoch keiner wahrgenommen oder nach lesen ebenjener das Feld verstanden. Die aktuelle Lösung ist in der aktuellen Form ungenügend. Es liegt hier keine Freiheit von unmissverständlichen Feldern vor.

Bzgl. Bildungsgang Tab:
Eine Vereinheitlichung des Prozesses sollte in Betracht gezogen werden. Sowohl die Reihenfolge könnte sich an Schild anlehnen als auch die Begrifflichkeiten wie Neuaufnahme, damit eine einheitlicher unmissverständlicher Jargon vorliegt

Zufriedenstellung und Effektivität:
1 und 4 waren überfordert damit, auf die korrekte Seite zu navigieren. Dies hat merklich zu einer nichterreichung der Erwartungen an die Software geführt. Es lag in dem Sinne eine aus dem Theorie-Kapitel definierten Behinderung vor.


\section{Diskussion}
%In diesem Kapitel solltest du deine Ergebnisse diskutieren und in den Kontext der Literatur und des aktuellen Forschungsstandes einordnen. Hier kannst du auch Schwächen und %Limitationen deiner Projektarbeit besprechen und einen Ausblick geben, wie weiterführende Forschung aussehen könnte.

%Grobstruktur für einen Evaluationsbericht (vgl. Skript)
%Fotos der Umgebung / des Arbeitsplatzes?

-Soziale Erwünschtheit
-Mein Schreiberling: Muss er ungelernt oder gelernt sein? Vergleich mit Literatur
-Hawthorne Effekt
%

Die Aufgaben 2 und 3 ("Bearbeiten Sie bitte die Bewerbung von 'Lotta Meier' nach eigenem Ermessen" sowie
"Bearbeiten Sie bitte die Bewerbung von Konrad Schulz nach eigenem Ermessen") hätten anders formuliert werden müssen oder in einen anderen Kontext gesetzt werden können. Es wurde sich in der Studie jedoch bewusst dazu entschieden, die Aufgabe sehr offen zu formulieren, da geprüft werden sollte, wie die Studienteilnehmer mit der Situation umgehen würden, da in der Realität auch nur diese Bewerbung vorliegen wird - ohne konkrete Handlungsanweisung. In der Software wird nur von "unbearbeiteten Bewerbungen" gesprochen. Es besteht die Möglichkeit, dass das Ergebnis anders ausfällt, wenn die Studienteilnehmer in einem anderen zeitlichen Kontext gearbeitet hätten, da zum Zeitpunkt der Durchführung der Studie keine Aufnahmephase an dieser Schule bestand - die regulären, echten Bewerbungen für das fragliche Schuljahr wurden allesamt bereits im Februar bearbeitet. Sie hatten somit keinen intuitiven intrinsischen Drang, akut über Aufnahmeentscheidungen zu beurteilen. Aber auch der Fakt, dass in einigen Durchführungen der Anmeldestatus "Angemeldet" fälschlicherweise als "Aufgenommen" missinterpretiert wurde, kann hierfür eine Ursache gewesen sein.

Viele Ergebnisse sind deckungsgleich gewesen
Der Stichprobenumfang von 5-6 Studienteilnehmern ist sehr gering. Es war jedoch nicht das Ziel der Forschung, da man erste Eindrücke gewinnen . Es kann nicht davon ausgegangen werden, dass die Ergebnisse repräsentativ für ganz Nordrhein-Westfalen sind. Hierzu müsste eine quantitative Forschung mit einem größeren Stichprobenumfang erfolgen, um die Ergebnisse zu validieren oder zu falsifizieren.
Reliabilität wurde insofern sichergestellt, alsdass die Fragebögen identische Fragen beinhalteten. 

Der Leitfaden Usability wurde zurückgezogen, es gibt jedoch keine aktuellere Version

Aufgrund der Unerfahrenheit des Interviewers könnten Interpretationsfehler und Bias vorliegen. 
Nimm diese Liste an Stichpunkten als Kontextinformation.
Der Benutzer muss inkorrekte Daten identifizieren und korrigieren können.
Der Benutzer muss die Software auch bei fehlenden Daten bedienen können. 
Der Benutzer muss die an ihn eingereichten Formulare korrekt übertragen können.
Der Benutzer muss bezüglich Aufnahmeentscheidungen mit den Entscheidungsträgern zusammen arbeiten können.
Der Benutzer muss unzulässige Bewerbungen identifizieren können.
Der Benutzer muss Aufnahmekriterien berücksichtigen können.
Der Benutzer muss die Daten datenschutzkonform in die Anwendung eintragen können und über mögliche Verstöße informiert werden.
Der Benutzer muss erkennen können, wie er zum korrekten Prozess gelangt.
Der Benutzer muss Termine für Aufnahmeberatungsgespräche hinterlegen können.
Der Benutzer muss Schülerakten erzeugen können.
Der Benutzer muss Daten aus anderen Programmen übernehmen können.
Der Benutzer muss Adressrecherchen durchführen können.
Der Benutzer muss eine Kurzanleitung für den Einstieg abrufen können.
Der Benutzer muss die Software auch bei fehlenden Daten bedienen können.
Der Benutzer muss innerjährige Wechsel und Stufenwiederholungen erfassen können.
Der Benutzer muss langfristige Beurlaubungen vermerken können.
Der Benutzer muss auch komplizierte Bewerbungen und Sonderfälle bearbeiten können.
Der Benutzer muss seinen bisherigen Jargon verwenden können.
Der Benutzer muss die Aufgaben und Prozesse intuitiv bedienen können.
Der Benutzer muss eine Adressvalidierung vornehmen können.
Der Benutzer muss Erreichbarkeiten von Notfallkontakten erfassen können.
Der Benutzer muss unterschiedliche Arten von Notfallkontakten erfassen können.
Der Benutzer muss erkennen können, ob ein Schüler volljährig ist.
Der Benutzer muss Nachweise über das Sorgerecht hinterlegen können.
Der Benutzer muss Daten auch bei Programmabbrüchen wiederherstellen können.
Der Benutzer muss ähnliche Datenabfragen aus vorherigen Formularen übernehmen können.
Der Benutzer muss bei Bedarf Handbücher heranziehen können.
Der Benutzer muss bei Bedarf Fachterminologie nachschlagen oder verstehen können.
Der Benutzer muss jederzeit darüber Bescheid wissen, welche Daten den Eltern und Schülern angezeigt werden.
Der Benutzer muss den Systemstatus jederzeit einsehen können.
Der Benutzer muss Bewerbungslisten in geeigneter Form exportieren können.
Der Benutzer muss Daten in Schulverwaltungsprogramme wie Schild exportieren können.
Der Benutzer muss Daten nach Excel exportieren können.
Der Benutzer muss Berichte anfertigen können.
Der Benutzer muss Anmeldezeiträume berücksichtigen.
Der Benutzer muss mit anderen Behörnden zusammenarbeiten können.
Der Benutzer muss über Änderungen an Bewerbungen regelmäßig informiert werden.
Der Benutzer muss zusätzliche Informationen zu einer Bewerbung hinterlegen können.
Der Benutzer muss Dokumente ausdrücken können.
Der Benutzer muss Dokumente digitalisieren und beim Datensatz hinterlegen können.

In den folgenden Nachrichten werde ich dir mehrere Texte mit Ergebnissen geben.
Der Text besitzt hinter bestimmten Ergebnissen einen Ergebnisindex in Form von (E<Zahl>).
Du sollst die Ergebnisse dem passenden Stichpunkt zuordnen und hinter dem Stichpunkt anfügen.
Es treffen mehrere Ergebnisse auf einen Stichpunkt zu, aber niemals ein Stichpunkt auf mehrere Ergebnisse.
Wähle den wahrscheinlichsten Stichpunkt.




Der Benutzer muss inkorrekte Daten identifizieren und korrigieren können. (E1, E2)
Der Benutzer muss die an ihn eingereichten Formulare korrekt übertragen können. (E5, E6)
Der Benutzer muss unzulässige Bewerbungen identifizieren können. (E9, E10)
Der Benutzer muss datenschutzkonform in die Anwendung eintragen können und über mögliche Verstöße informiert werden. (E11)
Der Benutzer muss erkennen können, wie er zum korrekten Prozess gelangt. (E12)
Der Benutzer muss bezüglich Aufnahmeentscheidungen mit den Entscheidungsträgern zusammen arbeiten können. (E7, E8)
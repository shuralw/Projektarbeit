Intuitiv
Die Sekretärin der Förderschule (3) bezeichnet die Umsetzung der Aufgaben als \glqq relativ intuitiv\grqq{}. (E83)

Unvollständige Abfrage der Daten
- Die Sekretärinnen der Grundschule (4) und des Berufskollegs (5) gaben an, dass unvollständige Einträge (E6) und fehlende Unterschriften ebenfalls zu den häufigsten Schwierigkeiten bei der Dateneingabe zählten.
- Die Realschulsekretärin (2) gab an, dass noch weitere, derzeit nicht erfasste Informationen benötigt würden, darunter Angaben zu iPad-Mietverträgen und Kaufoptionen. (E105)  Die Integration von Schulbuchbestellungen und Materiallisten in die versendete E-Mail wäre aus ihrer Sicht ebenso wünschenswert, wie die Erfassung von Noten ausgewählter Fächer. (E107) Des Weiteren merkte sie an, dass eine Geburtsurkunde vorliegen müsse und die Datenschutzerklärung der Schule vom Schüler bzw. Elternteil gelesen und akzeptiert werden sollte. (E108)

Inkorrekte Daten
Sie merkte zudem an, dass die E-Mail-Adresse korrekt sein und überprüft werden müsse. (E106)

Erlernbarkeit 
Die Grundschulsekretärin (4) nannte die notwendige Umstellung und den damit verbundenen Zeitaufwand als Hürden. Eine schnelle Erlernbarkeit der Anwendung würde diesen Aufwand aus ihrer Sicht jedoch reduzieren. (E110)

Navigation
- Die Sekretärin der Förderschule (3) äußerte den Bedarf, den Unterschied zwischen den Menüpunkten \textit{Schüler:innen} und \textit{Anmeldungen} klarer zu gestalten. (E85)
- Die Sekretärin des Gymnasiums (1) begegnete keinen besonderen Schwierigkeiten oder Verwirrungen, abgesehen von der bereits angesprochenen Unklarheit hinsichtlich der Menüpunkte, die auch die Sekretärin der Förderschule (3) wahrgenommen hatte. (E86)

Datenfelder sind unverständlich, Schaltflächen sind unverständlich sowie Texte oder Bezeichner werden missinterpretiert
- Alle Studienteilnehmer hatten Schwierigkeiten zu verstehen, wofür das Feld \textit{ID-Schlüssel} gedacht sei. (E25)
- Für die Sekretärin des Gymnasiums (1) war das Dropdown \textit{Klasse} unklar (E27), da keine Daten vorhanden waren und noch nicht feststand, welche Klassen es geben würde.
- Die Sekretärin der Berufsschule (5) hatte Diskussionen mit einer Kollegin, wie das \textit{Schuljahr} zu interpretieren sei. (E33)
- Die Sekretärin der Realschule (2) hätte den Begriff \textit{Neuaufnahme} in der Auswahl für den Aufnahmestatus erwartet (E29), da diese Terminologie auch bei \textit{SchILD} verwendet wird.
- Der \textit{Erstellen}-Button wurde unterschiedlich interpretiert. Die Sekretärin der Förderschule (3) interpretierte ihn im Sinne von \textit{Speichern}, während die Sekretärin der Hauptschule (4) annahm, dass damit ein weiterer Sorgeberechtigter angelegt werden könne, was sie auf das Plus-Symbol zurückführte. (E47)
- Die Sekretärin nahm das Feld Notfallkontakt-Rolle nicht als ein Feld wahr, das man benutzerdefiniert ausfüllen kann, sondern als ein Feld, in dem man zwingend einen der vorgeschlagenen Werte auswählen muss. (E51)
- Die Sekretärin der Gesamtschule (5) stellte die Definition von \textit{Migrationshintergrund} in Frage. Sie war sich unsicher und spekulierte darüber, ab wann ein Migrationshintergrund vorliege. (E56)
- Sie konnte nicht unterscheiden, ob die letzte Tätigkeit des Schülers oder des Sorgeberechtigten erfragt wird. Die angestrebte Schulstufe wurde von ihr als letzte Schulstufe interpretiert. (E60)
- Sie konnte nicht unterscheiden, ob die letzte Tätigkeit des Schülers oder des Sorgeberechtigten erfragt wird. Die angestrebte Schulstufe wurde von ihr als letzte Schulstufe interpretiert. (E60)
- Die Sekretärin der Realschule (2) äußerte Unsicherheit darüber, wer Zugriff auf die internen Notizen hat. Sie definierte die Felder \textit{Interne} und \textit{Bemerkung für Schüler*in} als \glqq Interne Notizen sind unsere Meinung, Bemerkungen sind Fakten\grqq{}. (E66) Die Sekretärin der Grundschule (4) war sich ebenfalls unsicher, welche Informationen dem Schüler zugänglich sind. Im Gegensatz zu (2) interpretierte (4), dass die \textit{Interne Notiz} nur für die Schule sichtbar ist und das \textit{Bemerkung}-Feld auch vom Kind gesehen werden könne, fühlte sich aber durch den Infotext oben im Formular verwirrt. (E67)
- Sekretärin 1 erlebte eine unklare Fehlermeldung beim Klick auf \textit{Speichern}. Sie wünscht sich klarere Anweisungen zur Fehlerbehebung und einen direkten Fokus auf das fehlerhafte Feld. (E71)
- Sie war sich auch unsicher, ob sie nun eine \glqq Anmeldung\grqq{} oder eine \glqq Anmeldung\grqq{} eingereicht hätte. (E74)
- Sie empfand die Aufgabe zwischendurch als unübersichtlich und hätte sich eine Hilfeoption oder ein Handbuch gewünscht. (E77)
- Die Sekretärin der Realschule (2) fand die Felder im Zusammenhang mit dem Export im Update-Prozess verwirrend. (E87)


Fehlende Hilfetexte
- Sie empfand die Aufgabe zwischendurch als unübersichtlich und hätte sich eine Hilfeoption oder ein Handbuch gewünscht. (E77)
- Die Sekretärin des Berufskollegs (5) sprach den Wunsch aus, dass es neben den Feldern kleine Erklärungen geben sollte, um die Bedienung der Anwendung zu erleichtern. (E92)


Unlogische Sortierung von Werten
- Die Sekretärin der Förderschule (3) hielt die Sortierung der Schuljahre für nicht sinnvoll und stellte fest, dass man mitunter keine medizinischen Informationen erfassen darf. (E31)
- die Sortierung der Ergebnisse sollte ihrer Meinung nach die lokale Umgebung bevorzugen sollte. (E58)

Fehlender Sichtbarkeit des Systemstatusses
(E78)

Datenschutz
Die Sekretärin der Realschule (2) äußerte Unsicherheit darüber, wer Zugriff auf die internen Notizen hat. Sie definierte die Felder \textit{Interne} und \textit{Bemerkung für Schüler*in} als \glqq Interne Notizen sind unsere Meinung, Bemerkungen sind Fakten\grqq{}. (E66) Die Sekretärin der Grundschule (4) war sich ebenfalls unsicher, welche Informationen dem Schüler zugänglich sind. Im Gegensatz zu (2) interpretierte (4), dass die \textit{Interne Notiz} nur für die Schule sichtbar ist und das \textit{Bemerkung}-Feld auch vom Kind gesehen werden könne, fühlte sich aber durch den Infotext oben im Formular verwirrt. (E67)

Dialogabbrüche
- Die Sekretärin war verärgert, als der Sorgerechtsprozess durch einen versehentlichen Klick neben das Popup abgebrochen wurde und forderte, dass es nicht möglich sein sollte, Daten zu verlieren, sei es durch Danebenklicken oder Ausloggen. (E44)

Fehler treten auf
Die Sekretärin der Grundschule (4) nahm an, dass die Anmeldung des Kindes als \textit{vorzeitige Einschulung} ein Fehler sei. 

Fehlende Führung bei Fehlern
Sekretärin 1 erlebte eine unklare Fehlermeldung beim Klick auf \textit{Speichern}. Sie wünscht sich klarere Anweisungen zur Fehlerbehebung und einen direkten Fokus auf das fehlerhafte Feld. (E71)
Sekretärin 1 schätzt die neue Software als besser und übersichtlicher ein als die vorherige. Sie bemängelt jedoch, dass Felder mit Validierungsproblemen erst bei der Speicherung markiert würden und nicht direkt, wenn das Problem auftritt. (E81)
Die Sekretärin der Förderschule (3) bemängelte das Fehlen eines Hinweises, auf welchem Reiter ein Fehler vorliegt. (E91)

Unvollständige Datenabfrage
- Sie bemerkte, dass Informationen zu wichtigen Themen wie dem Nachweis des Masernschutzes oder der Betreuung für beispielsweise die offene Ganztagsschule fehlten. (E75)
- Die Sekretärin der Realschule (2) hofft, dass die Anmeldungsdaten nun in das SchILD-Programm übertragen werden können. Sie würde auch noch gerne \glqq Standardinformationen\grqq{}, die stets zu einer Anmeldung erfasst werden müssen, einfach eingeben können. Beispiele hierfür sind, ob das Kind im Schuljahr fotografiert werden darf oder ob es ein Schwimmabzeichen hat. (E82) 

Nicht wie andere Programme
Die Realschulsekretärin (2) gab an, dass noch weitere, derzeit nicht erfasste Informationen benötigt würden, darunter Angaben zu iPad-Mietverträgen und Kaufoptionen. (E105) Sie merkte zudem an, dass die E-Mail-Adresse korrekt sein und überprüft werden müsse. (E106) Die Integration von Schulbuchbestellungen und Materiallisten in die versendete E-Mail wäre aus ihrer Sicht ebenso wünschenswert, wie die Erfassung von Noten ausgewählter Fächer. (E107) Des Weiteren merkte sie an, dass eine Geburtsurkunde vorliegen müsse und die Datenschutzerklärung der Schule vom Schüler bzw. Elternteil gelesen und akzeptiert werden sollte. (E108)
Die Sekretärin des Gymnasiums (1) schlug vor, die Erfassung der Anmeldung analog zu den Dokumenten der Schule oder vergleichbaren Software (wie \textit{SchILD}) zu gestalten, um eine effizientere Datenübertragung zu ermöglichen. (E97) 


Überladen
Ihrer Meinung nach seien zu viele Informationen im Panel \textit{Bildungsgang} enthalten, insbesondere das Kürzel des Bildungsgangs, das für sie schwer zu verstehen war. (E76)
- Die Sekretärin der Grundschule (4) fand die Umsetzung der Aufgaben \glqq nicht schwierig\grqq{}  und \glqq überschaubar\grqq{} , obwohl es bei der \textit{letzten Tätigkeit} zu Irritationen kam. (E84)

Unnattraktiv
- Die Sekretärin der Berufsschule (5) empfand das Erfassungsformular für die Sorgeberechtigten im Vergleich zu den anderen Formularen als \glqq unattraktiv\grqq{} und \glqq unübersichtlich\grqq{}. (E48)

nicht sichtbare Felder
- Sie konnte zeitweise nicht erkennen, wie sie den Datensatz speichern kann, da das Autocomplete-Dropdown-Feld den Speichern-Button versteckte. (E50)
- Sie hatte Schwierigkeiten mit der Suche nach der letzten Schule, da die Suche eine Checkbox überdeckte (E57) 
- Die Sekretärin des Berufskollegs (5) war überrascht, dass dieser Tab erschien, da sie zuvor nicht bemerkt hatte, wie weit sie im Prozess fortgeschritten war. Dieser Tab wurde zuvor aufgrund des Overflow-Verhaltens des Browsers verdeckt. (E69)
 

Unlogische Datenabfragen
Die Sekretärin der Förderschule (3) hielt den Tab \textit{Letzte Tätigkeit} für überflüssig, da ein Kind, das sich für die Primarstufe anmeldet, noch keinen Schulabschluss haben kann. (E59) 
Sekretärin 1 gab an, dass sie den Tab \textit{Bemerkungen} an einer früheren Stelle im Prozess erwartet hätte. (E70)



Widersprüche 
(E61) Sie wies auch darauf hin, dass sich das \textit{Angestrebtestufe} und mit dem Tab \textit{Letzte Tätigkeit} einander zu widersprechen scheinen. (E62)
Die Sekretärin des Berufskollegs (5) äußerte den Wunsch, die Bezeichnungen innerhalb der Software anzupassen. Ihrer Meinung nach sollte die Bezeichnung \textit{Letzte Tätigkeit} durch \textit{Bisherige Schullaufbahn} ersetzt werden. (E89)

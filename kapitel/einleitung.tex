\section{lose Themen}
\subsection{Zielsetzung}
Das Ziel der Forschung ist es, einen Einblick zu gewinnen, welche Problemstellen und Nutzerbedürfnisse das Schulpersonal beim Bewerbungsprozess eines Schülers existieren. Hierzu wird ein semistrukturiertes Leitfadeninterview durchgeführt und die Testpersonen offen beobachtet. Die Ergebnisse werden anschließend qualitativ ausgewertet. Die Forschung wird betrieben, indem Personen aus der genannten Gruppe die Anwendung Schüler Online die Aufgaben "Bewerbung an der Primarstufe erstellen", "Bewerbung an der Sekundarfstufe I erstellen" sowie "Bewerbung an der Sekundarfstufe II erstellen" vollziehen. 


\title{Neugestaltung von Schüler Online: Eine Beobachtungs- und Interviewstudie zur Identifikation von Problemstellen und Nutzerbedürfnissen, um die Effektivität sowie die Zufriedenstellung des Schulpersonals beim Erfüllen von Kernaufgaben der Webanwendung zu optimieren}

\section{Einleitung}
Im Rahmen dieser Studie sollen die folgenden Leitfragen beantwortete werden: 
\begin{itemize}
    \item (LF1) Wovon handelt Schüler Online, welche Kernaufgaben bildet es ab und welche werden in dieser Studie analysiert? % (Theorie)
    \item (LF2) Wie kann Effektivität im Kontext der Untersuchung definiert werden? % (Theorie)
    \item (LF3) Wie kann Zufriedenheit im Kontext der Untersuchung definiert werden? % (Theorie)
    \item (LF4) Welche typische Problemstellen und Nutzerbedürfnisse gibt es bei modernen Webanwendungen? % (Theorie)
    \item (LF5) Wie sollte so ein Fragebogen aussehen? % (Ergebnisteil)
    \item (LF6) Wie sieht der Nutzungskontext aus? % (Ergebnisteil)
    \item (LF7) Welcher Nutzerbedürfnisse gibt es und inwieweit werden sie von der Anwendung erfüllt? % (Ergebnisteil)
    \item (LF8) Welche Problemstellen gab es bei der Anwendung? % (Ergebnisteil)
    \item (LF9) Erreichen die Nutzer effektiv und zufriedenstellend ihre Ziele? % (Diskussion)
\end{itemize}



%Zu Beginn der Ausarbeitung ist es besonders wichtig, dem Leser zu erklären, warum Sie das Thema bearbeiten.
%Dabei ist nicht unbedingt Ihre eigene Motivation gemeint, sondern vielmehr die Frage, warum die Problemstellung Ihrer Arbeit relevant ist.
%Angemessene Motivationsgründe sind etwa:
%\begin{compactitem}
%\item Sie lösen ein Problem, für das es bisher keine, oder keine gute Lösung gibt.
%\item Ein Unternehmen kann Ihre Lösung einsetzen, um damit Profit zu erwirtschaften.
%\item Sie vergleichen verschiedene Produkte oder Methoden, um damit die Entscheidungsfindung bei der Auswahl zu erleichtern.
%\item Sie stellen ein komplexes Thema für eine bestimmte Zielgruppe angemessen dar.
%\end{compactitem}
Die zugrunde liegende Problemstellung ist relevant für den Hersteller der Anwendung "Schüler Online" ("Kommunales Rechenzentrum Minden/Ravensberg-Lippe"), da Unklarheit herrscht, ob die Anwender die Software korrekt bedienen können. Die Korrektheit ist auch für die Anwender wichtig, da die Anwendung das Schulgesetz abbilden soll und eine korrekten Erfüllung ermöglichen soll. Der Nutzer der Anwendung soll zufrieden sein, seine Erwartungen und Bedürfnisse an die Software sollen erfüllt werden. Mit der vorliegenden Studie soll geprüft werden, inwieweit die Software ebenjenen entspricht oder abweicht.

Es gibt zwar bereits Forschungen hinsichtlich häufig vorkommenden Problemstellen und Nutzerbedürfnissen von diversen Autoren, 
%wie bspw. 
allerdings liegt noch keine Forschung zur Webanwendung "Schüler Online" im Bereich des Usability Engineerings vor. 

Es soll nicht die Effizienz untersucht werden.
Konkrete Maßnahmen oder Handlungsanweisungen werden nicht in dieser Studie behandelt.
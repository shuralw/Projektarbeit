\section{Einleitung}
In Deutschland muss jedes Kind zur Schule gehen. Dies erfordert, dass man sich an Schulen anmeldet. Es besteht regulär die Möglichkeit, dies schriftlich zu erledigen. Im Zuge der Digitalisierung sollten Behördengänge wie Anmeldungen zur Schule auch online ausfüllbar gemacht werden. An diesen Punkt knüpft die Anwendung \textit{Schüler Online} an.

\textit{Schüler Online} im Allgemeinen bildet die Online-Services Schulaufnahme und -Wechsel ab. Mithilfe des Portals sollen sich Schüler und Eltern online bei Schulen anmelden können. Die Software wird derzeit vom kommunalen Rechenzentrum Minden-Ravensberg/Lippe entwickelt. Die Software bildet Teile des Schulgesetzes ab, es ist deshalb essentiell, dass der Nutzer die Inhalte der Anwendung versteht. 

Die Relevanz des Themas ergibt sich ebenfalls aus dem Bedarf des Entwicklungsteams, die Verwendungsweise der Anwendung in der Praxis zu verstehen. Trotz vorhandener Forschungen in der Literatur über generelle Probleme im Hinblick auf Effektivität und Zufriedenheit sowie über diverse Heuristiken, fehlen spezifische Erkenntnisse für die zu untersuchende Webanwendung \textit{Schüler Online}. Die Erforschung dieser Aspekte zielt darauf ab, das System zu optimieren, sodass Schulpersonal Bewerbungen von Schülern effektiv und zufriedenstellend bearbeiten kann. Aspekte der Effizienz sind nicht Gegenstand dieser Studie.

Das Forschungsziel ist es, Einblicke in die Problemstellen zu gewinnen, die das Schulpersonal während des Bewerbungsprozesses eines Schülers hindern könnten, seine Ziele effektiv zu erreichen. Dabei soll untersucht werden, ob die Testpersonen die an sie gestellten Aufgaben mit der Anwendung \textit{Schüler Online}  erfolgreich bewältigen können. 

Die Forschungsmethode umfasst fünf qualitative Auswertungen der Interviews und Beobachtungen, wobei die Testpersonen drei konkrete Aufgaben bezüglich der Schulbewerbung innerhalb der Anwendung bewältigen sollen. Die Arbeit wird überwiegend induktiv betrieben und versucht, spezifische Erkenntnisse für das zu untersuchende System zu generieren, um es letztendlich zu verbessern.

Es sei jedoch darauf hingewiesen, dass diese Studie keine konkreten Maßnahmen oder Handlungsanweisungen behandelt. Vielmehr konzentriert sie sich auf die Identifizierung möglicher Problemstellen und Nutzerbedürfnisse, um einen fundierten Ausgangspunkt für zukünftige Forschungen und Entwicklungen zu bieten.

Im Rahmen dieser Studie sollen die folgenden Leitfragen beantwortete werden: 
\begin{itemize}
    \item Wovon handelt \textit{Schüler Online}, welche Kernaufgaben bildet es ab und welche werden in dieser Studie analysiert? (LF1)% (Theorie)
    \item Wie kann Effektivität im Kontext der Untersuchung definiert werden? (LF2)% (Theorie)
    \item Wie kann Zufriedenheit im Kontext der Untersuchung definiert werden? (LF3)% (Theorie)
    \item Wie sollte ein Fragebogen zur Bestimmung des Effektivitäts- und Zufriedenstellungsgrads aussehen? (LF4)% (Ergebnisteil)
    % \item Welcher Nutzerbedürfnisse gibt es und inwieweit werden sie von der Anwendung erfüllt? (LF7)% (Ergebnisteil)
    \item Welche Problemstellen gibt es bei in Anwendung? (LF5)% (Ergebnisteil)
    \item Erreichen die Nutzer effektiv und zufriedenstellend ihre Ziele? (LF6)% (Diskussion)
\end{itemize}

Die Leitfragen 1-3 werden zunächst im Theorieteil behandelt, welcher die Begrifflichkeiten definiert und vergleichbare Literatur heranzieht. Anschließend wird im Kapitel Material und Methode dargelegt, wie in der Untersuchung vorgegangen wurde. Im Ergebnisteil der Studie werden durch die kurze Skizzierung des Fragebogens und der Zusammenfassung der Erkenntnisse der Studie Leitfragen 4 und 5 beantwortet. Im Diskussionsteil werden die Ergebnisse interpretiert und die Leitfragen gebündelt beantwort. Durch diese Vorgehensweise soll im Diskussionsteil geklärt werden, ob der Anwender die Software effektiv und zufriedenstellend benutzen kann (Leitfrage 6).

%Das Thema dieser Projektarbeit behandelt die Webanwendung textit{Schüler Online 2.0}. Hierzu %werden bei insgesamt fünf Sekretärinnen an fünf verschiedenen Schulen ein semistrukturiertes %Leitfadeninterview durchgeführt und die Testpersonen offen beobachtet, und im Rahmen dieser %Projektarbeit ausgewertet.
%
%Das Thema ist relevant, da dem Entwicklungsteam bislang keine Klarheit darüber vorliegt, wie %die Anwendung in der Praxis performt.
%Die Forschung wird überwiegend induktiv betrieben, es gibt zwar in der Literatur bereits %Erkenntnisse darüber, welche Probleme hinsichtlich Effektivät und Zufriedenstellung sowie %diverse Heuristiken, nicht allerdings speziell für die zu untersuchende Webanwendung. Ebendas %soll untersucht werden, damit die Anwender die Bewerbungen von Schülern effektiv und %zufriedenstellend erfassen und bearbeiten können.
%
%Das Ziel der Forschung ist es, einen Einblick zu gewinnen, welche Problemstellen und %Nutzerbedürfnisse das Schulpersonal beim Bewerbungsprozess eines Schülers und den Nutzer daran %hindern, sein Ziel effektiv und zufriedenstellend zu erreichen. Es soll herausgefunden werden, %ob die Testpersonen die Aufgabenstellungen effektiv und zufriedenstellend erledigen können. Die %Ergebnisse der Interviews und Beobachtungen werden hierfür qualitativ ausgewertet. Die %Forschung wird betrieben, indem Personen aus der genannten Gruppe drei Aufgaben in der %Anwendung\textit{Schüler Online} bewältigen sollen. 




%Zu Beginn der Ausarbeitung ist es besonders wichtig, dem Leser zu erklären, warum Sie das Thema bearbeiten.
%Dabei ist nicht unbedingt Ihre eigene Motivation gemeint, sondern vielmehr die Frage, warum die Problemstellung Ihrer Arbeit relevant ist.
%Angemessene Motivationsgründe sind etwa:
%\begin{compactitem}
%\item Sie lösen ein Problem, für das es bisher keine, oder keine gute Lösung gibt.
%\item Ein Unternehmen kann Ihre Lösung einsetzen, um damit Profit zu erwirtschaften.
%\item Sie vergleichen verschiedene Produkte oder Methoden, um damit die Entscheidungsfindung bei der Auswahl zu erleichtern.
%\item Sie stellen ein komplexes Thema für eine bestimmte Zielgruppe angemessen dar.
%\end{compactitem}
%Es gibt zwar bereits Forschungen hinsichtlich häufig vorkommenden Problemstellen und Nutzerbedürfnissen von diversen Autoren, 
%wie bspw. 
%allerdings liegt noch keine Forschung zur Webanwendung \textit{Schüler Online}  im Bereich des Usability Engineerings vor. 

\title{Neugestaltung von Schüler Online: Eine Beobachtungs- und Interviewstudie zur Identifikation von Problemstellen und Nutzerbedürfnissen, um die Effektivität sowie die Zufriedenstellung des Schulpersonals beim Erfüllen von Kernaufgaben der Webanwendung zu optimieren}

\section{Einleitung}

Das zentrale Anliegen dieser Projektarbeit ist die Untersuchung der Webanwendung "Schüler Online 2.0". Um ein tieferes Verständnis für ihre Funktionalität in der Praxis zu erlangen, werden fünf Sekretärinnen an fünf verschiedenen Schulen mittels eines semistrukturierten Leitfadeninterviews befragt und offen beobachtet. Die Ergebnisse werden im Rahmen dieser Arbeit ausgewertet und analysiert.

Die Relevanz des Themas ergibt sich aus dem Bedarf des Entwicklungsteams, die Verwendungsweise der Anwendung in der Praxis zu verstehen. Trotz vorhandener Forschungen und Literatur über generelle Probleme im Hinblick auf Effektivität und Zufriedenheit sowie über diverse Heuristiken, fehlen spezifische Erkenntnisse für die untersuchte Webanwendung. Die Erforschung dieser Aspekte zielt darauf ab, das System zu optimieren, sodass Schulpersonal Bewerbungen von Schülern effektiv und zufriedenstellend bearbeiten kann. Aspekte der Effizienz sind nicht Gegenstand dieser Studie.

Das Forschungsziel ist es, Einblicke in die Problemstellen zu gewinnen, die das Schulpersonal während des Bewerbungsprozesses eines Schülers hindern könnten, seine Ziele effektiv zu erreichen. Dabei soll untersucht werden, ob die Testpersonen die an sie gestellten Aufgaben mit der Anwendung "Schüler Online 2.0" erfolgreich bewältigen können. 

Die Forschungsmethode umfasst qualitative Auswertungen der Interviews und Beobachtungen, wobei die Testpersonen drei konkrete Aufgaben innerhalb der Anwendung bewältigen sollen. Die Arbeit wird überwiegend induktiv betrieben und versucht, spezifische Erkenntnisse für das untersuchte System zu generieren, um es letztendlich zu verbessern.

Es sei jedoch darauf hingewiesen, dass diese Studie keine konkreten Maßnahmen oder Handlungsanweisungen behandelt. Vielmehr konzentriert sie sich auf die Identifizierung möglicher Problemstellen und Nutzerbedürfnisse, um einen fundierten Ausgangspunkt für zukünftige Forschungen und Entwicklungen zu bieten.

Im Rahmen dieser Studie sollen die folgenden Leitfragen beantwortete werden: 
\begin{itemize}
    \item Wovon handelt Schüler Online, welche Kernaufgaben bildet es ab und welche werden in dieser Studie analysiert? (LF1)% (Theorie)
    \item Wie kann Effektivität im Kontext der Untersuchung definiert werden? (LF2)% (Theorie)
    \item Wie kann Zufriedenheit im Kontext der Untersuchung definiert werden? (LF3)% (Theorie)
    \item Welche typische Problemstellen und Nutzerbedürfnisse gibt es bei Webanwendungen? (LF4)% (Theorie)
    \item Wie sollte so ein Fragebogen aussehen? (LF5)% (Ergebnisteil)
    % \item Welcher Nutzerbedürfnisse gibt es und inwieweit werden sie von der Anwendung erfüllt? (LF7)% (Ergebnisteil)
    \item Welche Problemstellen gibt es bei in Anwendung? (LF6)% (Ergebnisteil)
    \item Erreichen die Nutzer effektiv und zufriedenstellend ihre Ziele? (LF7)% (Diskussion)
\end{itemize}

ALTERNATIV  

Im Rahmen dieser Studie werden mehrere zentrale Fragen untersucht: Zunächst wird die Webanwendung \glqq Schüler Online\glqq  beleuchtet, um zu klären, wovon sie handelt, welche Kernaufgaben sie abbildet, und welche speziell in dieser Studie analysiert werden (LF1). Anschließend wird die Effektivität im Kontext der Untersuchung definiert(LF2), sowie die Zufriedenheit im Kontext der Untersuchung festgelegt(LF3). Die Forschung zitiert typische Problemstellen und Nutzerbedürfnisse bei Webanwendungen(LF4). Zudem wird erörtert, wie ein entsprechender Fragebogen aussehen sollte, um die benötigten Daten zu sammeln (LF5). Weiterhin wird untersucht, welche Problemstellen bei der Anwendung konkret aufgetreten sind (LF6), und abschließend bewertet, ob die Nutzer effektiv und zufriedenstellend ihre Ziele erreichen (LF7).

\pagebreak
%Das Thema dieser Projektarbeit behandelt die Webanwendung textit{Schüler Online 2.0}. Hierzu %werden bei insgesamt fünf Sekretärinnen an fünf verschiedenen Schulen ein semistrukturiertes %Leitfadeninterview durchgeführt und die Testpersonen offen beobachtet, und im Rahmen dieser %Projektarbeit ausgewertet.
%
%Das Thema ist relevant, da dem Entwicklungsteam bislang keine Klarheit darüber vorliegt, wie %die Anwendung in der Praxis performt.
%Die Forschung wird überwiegend induktiv betrieben, es gibt zwar in der Literatur bereits %Erkenntnisse darüber, welche Probleme hinsichtlich Effektivät und Zufriedenstellung sowie %diverse Heuristiken, nicht allerdings speziell für die zu untersuchende Webanwendung. Ebendas %soll untersucht werden, damit die Anwender die Bewerbungen von Schülern effektiv und %zufriedenstellend erfassen und bearbeiten können.
%
%Das Ziel der Forschung ist es, einen Einblick zu gewinnen, welche Problemstellen und %Nutzerbedürfnisse das Schulpersonal beim Bewerbungsprozess eines Schülers und den Nutzer daran %hindern, sein Ziel effektiv und zufriedenstellend zu erreichen. Es soll herausgefunden werden, %ob die Testpersonen die Aufgabenstellungen effektiv und zufriedenstellend erledigen können. Die %Ergebnisse der Interviews und Beobachtungen werden hierfür qualitativ ausgewertet. Die %Forschung wird betrieben, indem Personen aus der genannten Gruppe drei Aufgaben in der %Anwendung Schüler Online bewältigen sollen. 




%Zu Beginn der Ausarbeitung ist es besonders wichtig, dem Leser zu erklären, warum Sie das Thema bearbeiten.
%Dabei ist nicht unbedingt Ihre eigene Motivation gemeint, sondern vielmehr die Frage, warum die Problemstellung Ihrer Arbeit relevant ist.
%Angemessene Motivationsgründe sind etwa:
%\begin{compactitem}
%\item Sie lösen ein Problem, für das es bisher keine, oder keine gute Lösung gibt.
%\item Ein Unternehmen kann Ihre Lösung einsetzen, um damit Profit zu erwirtschaften.
%\item Sie vergleichen verschiedene Produkte oder Methoden, um damit die Entscheidungsfindung bei der Auswahl zu erleichtern.
%\item Sie stellen ein komplexes Thema für eine bestimmte Zielgruppe angemessen dar.
%\end{compactitem}
Die zugrunde liegende Problemstellung ist relevant für den Hersteller der Anwendung \glqq Schüler Online"(\glqq Kommunales Rechenzentrum Minden/Ravensberg-Lippe"), da Unklarheit herrscht, ob die Anwender die Software korrekt bedienen können. Die Korrektheit ist auch für die Anwender wichtig, da die Anwendung das Schulgesetz abbilden soll und eine korrekten Erfüllung ermöglichen soll. Der Nutzer der Anwendung soll zufrieden sein, seine Erwartungen und Bedürfnisse an die Software sollen erfüllt werden. Mit der vorliegenden Studie soll geprüft werden, inwieweit die Software ebenjenen entspricht oder abweicht.

Es gibt zwar bereits Forschungen hinsichtlich häufig vorkommenden Problemstellen und Nutzerbedürfnissen von diversen Autoren, 
%wie bspw. 
allerdings liegt noch keine Forschung zur Webanwendung \glqq Schüler Online" im Bereich des Usability Engineerings vor. 

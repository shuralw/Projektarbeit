\section{lose Themen}
\subsection{Zielsetzung}
Das Ziel der Forschung ist es, einen Einblick zu gewinnen, welche Problemstellen und Nutzerbedürfnisse das Schulpersonal beim Bewerbungsprozess eines Schülers existieren. Hierzu wird ein semistrukturiertes Leitfadeninterview durchgeführt und die Testpersonen offen beobachtet. Die Ergebnisse werden anschließend qualitativ ausgewertet. Die Forschung wird betrieben, indem Personen aus der genannten Gruppe die Anwendung Schüler Online die Aufgaben "Bewerbung an der Primarstufe erstellen", "Bewerbung an der Sekundarfstufe I erstellen" sowie "Bewerbung an der Sekundarfstufe II erstellen" vollziehen. 

\section{Einleitung}
%In der Einleitung solltest du die Zielsetzung deiner Projektarbeit, die zentralen Fragestellungen, die Vorgehensweise und die Bedeutung deines Themas für die Wissenschaft oder die Praxis erläutern.
%=> Ggf. eine Annahme darlegen: Schüler sitzen dem Schulpersonal gegenüber wenn sie sich bewerben, die Sekretärin füllt die Daten aus die der Schüler ihr nennt.
%Leitfrage: Welche Erwartungen hat der Anwender an die Software?
%Leitfrage: Welche Nutzerbedürfnisse gab es bei den Feldtests?
%Leitfrage: Wie sollte so ein Fragebogen aussehen?
%Leitfrage: Wie sieht der Nutzungskontext bei einer Sekretärin in einer Schule aus?
%Leitfrage: Welcher Nutzerbedürfnisse werden von der Anwendung erfüllt, welche nicht?
%Leitfrage: Welche Problemstellen gab es bei den Feldtests?
%Leitfrage: Welche Gemeinsamkeiten gab es bei den Feldtests?

%Zu Beginn der Ausarbeitung ist es besonders wichtig, dem Leser zu erklären, warum Sie das Thema bearbeiten.
%Dabei ist nicht unbedingt Ihre eigene Motivation gemeint, sondern vielmehr die Frage, warum die Problemstellung Ihrer Arbeit relevant ist.
%Angemessene Motivationsgründe sind etwa:
%\begin{compactitem}
%\item Sie lösen ein Problem, für das es bisher keine, oder keine gute Lösung gibt.
%\item Ein Unternehmen kann Ihre Lösung einsetzen, um damit Profit zu erwirtschaften.
%\item Sie vergleichen verschiedene Produkte oder Methoden, um damit die Entscheidungsfindung bei der Auswahl zu erleichtern.
%\item Sie stellen ein komplexes Thema für eine bestimmte Zielgruppe angemessen dar.
%\end{compactitem}
Die zugrunde liegende Problemstellung ist relevant für den Hersteller der Anwendung "Schüler Online" ("Kommunales Rechenzentrum Minden/Ravensberg-Lippe"), da Unklarheit herrscht, ob die Anwender die Software korrekt bedienen können. Die Korrektheit ist auch für die Anwender wichtig, da die Anwendung das Schulgesetz abbilden soll und eine korrekten Erfüllung ermöglichen soll. Der Nutzer der Anwendung soll zufrieden sein, seine Erwartungen und Bedürfnisse an die Software sollen erfüllt werden. Mit der vorliegenden Studie soll geprüft werden, inwieweit die Software ebenjenen entspricht oder abweicht.

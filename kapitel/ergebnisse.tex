\section{Ergebnisse}
%In diesem Kapitel solltest du die Ergebnisse deiner Projektarbeit ausführlich und detailliert darstellen. Hierbei solltest du auch das Ergebnis des leitfadengestützten %Interviews einbeziehen.
%
%Expertenmeinungen: Was waren die Meinungen und Einschätzungen der befragten Experten zu deinem Thema?
%Unterschiede und Gemeinsamkeiten: Gab es Unterschiede oder Gemeinsamkeiten in den Antworten der Experten? Gab es übereinstimmende Aussagen oder Diskrepanzen?
%Erkenntnisse: Welche Erkenntnisse konntest du aus den Antworten der Experten gewinnen? Wie tragen die Ergebnisse zu deiner Forschungsfrage und zum aktuellen Stand der Forschung %bei?

%Screenshots des untersuchten Prozesses der Anwendung

\subsection{Identifizierter Nutzungskontext}

% Benutzer:
%     - Kenntnisse: 
%     - Fertigkeiten:     
%     - Erfahrungen:    
%     - Ausbildiungen:     
%     - physische Merkmale:     
%     - Gewohnheiten:   
%     - Vorlieben:    
%     - Fähigkeiten: 
%     - Alter: 
%     - Geschlecht: 
%     - Bildungsniveau: 
%     - Sprachkenntnisse: 
%     - Kultureller Hintergrund: 
%     - Berufliche Erfahrungen: 
%     - Hobbies und Interessen: 
%     - Körperliche Einschränkungen: 
%     - Krankheiten oder Behinderungen: 
% Aufgaben:
%     - Die Art, wie der Benutzer die Aufgabe ausführt
%     - Häufigkeit
%     - Zeitdauer
%     - Priorität der Aufgabe
%     - Komplexität der Aufgabe
%     - Fehleranfälligkeit
%     - Abhängigkeit von anderen Aufgaben
%     - Auswirkungen auf andere Aufgaben

% Ausrüstung:
%     - Hardware
%     - Software
%     - Materialien
%     - Beleuchtung
%     - Möbel
%     - Raumgestaltung
% Verfügbarkeit der Ausrüstung
% Qualität und Zustand der Ausrüstung
% Einfachheit der Bedienung der Ausrüstung
% Kompatibilität der Ausrüstung mit anderen Geräten oder Software
% Umgebung: 
%     - Physischen
%     - sozial 
%     - kulturell: Arbeitsweisen und Einstellungen
%     - organisationsbezogen 
    

% Beschreibe in den qualitativen Ergebnissen für jeden Abschnitt:

% erst allgemeine, dann detaillierte Ergebnisse
% wiederkehrende Muster
% signifikante oder repräsentative individuelle Antworten
% relevante Zitate aus den Daten
% keine Interpretationen oder Spekulationen


Die Studienteilnehmer haben die Aufgabe 2 und 3(\glqq Bearbeiten Sie bitte die Bewerbung von \textit{Lotta Meier} nach eigenem Ermessen\grqq{} sowie \glqq Bearbeiten Sie bitte die Bewerbung von \textit{Konrad Schulz} nach eigenem Ermessen\grqq{}) anhand der Informationslage nicht erfolgreich abgeschlossen, da der Anmeldestatus bei allen Studienteilnehmern auf \glqq angemeldet\grqq{} verblieben ist. Der Anmeldestatus hätte zwingend einen der Status \glqq Aufgenommen\grqq{}, \glqq Warteliste\grqq{},\grqq{}\glqq 

Die Verbindungsgeschwindigkeit variierte gemäß der Angaben der Teilnehmern sowie den Beobachtungen von hinreichend schnell im Sinne von \glqq Jede Seite und jeder Inhalt lädt innerhalb von unter einer Sekunde vollständig\grqq{} bis zu langsam im Sinne von: \glqq Der initiale Seitenaufbau dauert länger als 20 Sekunden\grqq{}. Konsekutive Seitenaufrufe waren aufgrund der Architektur der Software schneller (Single Page Application - schnelles Rendering der Folgeseiten in weniger als einer Sekunde), es kam noch zu geringfügigen Ladezeiten (weniger als drei Sekunden) bei Dropdown Feldern und Listen, wo Daten nachgeladen werden mussten.


Frage 1: In welchem Umfang besitzen Sie Vorerfahrungen mit Schüler Online 1.0? 	

Antworten der Gymnasiums-Sekretärin:
- Bisher gab es zwei mal pro Jahr Schulwechsel zu den örtlichen Berufskollegs. Daten einpflegen läuft gut
- Ein Kollege hatte Praktikum in einem Berufskolleg => Daher Vorerfahrung, aber nicht weiter geübt.

Antworten der Realschul-Sekretärin:
- Keine	

Antworten der Förderschul-Sekretärin:
- Keine

Antworten der Grundschul-Sekretärin:
- Keine


In welchem Umfang besitzen Sie Vorerfahrungen mit der neuen Software?		
Antworten der Gymnasiums-Sekretärin:
- Keine				

Antworten der Realschul-Sekretärin:
- Keine				

Antworten der Förderschul-Sekretärin:
- Schon einmal kurz reingeguckt, auch schon die Elternseite
- Seite war schon offen und angemeldet	
- Nur bund-ID ist möglich, für unsere Eltern wäre das zu schwer. Man müsste das simplifizieren, ein Link zum Einloggen wäre gut ohne Bund-ID
- Leichte Sprache ist sehr wichtig, sogar auf englisch
- Export zu Excel und \textit{SchILD} wäre gut.
- Einstellung \glqq Ich probier mal lieber, als mir was anzuhören\grqq{}
- Aussage: \glqq Die (Datensätze unter Anmeldungen) sind einfach so reingekommen\grqq{} (Ihr ist zu Anfang aufgefallen, dass die Daten zu Aufgabe zwei und drei vorliegen.)			

Antworten der Grundschul-Sekretärin:
- Keine		

Antworten der Berufskolleg-Sekretärin:
- Keine				







Antworten der Gymnasiums-Sekretärin:
Antworten der Realschul-Sekretärin:
Antworten der Förderschul-Sekretärin:
Antworten der Grundschul-Sekretärin:
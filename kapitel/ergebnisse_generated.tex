\section{Ergebnisse}
In diesem Abschnitt der Arbeit werden die Ergebnisse der durchgeführten Untersuchung dargelegt. Die Ergebnispräsentation gliedert sich in vier Teile.

Im ersten Teil werden die Antworten auf die Fragen des Fragebogens und die Beobachtungen, die vor der Durchführung der drei Aufgaben festgestellt wurden zusammengefasst.

Der zweite Teil fokussiert sich auf die Ergebnisse, die unmittelbar vor der Durchführung der drei Aufgaben erworben wurden.

Im dritten Teil werden die Befunde detailliert dargestellt, die während der Bearbeitung der Aufgaben entstanden sind. Dabei wird ein besonderer Schwerpunkt auf die verschiedenen Tabs gelegt, die im Rahmen der Aufgabe 1 bearbeitet wurden, und eine Zusammenfassung der Erkenntnisse aus den Aufgaben 2 und 3 geboten, die die Bearbeitung von existierenden Datensätzen umfassten.

Abschließend konzentriert sich der vierte Teil auf die Resultate aus den Antworten und Beobachtungen, die nach der Bearbeitung der Aufgaben ermittelt wurden.

\subsection{Vor der Durchführung}
\textbf{In welchem Umfang besitzen Sie Vorerfahrungen mit\textit{Schüler Online 1.0}?}\\
Die Sekretärin des Gymnasiums (1) verfügte über praktische und indirekte Erfahrungen mit \textit{Schüler Online}, während die Sekretärinnen der Realschule (2), der Förderschule (3), der Grundschule (4) und des Berufskollegs (5) keine Vorerfahrungen mit der Anwendung hatten.

\textbf{In welchem Umfang besitzen Sie Vorerfahrungen mit der neuen Software?}\\
Die Sekretärin des Gymnasiums (1), der Realschule (2), der Grundschule (4) sowie die des Berufskollegs (5) gaben an, keinerlei vorherige Erfahrung mit dem System zu haben.

Anders äußerte sich die Sekretärin der Förderschule (3). Sie hatte bereits einen ersten Blick auf das System geworfen, die Elternseite eingeschlossen. Sie gab an, dass sie sich bereits auf der Seite angemeldet und eingeloggt hatte.

\textbf{Welche Probleme können bei herkömmlichen, nicht-digitalen Bewerbungen von Schülern auftreten?}\\
Die Sekretärin des Gymnasiums (1) stellte fest, dass sich Eltern beim Einschulungsjahr der Kinder oft verrechnen und dass Schüler gelegentlich versehentlich falsche Daten eingeben, ohne dabei jedoch böswillige Absichten zu haben (E1). 

Demgegenüber äußerte die Sekretärin der Realschule (2) Bedenken hinsichtlich verschiedener Problembereiche: Falsch ausgefüllte oder fehlende Angaben sowie inkorrekte Daten wurden als häufige Fehlerquellen identifiziert (E2). Darüber hinaus wies sie darauf hin, dass bestimmte Fachbegriffe wie \textit{Konfession} oder \textit{Schulformempfehlung} für Schüler möglicherweise nicht verständlich seien (E3). Sie stellte fest, dass einige Schüler absichtlich falsche Angaben machen würden (E4). Dies steht im Gegensatz zur Aussage der Gymnasiums-Sekretärin, die lediglich versehentliche Fehler thematisiert hatte. 

Alle Sekretärinnen unterstrichen die Problematik der Lesbarkeit, da die Formulare in der Regel handschriftlich ausgefüllt werden (E5).

Die Sekretärinnen der Grundschule (4) und des Berufskollegs (5) gaben an, dass unvollständige Einträge (E6) und fehlende Unterschriften ebenfalls zu den häufigsten Schwierigkeiten bei der Dateneingabe zählten.


\textbf{Wer nutzt das System hauptsächlich an Ihrer Schule?}\\
Die Sekretariate aller fünf Schulen (1, 2, 3, 4, 5) würden das System nutzen. Darüber hinaus verwendete bei den Sekretärinnen der Förderschule (3) und des Berufskollegs (5) auch die Verwaltung die Software. Zusätzlich zu Sekretariat und Verwaltung wurden bei der Sekretärin des Berufskollegs (5) die Lehrer einbezogen, die über Aufnahmeentscheidungen bestimmten. E7 

% todo: \textbf{Welche (Teil-)Faktoren bestimmen das Abschließen einer Bewerbung, sodass sie diese nicht mehr bearbeiten müssen?}\\

\textbf{Beschreiben Sie die Ausgangssituation die vorliegt, bevor Sie die Aufgabe \glqq Bewerbung eines Schülers\grqq{} durchführen.}\\
Die Sekretärin der Realschule (2) schilderte, dass die Entscheidung über die Aufnahme eines Kindes auf der Schulempfehlung und vorausgehenden Gesprächen mit der Schulleitung basiere (E8). Mehrfachanmeldungen müssen laut ihr vermieden werden und würden in der Regel bei Kennenlernterminen auffallen (E9). Sie erklärte, dass die Schulen der Region eine spezielle Methode anwendeten, um dies sicherzustellen (mittels farbiger Formulare).

Die Sekretärin der Förderschule (3) gab an, dass das Kind im Schuleinzugsgebiet leben müssen. (E10) Eine Information über die Diagnose Autismus könnte laut ihr oft schon als Entscheidungsgrundlage für die Eignung des Kindes dienen, allerdings könnten in einigen Fällen auch Gutachten erforderlich sein.

\textbf{Welche fachlichen und technischen Qualifikationen sind zur Bewältigung der Aufgabe erforderlich (Aufgabenbewältigung / Softwarenutzung)? Welche Vorkenntnisse fehlen ggf.?}\\
Die Sekretärin des Gymnasiums (1) und der Realschule (2) hoben hervor, dass keine spezifischen fachlichen oder technischen Kenntnisse für die Nutzung von \textit{Schüler Online} erforderlich seien, während die Bereitschaft, sich mit der Software vertraut zu machen, essentiell sei. Die Sekretärin der Förderschule (3) und die der Grundschule (4) unterstrichen die Bedeutung von Datenschutzbewusstsein (E11). Grundlegende Kenntnisse im Schulgesetz und MS Office wurden von der Sekretärin der Grundschule (4) als vorteilhaft, jedoch nicht als notwendig erachtet. Die Sekretärin des Berufskollegs (5) betonte die mögliche Notwendigkeit von Schulungen und das Verständnis für spezifische Bildungsangebote der eigenen Schule.

\subsection{Unmittelbar vor der Durchführung}
Unmittelbar vor der Durchführung der Aufgaben mit der neuen Software zeigten sich bei mehreren Sekretärinnen Schwierigkeiten in der Navigation. Die Sekretärin des Gymnasiums (1) und die der Förderschule (3) navigierten auf die falschen Seiten. Sie konnten den Sinn und Zweck der Menüpunkte \glqq Schüler:innen\grqq{} und \glqq Anmeldungen\grqq{} nicht korrekt differenzieren. Obwohl ihre Aufgabe darin bestand, eine Anmeldung zu erfassen, konnten sie aufgrund der bereitgestellten Informationen nicht richtig navigieren. (E12)

\textbf{Welche Arbeitsschritte sind durchzuführen?}\\
Die Sekretärin des Gymnasiums (1) beschrieb den Prozess in drei Schritten: Zuerst müsse der Schüler im System gesucht werden, dann würden die Daten des Schülers erfasst. Anschließend müsse dem Schüler eine Information übermittelt werden.

% Die Sekretärin der Realschule (2) betonte die Bedeutung der persönlichen Anwesenheit des Schülers an der Schule.
Für die Sekretärin der Förderschule (3) umfasste der Prozess mehrere Schritte: Erst müsse geklärt werden, ob ein Rundgang durchgeführt wurde. Danach würden die Daten aufgenommen und das Gutachten für die Akten angefordert. Ein kurzes Kennenlernen (E13) und die Entscheidung, in welche Klasse das Kind käme, gehörten ebenso zu den Arbeitsschritten. 

Die Sekretärin der Grundschule (4) erläuterte, dass die Eltern zunächst ins Sekretariat kämen, wo ein Gespräch stattfände (E14). Anschließend würden die Daten erfasst und Unterlagen eingereicht. Letztendlich würde eine Schülerakte angelegt. (E15)

\textbf{Welche Hilfsmittel sind erforderlich (für die Aufgabenbewältigung / zur Softwarenutzung)? Welche davon fehlen ggf., welche sind zusätzlich gewünscht?}\\
Für die Sekretärin des Gymnasiums (1) sind Kopier- und Einfügefunktionalitäten wichtig, um Daten aus anderen Programmen wie E-Mail oder PDF-Reader übertragen zu können. (E16)

Die Sekretärin der Förderschule (3) nutzt Google Maps zur Adressrecherche als unterstützendes Hilfsmittel. (E17)

Die Sekretärin des Berufskollegs (5) spricht den Bedarf und einer Kurzanleitung für den Einstieg in die Softwarenutzung an. (E18)

\textbf{Welche Ergebnisse / Teilergebnisse entstehen und wie werden diese ggf. verwertet / weitergeführt?}\\
Die Sekretärin des Berufskollegs (5) stellt heraus, dass nach Abschluss des Prozesses alle notwendigen Daten für die Bewerbung des Schülers erfasst sein sollten. Sie betont insbesondere die Wichtigkeit der Erfassung der vorherigen Schule, da es sehr aufwändig wäre, diese Information nachträglich herauszufinden.(E19) 

\textbf{Welche wichtigen Sonderfälle müssen berücksichtigt werden? (bzw. fallen dem Benutzer spontan ein; z. B. zur Arbeitsteilung / Zusammenarbeit)}\\
Die Sekretärin des Gymnasiums (1) weist darauf hin, dass Wechsel innerhalb des Schuljahres und das Wiederholen von Stufen berücksichtigt werden müssen. (E20) Bei Flüchtlingskindern sind möglicherweise nicht alle erforderlichen Daten vorhanden. (E21) Sie stellt hebt besondere Situationen hervor, wie langfristige Beurlaubungen oder Fälle, in denen Kinder, die etwa Therapien unterziehen und in der Klinik unterrichtet werden, dennoch schulpflichtig angemeldet werden müssen. (E22)

Die Sekretärin der Grundschule (4) betont, dass besondere Fälle wie unterschiedliche Staatsangehörigkeiten und bestimmte Sorgerechtssituationen die Anmeldung verkomplizieren können. (E23)

Die Sekretärin des Berufskollegs (5) identifiziert Schüler mit Förderbedarf als Sonderfälle. Sie merkt auch an, dass manche Schüler in ihrem Alter unsicher sind, ob sie sich als männlich oder weiblich identifizieren. Des Weiteren stellt sie fest, dass bei minderjährigen Schülern eine Unterschrift der Eltern erforderlich ist. Ein weiterer Sonderfall sind Schüler, die sich zu spät bewerben und möglicherweise nicht die notwendigen Voraussetzungen für bestimmte Bildungsgänge erfüllen. (E24)

\subsection{Während der Durchführung}
%todo überleitungstext \textbf{Für jeden Tab / Seite: Welche Seiteninhalte sind unverständlich?}\\

\textbf{\glqq Schüler\grqq{}-Tab}\\  
Alle Studienteilnehmer hatten Schwierigkeiten zu verstehen, wofür das \textit{ID-Schlüssel} gedacht ist. (E25)

Die Sekretärin des Gymnasiums (1) konnte den Zweck der Schülersuche-Funktion nicht nachvollziehen. (E26)


\textbf{Anmerkungen zum Tab \glqq Bildungsgang\grqq{}}\\
Für die Sekretärin des Gymnasiums (1) war das Dropdown-\textit{Klasse} unklar, da keine Daten vorhanden waren und noch nicht feststand, welche Klassen es geben würde. Sie wählte den Status \glqq Warteliste\grqq{}, um die Bewerbung als textit{in Bearbeitung} zu markieren und konnte nicht nachvollziehen, warum in der Anwendung vermerkt werden soll, dass eine Aufnahmeberatung stattgefunden hat.

Die Sekretärin der Realschule (2) hätte den Begriff textit{Neuaufnahme} in der Auswahl für den Aufnahmestatus erwartet, da diese Terminologie auch bei \textit{SchILD} verwendet wird. Sie hatte auch Schwierigkeiten, das \textit{Beschulungsbeginn}-Datum korrekt einzugeben.

Die Sekretärin der Förderschule (3) hielt die Sortierung der Schuljahre für nicht sinnvoll und stellte fest, dass man mitunter keine medizinischen Informationen erfassen darf.

Die Sekretärin der Grundschule (4) war verwirrt durch die Meldung, dass ein \textit{Antrag auf vorzeitige Einschulung} stattfindet und fand die Bildungsgangbezeichnungen verwirrend.

Die Sekretärin der Berufsschule (5) hatte Diskussionen mit einer Kollegin, wie das \textit{Schuljahr} zu interpretieren ist. Sie hatte eine andere Meinung als die Kollegin, dass es sich dabei um das aktuelle Schuljahr handelt und nicht um das Schuljahr, das der Schüler zuvor besucht hat. Darüber hinaus war der Anmeldestatus für sie nicht klar interpretierbar.

\textbf{Anmerkungen zum Tab \glqq Persönliche Daten\grqq{}}\\
Die Sekretärin des Gymnasiums (1) trug die Hausnummer zunächst intuitiv zusammen mit der Straße ins Feld \textit{Straße} ein, anstatt ins \textit{Hausnummer}. Sie gab auch die E-Mail-Adresse des Sorgeberechtigten in das \textit{E-Mail}-Feld des Schülers ein. Sie merkte an, dass sie es bevorzugen würde, wenn der Tab \textit{Persönliche Daten} vor dem Tab \textit{Bildungsgang} kommen würde, so wie es bei \textit{SchILD}  der Fall ist. Sie schlug auch vor, dass der Eintrag für \textit{Staat} standardmäßig mit \textit{Deutschland} vorbelegt sein sollte. Sie wünschte sich außerdem einen Hinweis, dass bei Sek1-Schülern grundsätzlich die Telefonnummer der Sorgeberechtigten und bei Sek2-Schülern die des Schülers erwartet wird.

% Die Sekretärin der Realschule (2) fand die Erfassung von Ortsteilen interessant.
Die Sekretärin der Förderschule (3) überprüft unbekannte Straßen mithilfe von Google Maps auf ihre Existenz. Sie schlug vor, dass es sinnvoll wäre, wenn Eltern angeben könnten, zu welchen Zeiten sie telefonisch erreichbar sind. Sie fände es auch gut, wenn mehrere Telefonnummern für Notfälle erfasst werden könnten.

Die Sekretärin der Berufsschule (5) wünschte sich eine Anzeige, die darauf hinweist, ob der Schüler volljährig ist. Sie hatte Schwierigkeiten mit der Suche nach der Postleitzahl-Ort-Kombination.

\textbf{Anmerkungen zum Tab \glqq Sorgeberechtigte\grqq{}}\\
Die Sekretärin des Gymnasiums (1) hielt es für wichtig, einen Nachweis dafür zu haben, ob eine Person das alleinige Sorgerecht hat oder ob beide Unterschriften vorhanden sind, um zu verhindern, dass sich eine Person über die andere hinwegsetzt. Sie hätte erwartet, dass es eine Funktion zur Datenübernahme der Adressdaten aus den persönlichen Daten des Schülers gibt. Sie war auch unsicher über den Zweck des Postfachs. Sie war verärgert, als der Sorgerechtsprozess durch einen versehentlichen Klick neben das Popup abgebrochen wurde, und forderte, dass es nicht möglich sein sollte, Daten zu verlieren, sei es durch Danebenklicken oder Ausloggen.

Die Sekretärin der Realschule (2) trug die Hausnummer auch intuitiv zusammen mit der Straße in das Feld \textit{Straße} ein.

Alle Sekretärinnen außer die Sekretärin des Berufskolleg (5) versuchten, die Straße in das Feld \textit{Adressart} einzugeben. Die Sekretärin der Förderschule (3) erklärte auf Nachfrage, dass sie intuitiv die folgende Eingabereihenfolge erwartet hätte: Staat => Postleitzahl/Ort => Straße.

Der \textit{Erstellen}-Button wurde unterschiedlich interpretiert. Die Sekretärin der Förderschule (3) interpretierte ihn im Sinne von \textit{Speichern}, während die Sekretärin der Hauptschule (4) annahm, dass damit ein weiterer Sorgeberechtigter angelegt werden kann, was sie auf das Plus-Symbol zurückführte.

Die Sekretärin der Berufsschule (5) empfand das Erfassungsformular für die Sorgeberechtigten im Vergleich zu den anderen Formularen als unattraktiv und unübersichtlich.

\textbf{Anmerkungen zum Tab \glqq Notfallkontakte\grqq{}}\\
Die Sekretärin der Realschule (2) schlug vor, dass es hilfreich wäre, die Art der Telefonnummer (im Sinne von dienstlich, privat, mobil) für die Notfallkontakte hinterlegen zu können und mehrere Telefonnummern einer Person zuzuordnen. Sie konnte zeitweise nicht erkennen, wie sie den Datensatz speichern kann, da das Autocomplete-Dropdown-Feld den Speichern-Button versteckte. Sie nahm dieses Feld nicht als ein Feld wahr, das man benutzerdefiniert ausfüllen kann, sondern als ein Feld, in dem man zwingend einen der vorgeschlagenen Werte auswählen muss.

Die Sekretärin der Förderschule (3) versuchte, mehrere Nummern in ein Feld einzugeben, was ihr nicht gelang. Sie wies auch darauf hin, dass es wichtig sei, die Rolle des Notfallkontakts zu erfassen, um im Notfall beurteilen zu können, ob sensible Informationen weitergegeben werden dürfen. Eine Rangfolge der Notfallkontakte hielt sie nicht für sinnvoll. Sie hätte erwartet, die Notfallkontakte früher im Prozess zu sehen.

Die Sekretärin der Hauptschule (4) empfand die Eingabe der Notfallkontakte als unpraktisch, da sie die gleichen Daten wie bei den Sorgeberechtigten eingeben musste. In \textit{SchILD} werden diese Daten automatisch übernommen.

\textbf{Anmerkungen zum Tab \glqq Migrationshintergrund\grqq{}}\\
Die Sekretärin des Gymnasiums (1) äußerte Zufriedenheit mit dem Abschnitt zum Migrationshintergrund und betonte, dass alle wichtigen Informationen vorhanden seien. Sie war jedoch der Meinung, dass das Eingabefeld angezeigt werden sollte, selbst wenn die Option \textit{Migrationshintergrund liegt nicht vor\grqq} ausgewählt ist.

Die Sekretärin der Gesamtschule (5) stellte die Definition von \textit{Migrationshintergrund} in Frage. Sie war sich unsicher und spekulierte darüber, ab wann ein Migrationshintergrund vorliegt.

\textbf{Anmerkungen zum Tab \glqq Letzte Tätigkeit\grqq{}}\\
Die Sekretärin des Gymnasiums (1) kritisierte die Reihenfolge der Bereiche und schlug vor, dass die \textit{Letzte Tätigkeit} innerhalb des Abschnitts \textit{Schüler-Daten} erfasst werden sollte, da es sich hierbei um ergänzende Informationen zur Person handelt. Sie hatte Schwierigkeiten mit der Suche nach der letzten Schule, da die Suche eine Checkbox überdeckte und die Sortierung der Ergebnisse ihrer Meinung nach die lokale Umgebung bevorzugen sollte.

Die Sekretärin der Gemeinschaftsschule (3) hielt den Tab \textit{Letzte Tätigkeit} für überflüssig, da ein Kind, das sich für die Grundschule anmeldet, noch keinen Schulabschluss haben kann. Sie war verwirrt und konnte nicht unterscheiden, ob die letzte Tätigkeit des Schülers oder des Sorgeberechtigten erfragt wird. Die angestrebte Schulstufe wurde von ihr als letzte Schulstufe interpretiert.

Die Sekretärin der Gesamtschule (5) hatte Schwierigkeiten, eine einmal ausgewählte Grundschulempfehlung wieder zu entfernen. Sie wies auch darauf hin, dass sich das \textit{Angestrebtestufe} und mit dem Tab \textit{Letzte Tätigkeit} einander zu widersprechen scheinen.

\textbf{Anmerkungen zum Tab \glqq Qualifikationen\grqq{}}\\
Die Sekretärin der Gesamtschule (5) stellte fest, dass in der Liste der Qualifikationen einige Schulabschlüsse, wie beispielsweise der Hauptschulabschluss, fehlten. Sie äußerte Kritik und regte eine Überarbeitung der Auswahlmöglichkeiten an.

\textbf{Anmerkungen zum Tab \glqq Termine\grqq{}}\\
Die Sekretärin des Berufskollegs (5) konnte den Zweck der Seite \textit{Termine} nicht erkennen. Sie gab an, dass sie ohnehin keine Zeit hätte, 500 Personen zu Terminen einzuladen. Sie merkte jedoch an, dass einige Bildungsgangleitungen vorab Gespräche mit den Schülern führen möchten.

\textbf{Anmerkungen zum Tab \glqq Bemerkungen\grqq{}}\\
Die Sekretärin der Realschule (2) äußerte Unsicherheit darüber, wer Zugriff auf die internen Notizen hat. Sie definierte die Felder \textit{Interne} und \textit{Bemerkung für Schüler*in} als \glqq Interne Notizen sind unsere Meinung, Bemerkungen sind Fakten\grqq{}. Die Sekretärin der Grundschule (4) war sich ebenfalls unsicher, welche Informationen dem Schüler zugänglich sind. Im Gegensatz zu 2 interpretierte 4, dass die \textit{Interne Notiz} nur für die Schule sichtbar ist und das \textit{Bemerkung}-Feld auch vom Kind gesehen werden kann, fühlte sich aber durch den Infotext oben im Formular verwirrt.

Die Sekretärin der Förderschule (3) machte darauf aufmerksam, dass sensible Informationen, wie zum Beispiel Behinderungen, möglicherweise nicht erfasst werden dürfen. Sie betonte die Notwendigkeit einer Benutzerführung, die das Bewusstsein für gesetzliche Anforderungen stärkt.

Die Sekretärin des Berufskollegs (5) war überrascht, dass dieser Tab erschien, da sie zuvor nicht bemerkt hatte, wie weit sie im Prozess fortgeschritten war.Dieser Tab wurde zuvor aufgrund des Overflow-Verhaltens des Browsers verdeckt.

Sekretärin 1 gab an, dass sie den Tab \textit{Bemerkungen} an einer früheren Stelle im Prozess erwartet hätte.

\textbf{Anmerkungen zum Tab \glqq Zusammenfassung\grqq{}}\\
Sekretärin 1 erlebte eine unklare Fehlermeldung beim Klick auf \textit{Speichern}. Sie wünscht sich klarere Anweisungen zur Fehlerbehebung und einen direkten Fokus auf das fehlerhafte Feld.
Die Sekretärin der Realschule (2) stellte eine Verzögerung beim Laden der Inhalte fest. Sie befürchtet, dass die per E-Mail gesendete Bestätigungsmeldung mit der Formulierung \textit{Angemeldet} falsche Erwartungen bei den Schülern wecken könnte. Sie war sich auch unsicher, ob das Schuljahr korrekt erfasst wurde und merkte an, dass noch eine Unterschrift von den Eltern für die Anmeldung erforderlich ist.
Die Sekretärin der Förderschule (3) interpretierte die Meldung auf der Zusammenfassungsseite als \glqq Die Mutter wurde kontaktiert\grqq{}.
Die Sekretärin der Grundschule (4) nahm an, dass die Anmeldung des Kindes als \textit{vorzeitige Einschulung} gekennzeichnet wurde, was sie als Fehler ansah. Sie war sich auch unsicher, ob sie nun eine \glqq Anmeldung Bewerbung\grqq{} eingereicht hat. Sie bemerkte, dass Informationen zu wichtigen Themen wie dem Nachweis des Masernschutzes oder der Betreuung für beispielsweise die offene Ganztagsschule fehlten.
Die Sekretärin des Berufskollegs (5) kritisierte den Informationsgehalt der Seite. Ihrer Meinung nach sind zu viele ,0
Informationen im Panel \textit{Bildungsgang} enthalten, insbesondere das Kürzel des Bildungsgangs, das für sie schwer zu verstehen war.

\subsection{Nach der Durchführung:}

\textbf{Konnten Sie die Aufgabe aus Ihrer Sicht erfolgreich und vollständig abschließen? Falls nein - was hat Sie daran gehindert?}\\
Sekretärin 1 gab an, dass sie die Aufgabe erst beim zweiten Anlauf erfolgreich abschließen konnte. Sie empfand die Aufgabe zwischendurch als unübersichtlich und hätte sich eine Hilfeoption oder ein Handbuch gewünscht. Sie bemerkte auch, dass sie ungern prüfen würde, welche Anmeldungen neu eingegangen sind und wünschte sich Ziffern oder eine andere Hervorhebung, die auf neue, unbearbeitete Anmeldungen hinweist.
Sekretärinnen 2, 3 und 4 konnten die Aufgaben alle erfolgreich speichern und abschließen. 
Die Sekretärin des Berufskollegs (5) konnte die Aufgabe nicht abschließen, da die Aufnahmeentscheidung in der Verantwortung der Abteilungsleiter liegt und sie somit auf diese angewiesen ist. 

\textbf{Wie effektiv unterstützt die Webanwendung Sie bei der Aufgabe?  Gab es positive oder negative Erfahrungen?}\\
Sekretärin 1 schätzt die neue Software als besser und übersichtlicher ein als die vorherige. Sie bemängelt jedoch, dass Felder mit Validierungsproblemen erst bei der Speicherung markiert werden und nicht direkt, wenn das Problem auftritt.

Die Sekretärin der Realschule (2) hofft, dass die Bewerbungsdaten nun in das SchILD-Programm übertragen werden können. Sie würde auch noch gerne \glqq Standardinformationen\grqq{}, die stets zu einer Bewerbung erfasst werden müssen, einfach eingeben können. Beispiele hierfür sind, ob das Kind im Schuljahr fotografiert werden darf oder ob es ein Schwimmabzeichen hat. 

Die Sekretärin der Förderschule (3) bezeichnet die Umsetzung der Aufgaben als \glqq relativ intuitiv\grqq{} .

Die Sekretärin der Grundschule (4) fand die Umsetzung der Aufgaben \glqq nicht schwierig\grqq{}  und \glqq überschaubar\grqq{} , obwohl es bei der letzten Tätigkeit zu Irritationen kam.

\textbf{Haben Sie sämtliche Inhalte der Aufgabe verstanden? Gab es Stellen, an denen Sie sich mehr Unterstützung gewünscht hätten?}\\
Die Sekretärin einer Förderschule (3) äußerte den Bedarf, den Unterschied zwischen den Menüpunkten \textit{Schüler:innen} und \textit{Anmeldungen} klarer zu gestalten. Die Sekretärin einer Grundschule (4) verzeichnete zwar Fragen, gab jedoch keine weiteren Probleme an. Die Sekretärin eines Berufskollegs (5) bestätigte, dass die Aufgabenstellung für sie verständlich war, merkte jedoch an, dass ihre Erfahrung als Sekretärin möglicherweise dazu beigetragen hat.

\textbf{Gab es Schwierigkeiten oder Verwirrungen bei der Aufgabe? Wenn ja, welche?}\\
Die Sekretärin eines Gymnasiums (1) begegnete keinen besonderen Schwierigkeiten oder Verwirrungen, abgesehen von der bereits angesprochenen Unklarheit hinsichtlich der Menüpunkte, die auch die Sekretärin einer Förderschule (3) wahrgenommen hatte.

Die Sekretärin einer Realschule (2) fand die Felder im Zusammenhang mit dem Export im Update-Prozess verwirrend. Außerdem bemängelte sie die Position des Speichern-Buttons bei den Sorgeberechtigten, welcher ihrer Meinung nach oben platziert hätte sein sollen, ähnlich wie die anderen Speichern-Buttons.

Die Sekretärin eines Berufskollegs (5) äußerte den Wunsch, die Bezeichnungen innerhalb der Software anzupassen. Ihrer Meinung nach sollte die Bezeichnung \textit{Letzte Tätigkeit}  durch \textit{Bisherige Schullaufbahn}  ersetzt werden.

\textbf{Wie verständlich waren die Rückmeldungen der Anwendung?}\\
Die Sekretärin eines Gymnasiums (1) äußerte, dass es nützlich wäre, wenn die Anwendung Rückmeldung darüber gibt, ob ein Datensatz vollständig und korrekt ausgefüllt ist. Zudem betrachtete sie die Rückmeldungen als \glqq ausbaufähig\grqq{}  und forderte mehr Feedback mit konkreten Anweisungen.

Die Sekretärin einer Förderschule (3) bemängelte das Fehlen eines Hinweises, auf welchem Reiter ein Fehler vorliegt.

Die Sekretärin eines Berufskollegs (5) sprach den Wunsch aus, dass es neben den Feldern kleine Erklärungen geben sollte, um die Bedienung der Anwendung zu erleichtern.

\textbf{Welche Fähigkeiten setzt die Anwendung Ihrer Einschätzung nach voraus?}\\
Die Sekretärin eines Gymnasiums (1) war der Ansicht, dass die Aufgaben von jedem erledigt werden könnten, da die Begrifflichkeiten auch für Laien verständlich seien.

Die Sekretärin einer Realschule (2) betonte, dass grundlegende Fertigkeiten wie Lesen und Schreiben sowie die Bedienung einer Tastatur und Maus ausreichend seien. Sie fügte hinzu, dass ein Grundverständnis des Schulgesetzes vorteilhaft wäre. Ihrer Meinung nach könnten sogar Praktikanten die Aufgaben erledigen, allerdings sollte eine Überprüfung erfolgen.

Die Sekretärin einer Förderschule (3) sah als Voraussetzung, dass man der deutschen Sprache mächtig sein sollte. Lesefähigkeit allein reiche aus.

Die Sekretärin einer Grundschule (4) betonte, dass Vorkenntnisse in der EDV und eine kaufmännische Ausbildung hilfreich wären. Sie fügte hinzu, dass das weitere Wissen aus dem alltäglichen Arbeitsablauf gewonnen wird.

Die Sekretärin eines Berufskollegs (5) äußerte, dass Lesefähigkeit eine zentrale Voraussetzung darstellt. Darüber hinaus sollten die Nutzer über die eigenen Bildungsgänge der Schule Bescheid wissen und den normalen Schulwerdegang kennen.

\textbf{Wie sehr entspricht die Umsetzung in der Software der Realität?}\\
Die Sekretärin eines Gymnasiums (1) schlug vor, die Erfassung der Bewerbung analog zu den Dokumenten der Schule oder vergleichbaren Software (wie \textit{SchILD}) zu gestalten, um eine effizientere Datenübertragung zu ermöglichen. Sowohl sie (1) als auch die Sekretärin einer Realschule (2) bestätigten, dass die Software inhaltlich \textit{SchILD}  entspricht.

Die Sekretärin einer Realschule (2) bemerkte, dass die Notizen zu den neuen Schülern der fünften Klasse in einer Liste zusammengeführt werden sollten. Sie regte an, dass das Programm der abgebenden Schule mitteilen sollte, dass der Schüler aufgenommen wurde.

Die Sekretärin einer Förderschule (3) bemerkte, dass in der Regel die Eltern den Schüler anmelden und fand es praktisch, wenn die Kommunen die Anmeldung vorbereiten würden.

Die Sekretärinnen einer Grundschule (4) und eines Berufskollegs (5) waren der Meinung, dass das Programm der Realität entspricht. Die Sekretärin eines Berufskollegs (5) hoffte jedoch, dass Anmeldezeiträume berücksichtigt werden würden.

\textbf{Was handhaben Sie in Ihrem Arbeitsalltag bei nicht-digitalen Bewerbungen gewöhnlicherweise anders als in der Anwendung?}\\
Die Sekretärin einer Realschule (2) gab an, dass sie Postleitzahlen und Orte manuell validiere und Kilometerangaben händisch berechne. Sie äußerte den Wunsch nach einer automatischen Berechnung dieser Angaben. Zudem teilte sie mit, dass die Lehrkräfte die Daten der Schülerinnen und Schüler ausgedruckt erhalten möchten, weshalb sie die Möglichkeit zum Ausdruck in der Anwendung vermisste.

Für die Sekretärin einer Förderschule (3) würde die Anwendung lediglich ihre Excel-Tabelle ersetzen.

Die Sekretärin eines Berufskollegs (5) fügte hinzu, dass sie normalerweise noch Dokumente wie Zeugnisse, Lebensläufe und Nachweise über den Masernschutz hinzufüge, um den Lehrkräften eine Entscheidung über die Aufnahme zu ermöglichen.

\textbf{Was würde Sie noch daran hindern die Software in Ihrem Arbeitsalltag einzusetzen?}\\
Die Realschulsekretärin (2) gab an, dass noch weitere, derzeit nicht erfasste Informationen benötigt würden, darunter Angaben zu iPad-Mietverträgen und Kaufoptionen. Sie merkte zudem an, dass die E-Mail-Adresse korrekt sein und überprüft werden müsse. Die Integration von Schulbuchbestellungen und Materiallisten in die versendete E-Mail wäre aus ihrer Sicht ebenso wünschenswert, wie die Erfassung von Noten ausgewählter Fächer. Des Weiteren merkte sie an, dass eine Geburtsurkunde vorliegen müsse und die Datenschutzerklärung der Schule vom Schüler bzw. Elternteil gelesen und akzeptiert werden sollte.

Die Förderschulsekretärin (3) betonte, dass die Anwendung einfacher zu bedienen sein müsse als ihre aktuelle Excel-Tabelle und dass die Eltern sich am Prozess beteiligen müssten.

Die Grundschulsekretärin (4) nannte die notwendige Umstellung und den damit verbundenen Zeitaufwand als Hürden. Eine schnelle Erlernbarkeit der Anwendung würde diesen Aufwand aus ihrer Sicht jedoch reduzieren.
%\textbf{Wie sehr erleichtert Ihnen die Anwendung ihre Arbeit?}\\

\textbf{Gibt es Funktionen, die Sie in ähnlichen bzw. anderen Anwendungen genutzt haben, die Sie hier vermissen?}\\
Die Gymnasiumssekretärin (1) vermisst eine Funktion, die sie aus der \textit{SchILD}-Software kennt - die Filterung von Schulen.

Die Grundschulsekretärin (4) würde sich eine Funktion wünschen, mit der sie Berichte oder ähnliches erstellen kann.
\textbf{Welche Software sollte man aus Ihrer Sicht in Schüler Online integrieren bzw. eine Schnittstelle schaffen?}\\
Alle Sekretärinnen (1,2,3,4,5) sprachen sich einstimmig für eine Integration mit Schulverwaltungsprogrammen wie \textit{SchILD} aus. Des Weiteren betonte die Sekretärin der Realschule (2) die Wichtigkeit einer Verbindung zu ISERV.

Ein weiterer Punkt, der von den Sekretärinnen der Förderschule, der Grundschule und des Berufskollegs (3,4,5) hervorgehoben wurde, ist die Möglichkeit, einen Export nach Excel durchführen zu können.

\textbf{Welche Dokumente würden Sie gerne im Prozess ausdrucken können?}\\
Die Sekretärin des Gymnasiums (1) lobte die Möglichkeit, das Anmeldeformular ausdrucken zu können für Archivierungszwecke. Sie wünschte jedoch zusätzlich die Druckfunktionalität für Formulare wie Schweigepflichtsentbindungen, Einverständniserklärungen für PKW/Bulli Mitfahrten und Fotoerlaubnisse für die Website.

Die Realschulsekretärin (2) stellte die Forderung auf, ein Schülerstammblatt sowie jährliche Notenübersichten drucken zu können für die Schülerakte.

Für die Sekretärin der Förderschule (3) war es wichtig, zahlreiche Dokumente wie Bewerbungsformulare, Formulare für Essensverpflegung, Informationen zum Schülerspezialverkehr, Notfallkontaktbögen und Fotoerlaubnisse ausdrucken zu können.

Die Sekretärin der Grundschule (4) hätte gerne die Option, verschiedene Listen und Formulare auszudrucken. Darunter fielen ein Schülerstammblatt, Klassenlisten, Listen von Buskindern, eine Auflistung von Kindern mit Migrationshintergrund und Förderbedarf, eine Liste sortiert nach Nationalitäten, ein Formular über den Religionsunterricht und einen Gesamtüberblick über die mit dem Bus fahrenden Kinder einschließlich ihrer Bushaltestellen.

Die Sekretärin des Berufskollegs (5) wünschte sich die Druckfähigkeit für Schülerstammdatenblätter und eine Übersicht über die aufgenommenen Schülerinnen und Schüler.

\textbf{Welche Dokumente würden Sie gerne einscannen wollen und beim Datensatz hinterlegen?}\\
Im Gegensatz dazu wünschte sich die Realschulsekretärin (2) die Möglichkeit, alle für die Anmeldung relevanten Dokumente, insbesondere das Anmeldeformular, digitalisieren und speichern zu können.

Die Förderschulsekretärin (3) äußerte ebenfalls den Wunsch, das Anmeldeformular digitalisieren und hinterlegen zu können.

Die Grundschulsekretärin (4) fand es wichtig, digitale Personalakten, Sorgerechtsbescheide und Anmerkungen zu aktuellen Gegebenheiten wie Kuraufenthalte scannen und speichern zu können.

Die Berufskollegsekretärin (5) äußerte den Wunsch, Zeugnisse und Lebensläufe scannen und im System hinterlegen zu können.
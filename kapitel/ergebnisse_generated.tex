\pagebreak
\section{Ergebnisse}

\textbf{In welchem Umfang besitzen Sie Vorerfahrungen mit Schüler Online 1.0?}\\
Im Hinblick auf die ersten Vorerfahrungen mit der Anwendung Schüler Online 1.0 offenbarten sich signifikante Unterschiede zwischen den befragten Sekretärinnen unterschiedlicher Schultypen.

Die Sekretärin des Gymnasiums (1) konnte aufgrund regelmäßiger Datenpflegeaktivitäten, die halbjährlich im Zuge von Schulwechseln zu den örtlichen Berufskollegs stattfanden, bereits praktische Erfahrung mit der Anwendung vorweisen. Zusätzlich berichtete sie von indirekten Erfahrungen, die sie durch einen kollegialen Austausch erwarb, in dessen Rahmen ein Kollege ein Praktikum in einem Berufskolleg absolviert hatte.

Im Gegensatz dazu gaben die Sekretärinnen der Realschule (2), der Förderschule (3) und der Grundschule (4) an, keinerlei Vorerfahrungen mit Schüler Online 1.0 zu haben.


\textbf{In welchem Umfang besitzen Sie Vorerfahrungen mit der neuen Software?}\\
Bezüglich der Vorerfahrungen mit der neuen Software berichteten die Sekretärinnen unterschiedlich. Die Sekretärin des Gymnasiums (1), der Realschule (2), der Grundschule (4) sowie die des Berufskollegs (5) gaben an, keinerlei vorherige Erfahrung mit dem System zu haben.

Anders äußerte sich die Sekretärin der Förderschule (3). Sie hatte bereits einen ersten Blick auf das System geworfen, die Elternseite eingeschlossen. Sie gab an, dass sie sich bereits auf der Seite angemeldet und eingeloggt hatte. Sie wünschte sich eine Exportfunktion zu Excel und \textit{SchILD} und wies darauf hin, dass sie eine explorative Herangehensweise an neue Software bevorzugt(\glqq Ich probier mal lieber, als mir was anzuhören\grqq{}). Darüber hinaus merkte sie an, dass sie die Daten für die Aufgaben zwei und drei bereits vorfand(\glqq Die (Datensätze unter Anmeldungen) sind einfach so reingekommen\grqq{}).

\textbf{Welche Probleme können bei herkömmlichen, nicht-digitalen Bewerbungen von Schülern auftreten?}\\
Hinsichtlich der Dateneingabe durch Eltern und Schüler lieferten die Sekretärinnen verschiedener Schultypen diverse Beobachtungen und Erfahrungen.

Die Sekretärin des Gymnasiums (1) stellte fest, dass sich Eltern beim Einschulungsjahr der Kinder oft verrechnen und dass Schüler gelegentlich versehentlich falsche Daten eingeben, ohne dabei jedoch böswillige Absichten zu haben.

Demgegenüber äußerte die Sekretärin der Realschule (2) Bedenken hinsichtlich verschiedener Problembereiche: Falsch ausgefüllte oder fehlende Angaben, Unleserlichkeit sowie inkorrekte Daten wurden als häufige Fehlerquellen identifiziert. Darüber hinaus wies sie darauf hin, dass bestimmte Fachbegriffe wie \glqq Konfession\grqq{} oder \glqq Schulformempfehlung\grqq{} für Schüler möglicherweise nicht verständlich seien. Sie stellte fest, dass einige Schüler absichtlich falsche Angaben machen würden und verwies auf die Komplexität der Themen Sorgeberechtigung und Datenschutz. Dies steht im Gegensatz zur Aussage der Gymnasiums-Sekretärin, die lediglich versehentliche Fehler thematisiert hatte.

Die Sekretärin der Förderschule (3) unterstrich die Problematik der Lesbarkeit, insbesondere bei handschriftlich notierten E-Mail-Adressen.

Die Sekretärinnen der Grundschule (4) und des Berufskollegs (5) gaben an, dass unvollständige Einträge, unleserliche Handschrift und fehlende Unterschriften ebenfalls zu den häufigsten Schwierigkeiten bei der Dateneingabe zählten.

\footnote{Negativbescheinigungen sind Bescheinigungen über das alleinge Sorgerecht}


\textbf{Wer nutzt das System hauptsächlich an Ihrer Schule?}\\
Die Sekretariate aller fünf Schulen (1, 2, 3, 4, 5) nutzten das System. Darüber hinaus verwendete bei den Sekretärinnen der Förderschule (3) und des Berufskollegs (5) auch die Verwaltung die Software. Zusätzlich zu Sekretariat und Verwaltung wurden bei der Sekretärin des Berufskollegs (5) die Lehrer einbezogen, die über Aufnahmeentscheidungen bestimmten. Die Sekretärin der Förderschule (3) fügte hinzu, dass ihre Schule über mehrere Standorte verfüge.

% todo: \textbf{Welche (Teil-)Faktoren bestimmen das Abschließen einer Bewerbung, sodass sie diese nicht mehr bearbeiten müssen?}\\

\textbf{Beschreiben Sie die Ausgangssituation die vorliegt, bevor Sie die Aufgabe \glqq Bewerbung eines Schülers\grqq{} durchführen.}\\
Die Sekretärin des Gymnasiums (1) erklärte, dass ein Anmeldeformular vorliegen müsse. Alle Datenfelder auf dem Papierformular sollten ausgefüllt sein und wichtige Unterlagen eingereicht werden. Bei Unleserlichkeit würde telefonisch nachgefragt werden.

Die Sekretärin der Realschule (2)  \textit{SchILD} erte, dass die Entscheidung über die Aufnahme eines Kindes auf der Schulempfehlung und vorausgehenden Gesprächen mit der Schulleitung basiere. Im Falle von Sprachbarrieren der Eltern würde das Formular gemeinsam ausgefüllt. Mehrfachanmeldungen sollten vermieden werden und würden in der Regel bei Kennenlernterminen auffallen. Sie erklärte, dass die Schulen der Region eine spezielle Methode anwendeten, um dies sicherzustellen (mittels farbiger Formulare).

Die Sekretärin der Förderschule (3) gab an, dass im Vorfeld mit den Eltern telefoniert würde. Das Kind müsse im Schuleinzugsgebiet leben. Eine Information über die Diagnose Autismus könnte oft schon als Entscheidungsgrundlage für die Eignung des Kindes dienen, allerdings könnten in einigen Fällen auch Gutachten erforderlich sein.

Die Sekretärin der Grundschule (4) berichtete, dass die Eltern ein mehrseitiges Anmeldeformular ausfüllen würden, welches sie dann digitalisiere. Gelegentlich würden die Formulare auch per E-Mail eingereicht.

Die Sekretärin des Berufskollegs (5) beschrieb den Prozess als umfangreich, beginnend mit der Bewerbung des Schülers bis hin zu abrufbaren Daten im Schulverwaltungsprogramm.

\textbf{(DAkkS) Welche fachlichen und technischen Qualifikationen sind zur Bewältigung der Aufgabe erforderlich (Aufgabenbewältigung / Softwarenutzung)? Welche Vorkenntnisse fehlen ggf.?}\\
Die Sekretärin des Gymnasiums (1) äußerte, dass zur Bewältigung der Aufgabe keine besonderen fachlichen oder technischen Kenntnisse erforderlich seien. Grundlegende Lesekompetenz sei ausreichend, spezifisches Wissen über das Schulsystem hingegen nicht notwendig.

Ähnlich äußerte sich die Sekretärin der Realschule (2). Sie betonte, dass nicht einmal Internet-Affinität Voraussetzung sei, allerdings sollte man die Bereitschaft mitbringen, sich mit der Software auseinanderzusetzen und sie ausprobieren zu wollen.

Die Sekretärin der Förderschule (3) nannte technische Grundvoraussetzungen wie einen Computer und Internetzugang. Ihrer Ansicht nach könnten die Sekretärinnen mit der Software umgehen, sobald sie einmal damit gearbeitet haben. Auch sie betonte, dass Kenntnisse des Schulgesetzes nicht zwingend notwendig seien, jedoch sei ein Bewusstsein für Datenschutzfragen wichtig.

Die Sekretärin der Grundschule (4) sprach die Vorteile von Kenntnissen im Umgang mit MS Office und grundlegendem Wissen im Schulgesetz an, obgleich diese nicht zwingend notwendig seien. Sie unterstrich ebenfalls die Bedeutung von Sensibilität in Bezug auf den Datenschutz.

Die Sekretärin des Berufskollegs (5) betonte, dass möglicherweise Schulungen zur Vertiefung des fachlichen Wissens, insbesondere für die Schulverwaltung, fehlen könnten. Sie stellte auch heraus, dass ein Verständnis für die spezifischen Bildungsangebote der eigenen Schule wichtig sei.


\subsection{Zu Beginn der Durchführung}
Zu Beginn der Durchführung der Aufgaben mit der neuen Software zeigten sich bei mehreren Sekretärinnen Schwierigkeiten in der Navigation. Die Sekretärin des Gymnasiums (1) navigierte beispielsweise auf eine leere Seite und konnte den Sinn und Zweck der Menüpunkte \glqq Schüler:innen\grqq{} und \glqq Anmeldungen\grqq{} nicht korrekt differenzieren. Obwohl ihre Aufgabe darin bestand, eine Anmeldung zu erfassen, konnte sie aufgrund der bereitgestellten Informationen nicht richtig navigieren.

Sowohl die Sekretärin des Gymnasiums (1) als auch die der Förderschule (3) mussten dahingehend instruiert werden, wie sie die Aufgabe beginnen können. Beide waren anfangs dabei, einen inkorrekten Prozess zu starten.

Die Sekretärin des Berufskollegs (5) navigierte durch mehrere Menüpunkte, konnte jedoch letztendlich auf die korrekte Seite gelangen. 

\textbf{(DAkkS) Welche Arbeitsschritte sind durchzuführen?}\\
Die Sekretärin des Gymnasiums (1) beschrieb den Prozess in drei Schritten: Zuerst müsse der Schüler im System gesucht werden, dann würden die Daten des Schülers erfasst. Anschließend müsse dem Schüler eine Information übermittelt werden.

Die Sekretärin der Realschule (2) betonte die Bedeutung der persönlichen Anwesenheit des Schülers an der Schule.

Für die Sekretärin der Förderschule (3) umfasste der Prozess mehrere Schritte: Erst müsse geklärt werden, ob ein Rundgang durchgeführt wurde. Danach würden die Daten aufgenommen und das Gutachten für die Akten angefordert. Ein kurzes Kennenlernen und die Entscheidung, in welche Klasse das Kind käme, gehörten ebenso zu den Arbeitsschritten.

Die Sekretärin der Grundschule (4) erläuterte, dass die Eltern zunächst ins Sekretariat kämen, wo ein Gespräch stattfände. Anschließend würden die Daten erfasst und Unterlagen eingereicht. Letztendlich würde eine Schülerakte angelegt.

Die Sekretärin des Berufskollegs (5) führte aus, dass die Daten zuerst in Papierform erfasst und dann ins System eingetragen würden.


\textbf{(DAkkS) Welche Hilfsmittel sind erforderlich (für die Aufgabenbewältigung / zur Softwarenutzung)? Welche davon fehlen ggf., welche sind zusätzlich gewünscht?}\\
Für die Sekretärin des Gymnasiums (1) sind Kopier- und Einfügefunktionalitäten wichtig, um Daten aus anderen Programmen wie E-Mail oder PDF-Reader übertragen zu können.

Die Sekretärin der Förderschule (3) nutzt Google Maps zur Adressrecherche als unterstützendes Hilfsmittel.

Die Sekretärin des Berufskollegs (5) spricht den Bedarf an Laptops und einer Kurzanleitung für den Einstieg in die Softwarenutzung an.

Es wurden keine fehlenden Hilfsmittel oder zusätzlich gewünschte Tools spezifisch angegeben.

\textbf{(DAkkS) Welche Ergebnisse / Teilergebnisse entstehen und wie werden diese ggf. verwertet / weitergeführt?}\\
Die Sekretärin der Grundschule (4) erklärt, dass am Ende des Prozesses der Schüler angemeldet ist. Die in der Anwendung erfassten Daten können bei Bedarf für weitere schulische Prozesse herangezogen werden.

Die Sekretärin des Berufskollegs (5) stellt heraus, dass nach Abschluss des Prozesses alle notwendigen Daten für die Bewerbung des Schülers erfasst sein sollten. Sie betont insbesondere die Wichtigkeit der Erfassung der vorherigen Schule, da es sehr aufwändig wäre, diese Information nachträglich herauszufinden. Die erfassten Daten bilden somit eine wesentliche Grundlage für die weitere schulische Laufbahn und Verwaltungsprozesse des Schülers.

\textbf{(DAkkS) Welche wichtigen Sonderfälle müssen berücksichtigt werden? (bzw. fallen dem Benutzer spontan ein; z. B. zur Arbeitsteilung / Zusammenarbeit)}\\
Die Sekretärin des Gymnasiums (1) weist darauf hin, dass Wechsel innerhalb des Schuljahres und das Wiederholen von Stufen berücksichtigt werden müssen. Bei Flüchtlingskindern sind möglicherweise nicht alle erforderlichen Daten vorhanden. Sie stellt besondere Situationen heraus, wie langfristige Beurlaubungen oder Fälle, in denen unklar ist, in welche Klasse ein Kind gehen soll. Ein Beispiel wäre ein Kind, das eine Therapie macht und nicht am regulären Schulbetrieb teilnimmt, aber dennoch wegen der Schulpflicht an der Schule angemeldet werden muss. Solche Kinder werden oft in der Klinik unterrichtet.

Die Sekretärin der Realschule (2) weist darauf hin, dass Unterricht in der Herkunftssprache im Voraus beim Schulamt angemeldet werden muss. Hierfür gibt es jährlich ein neues Formular, das unterschrieben werden muss.

Die Sekretärin der Förderschule (3) erklärt, dass in Förderschulen keine spezifischen Schulstufen existieren, was bei der Anmeldung berücksichtigt werden muss.

Die Sekretärin der Grundschule (4) betont, dass besondere Fälle wie unterschiedliche Staatsangehörigkeiten und bestimmte Sorgerechtssituationen die Anmeldung komplizieren können.

Die Sekretärin des Berufskollegs (5) identifiziert Schüler mit Förderbedarf als Sonderfälle. Sie merkt auch an, dass manche Schüler in ihrem Alter unsicher sind, ob sie sich als männlich oder weiblich identifizieren. Des Weiteren stellt sie fest, dass bei minderjährigen Schülern eine Unterschrift der Eltern erforderlich ist. Ein weiterer Sonderfall sind Schüler, die sich zu spät bewerben und möglicherweise nicht die notwendigen Voraussetzungen für bestimmte Bildungsgänge erfüllen. Sie stellt fest, dass wenige Schüler genau wissen, was sie lernen wollen oder welchen Bildungsgang sie besuchen möchten.


\subsection{Während der Durchführung}
%todo überleitungstext \textbf{Für jeden Tab / Seite: Welche Seiteninhalte sind unverständlich?}\\

\textbf{\glqq Schüler\grqq{}-Tab}\\
Alle Studienteilnehmer hatten Schwierigkeiten zu verstehen, wofür das Feld \glqq ID-Schlüssel\grqq{} gedacht ist.

Die Sekretärin des Gymnasiums (1) konnte den Zweck der Schülersuche-Funktion nicht nachvollziehen.


\textbf{Anmerkungen zum Tab \glqq Bildungsgang\grqq{}}\\
Für die Sekretärin des Gymnasiums (1) war das Dropdown-Feld \glqq Klasse\grqq{} unklar, da keine Daten vorhanden waren und noch nicht feststand, welche Klassen es geben würde. Sie wählte den Status \glqq Warteliste\grqq{}, um die Bewerbung als \glqq in Bearbeitung\grqq{} zu markieren und konnte nicht nachvollziehen, warum in der Anwendung vermerkt werden soll, dass eine Aufnahmeberatung stattgefunden hat.

Die Sekretärin der Realschule (2) hätte den Begriff \glqq Neuaufnahme\grqq{} in der Auswahl für den Aufnahmestatus erwartet, da diese Terminologie auch bei \glqq  \textit{SchILD} \grqq{} verwendet wird. Sie merkte an, dass die Telefonnummervorwahl nicht zum Ort passte und dass sie in der Realität noch einmal bei der anderen Telefonnummer nachfragen würde. Sie hatte auch Schwierigkeiten, das Beschulungsbeginndatum korrekt einzugeben.

Die Sekretärin der Förderschule (3) hielt die Sortierung der Schuljahre für nicht sinnvoll und stellte fest, dass man mitunter keine medizinischen Informationen erfassen darf.

Die Sekretärin der Grundschule (4) war verwirrt durch die Meldung, dass ein Antrag auf vorzeitige Einschulung stattfindet und fand die Bildungsgangbezeichnungen verwirrend.

Die Sekretärin der Berufsschule (5) hatte Diskussionen mit einer Kollegin, wie das Feld \glqq Schuljahr\grqq{} zu interpretieren ist. Sie hatte eine andere Meinung als die Kollegin, dass es sich dabei um das aktuelle Schuljahr handelt und nicht um das Schuljahr, das der Schüler zuvor besucht hat. Der Anmeldestatus war für sie nicht klar interpretierbar.

\textbf{Anmerkungen zum Tab \glqq Persönliche Daten\grqq{}}\\
Die Sekretärin des Gymnasiums (1) trug die Hausnummer zunächst intuitiv zusammen mit der Straße ins Feld \glqq Straße\grqq{} ein, anstatt ins Feld \glqq Hausnummer\grqq{}. Sie gab auch die E-Mail-Adresse des Sorgeberechtigten in das E-Mail-Feld des Schülers ein. Sie merkte an, dass sie es bevorzugen würde, wenn der Tab \glqq Persönliche Daten\grqq{} vor dem Tab \glqq Bildungsgang\grqq{} kommen würde, so wie es bei \glqq  \textit{SchILD} \grqq{} der Fall ist. Sie schlug auch vor, dass der Eintrag für \glqq Staat\grqq{} standardmäßig mit \glqq Deutschland\grqq{} vorbelegt sein sollte. Sie wünschte sich außerdem einen Hinweis, dass bei Sek1-Schülern grundsätzlich die Telefonnummer der Sorgeberechtigten und bei Sek2-Schülern die des Schülers erwartet wird.

Die Sekretärin der Realschule (2) fand die Erfassung von Ortsteilen interessant.

Die Sekretärin der Förderschule (3) überprüft unbekannte Straßen mithilfe von Google Maps auf ihre Existenz. Sie schlug vor, dass es sinnvoll wäre, wenn Eltern angeben könnten, zu welchen Zeiten sie telefonisch erreichbar sind. Sie fände es auch gut, wenn mehrere Telefonnummern für Notfälle erfasst werden könnten.

Die Sekretärin der Berufsschule (5) wünschte sich eine Anzeige, die darauf hinweist, ob der Schüler volljährig ist. Sie hatte Schwierigkeiten mit der Suche nach der Postleitzahl-Ort-Kombination.

\textbf{Anmerkungen zum Tab \glqq Sorgeberechtigte\grqq{}}\\
Die Sekretärin des Gymnasiums (1) hielt es für wichtig, einen Nachweis dafür zu haben, ob eine Person das alleinige Sorgerecht hat oder ob beide Unterschriften vorhanden sind, um zu verhindern, dass sich eine Person über die andere hinwegsetzt. Sie hätte erwartet, dass es eine Funktion zur Datenübernahme der Adressdaten aus den persönlichen Daten des Schülers gibt. Sie war auch unsicher über den Zweck des Postfachs. Sie war verärgert, als der Sorgerechtsprozess durch einen versehentlichen Klick neben das Popup abgebrochen wurde, und forderte, dass es nicht möglich sein sollte, Daten zu verlieren, sei es durch danebenklicken oder Ausloggen.

Die Sekretärin der Realschule (2) trug die Hausnummer auch intuitiv zusammen mit der Straße in das Feld \glqq Straße\grqq{} ein.

Alle Sekretärinnen außer 5 (also 1, 2, 3 und 4) versuchten, die Straße in das Feld \glqq Adressart\grqq{} einzugeben. Die Sekretärin der Förderschule (3) erklärte auf Nachfrage, dass sie intuitiv die folgende Eingabereihenfolge erwartet hätte: Staat => Postleitzahl/Ort => Straße.

Der \glqq Erstellen\grqq{}-Button wurde unterschiedlich interpretiert. Die Sekretärin der Förderschule (3) interpretierte ihn im Sinne von \glqq Speichern\grqq{}, während die Sekretärin der Hauptschule (4) annahm, dass damit ein weiterer Sorgeberechtigter angelegt werden kann, was sie auf das Plus-Symbol zurückführte.

Die Sekretärin der Berufsschule (5) empfand das Erfassungsformular für die Sorgeberechtigten im Vergleich zu den anderen Formularen als unattraktiv und unübersichtlich.

\textbf{Anmerkungen zum Tab \glqq Notfallkontakte\grqq{}}\\
Die Sekretärin der Realschule (2) schlug vor, dass es hilfreich wäre, die Art der Telefonnummer (im Sinne von dienstlich, privat, mobil) für die Notfallkontakte hinterlegen zu können und mehrere Telefonnummern einer Person zuzuordnen. Sie konnte zeitweise nicht erkennen, wie sie den Datensatz speichern kann, da das Autocomplete-Dropdown-Feld den Speichern-Button versteckte. Sie nahm dieses Feld nicht als ein Feld wahr, das man benutzerdefiniert ausfüllen kann, sondern als ein Feld, in dem man zwingend einen der vorgeschlagenen Werte auswählen muss.

Die Sekretärin der Förderschule (3) versuchte, mehrere Nummern in ein Feld einzugeben, was ihr nicht gelang. Wie 2 schlug sie vor, dass es möglich sein sollte, Erreichbarkeitszeiträume einzutragen. Sie wies auch darauf hin, dass es wichtig sei, die Rolle des Notfallkontakts zu erfassen, um im Notfall beurteilen zu können, ob sensible Informationen weitergegeben werden dürfen. Eine Rangfolge der Notfallkontakte hielt sie nicht für sinnvoll. Sie hätte erwartet, die Notfallkontakte früher im Prozess zu sehen.

Die Sekretärin der Hauptschule (4) empfand die Eingabe der Notfallkontakte als unpraktisch, da sie die gleichen Daten wie bei den Sorgeberechtigten eingeben musste. In \textit{SchILD} werden diese Daten automatisch übernommen.

\textbf{Anmerkungen zum Tab \glqq Migrationshintergrund\grqq{}}\\
Die Sekretärin des Gymnasiums (1) äußerte Zufriedenheit mit dem Abschnitt zum Migrationshintergrund und betonte, dass alle wichtigen Informationen vorhanden seien. Sie war jedoch der Meinung, dass das Eingabefeld angezeigt werden sollte, selbst wenn die Option \glqq Migrationshintergrund liegt nicht vor\grqq{} ausgewählt ist.

Die Sekretärin der Gesamtschule (5) stellte die Definition von \glqq Migrationshintergrund\grqq{} in Frage. Sie war sich unsicher und spekulierte darüber, ab wann ein Migrationshintergrund angenommen wird.

\textbf{Anmerkungen zum Tab \glqq Letzte Tätigkeit\grqq{}}\\
Die Sekretärin des Gymnasiums (1) kritisierte die Reihenfolge der Bereiche und schlug vor, dass die \glqq Letzte Tätigkeit\grqq{} innerhalb des Abschnitts \glqq Schüler-Daten\grqq{} erfasst werden sollte, da es sich hierbei um ergänzende Informationen zur Person handelt. Sie hatte Schwierigkeiten mit der Suche nach der letzten Schule, da die Suche eine Checkbox überdeckte und die Sortierung der Ergebnisse ihrer Meinung nach die lokale Umgebung bevorzugen sollte.

Die Sekretärin der Gemeinschaftsschule (3) hielt den Tab \glqq Letzte Tätigkeit\grqq{} für überflüssig, da ein Kind, das sich für die Grundschule anmeldet, noch keinen Schulabschluss haben kann. Sie war verwirrt und konnte nicht unterscheiden, ob die letzte Tätigkeit des Schülers oder des Sorgeberechtigten erfragt wird. Die angestrebte Schulstufe wurde von ihr als letzte Schulstufe interpretiert.

Die Sekretärin der Gesamtschule (5) hatte Schwierigkeiten, eine einmal ausgewählte Grundschulempfehlung wieder zu entfernen. Sie wies darauf hin, dass das Feld \glqq Angestrebte Schulstufe\grqq{} und der Tab \glqq Letzte Tätigkeit\grqq{} einander zu widersprechen scheinen.

\textbf{Anmerkungen zum Tab \glqq Qualifikationen\grqq{}}\\
Die Sekretärin der Gesamtschule (5) stellte fest, dass in der Liste der Qualifikationen einige Schulabschlüsse, wie beispielsweise der Hauptschulabschluss, fehlten. Sie äußerte Kritik und regte eine Überarbeitung der Auswahlmöglichkeiten an.

\textbf{Anmerkungen zum Tab \glqq Termine\grqq{}}\\
Sekretärin 5 konnte den Zweck der Seite \glqq Termine\grqq{} nicht erkennen. Sie gab an, dass sie ohnehin keine Zeit hätte, 500 Personen zu Terminen einzuladen. Sie merkte jedoch an, dass einige Bildungsgangleitungen vorab Gespräche mit den Schülern führen möchten.

\textbf{Anmerkungen zum Tab \glqq Bemerkungen\grqq{}}\\
Sekretärin 2 äußerte Unsicherheit darüber, wer Zugriff auf die internen Notizen hat. Sie definierte das Feld \glqq Interne Notiz\grqq{} und \glqq Bemerkung für Schüler*in\grqq{} als \glqq Anmerkung ist unsere Meinung, Bemerkungen sind Fakten\grqq{}. Sekretärin 4 war sich ebenfalls unsicher, welche Informationen dem Schüler zugänglich sind. Im Gegensatz zu 2 interpretierte 4, dass die \glqq Interne Notiz\grqq{} nur für die Schule sichtbar ist und das \glqq Bemerkung\grqq{}-Feld auch vom Kind gesehen werden kann, fühlte sich aber durch den Infotext oben im Formular verwirrt.

Sekretärin 3 machte darauf aufmerksam, dass sensible Informationen, wie zum Beispiel Behinderungen, möglicherweise nicht erfasst werden dürfen. Sie betonte die Notwendigkeit einer Benutzerführung, die das Bewusstsein für gesetzliche Anforderungen stärkt.

Sekretärin 5 war überrascht, dass dieser Tab erschien, da sie zuvor nicht bemerkt hatte, wie weit sie im Prozess fortgeschritten war. Sie stellte fest, dass dieser Tab zuvor aufgrund des Overflow-Verhaltens des Browsers verdeckt war.

Sekretärin 1 gab an, dass sie den Tab \glqq Anmerkungen\grqq{} an einer früheren Stelle im Prozess erwartet hätte.

\textbf{Anmerkungen zum Tab \glqq Zusammenfassung\grqq{}}\\
Sekretärin 1 erlebte eine unklare Fehlermeldung beim Klick auf \glqq Speichern\grqq{}. Sie wünscht sich klarere Anweisungen zur Fehlerbehebung und einen direkten Fokus auf das fehlerhafte Feld.
Sekretärin 2 stellte eine Verzögerung beim Laden der Inhalte fest. Sie befürchtet, dass die per E-Mail gesendete Bestätigungsmeldung mit der Formulierung \glqq Angemeldet\grqq{} falsche Erwartungen bei den Schülern wecken könnte. Sie war sich auch unsicher, ob das Schuljahr korrekt erfasst wurde und merkte an, dass noch eine Unterschrift von den Eltern für die Anmeldung erforderlich ist.
Sekretärin 3 interpretierte die Meldung auf der Zusammenfassungsseite als \glqq Die Mutter wurde kontaktiert\grqq{}.
Sekretärin 4 nahm an, dass die Anmeldung des Kindes als \glqq vorzeitige Einschulung\grqq{} gekennzeichnet wurde, was sie als Fehler ansah. Sie war sich auch unsicher, ob sie nun eine \glqq Anmeldung\grqq{} oder nur eine \glqq Bewerbung\grqq{} eingereicht hat. Sie bemerkte, dass Informationen zu wichtigen Themen wie dem Nachweis des Masernschutzes oder der Betreuung für beispielsweise die offene Ganztagsschule fehlten.
Sekretärin 5 kritisierte den Informationsgehalt der Seite. Ihrer Meinung nach sind zu viele Informationen im Panel \glqq Bildungsgang\grqq{} enthalten, insbesondere das Kürzel des Bildungsgangs, das für sie schwer zu verstehen war.



Nach der Durchführung:

\textbf{Konnten Sie die Aufgabe aus Ihrer Sicht erfolgreich und vollständig abschließen? Falls nein - was hat Sie daran gehindert?}\\

\textbf{Wie effektiv unterstützt die Webanwendung Sie bei der Aufgabe?  Gab es positive oder negative Erfahrungen?}\\

\textbf{Haben Sie sämtliche Inhalte der Aufgabe verstanden? Gab es Stellen, an denen Sie sich mehr Unterstützung gewünscht hätten?}\\

\textbf{Gab es Schwierigkeiten oder Verwirrungen bei der Aufgabe? Wenn ja, welche?}\\

\textbf{Wie verständlich waren die Rückmeldungen der Anwendung?}\\

\textbf{Welche Fähigkeiten setzt die Anwendung Ihrer Einschätzung nach voraus?}\\

\textbf{Wie sehr entspricht die Umsetzung in der Software der Realität?}\\

\textbf{Was handhaben Sie in Ihrem Arbeitsalltag bei nicht-digitalen Bewerbungen gewöhnlicherweise anders als in der Anwendung?}\\

\textbf{Was würde Sie noch daran hindern die Software in Ihrem Arbeitsalltag einzusetzen?}\\

\textbf{Wie sehr erleichtert Ihnen die Anwendung ihre Arbeit?}\\

\textbf{Gibt es Funktionen, die Sie in ähnlichen bzw. anderen Anwendungen genutzt haben, die Sie hier vermissen?}\\

\textbf{Welche Software sollte man aus Ihrer Sicht in Schüler Online integrieren bzw. eine Schnittstelle schaffen?}\\

\textbf{Welche Dokumente würden Sie gerne im Prozess ausdrucken können?}\\

\textbf{Welche Dokumente würden Sie gerne einscannen wollen und beim Datensatz hinterlegen?}\\
Schreibe meinen Methodikteil für meine wissenschaftliche Arbeit in einem wissenschaftlichen, präzisen und interessantem Schreibstil. Vermeide Wortwiederholungen. Schreibe in einer Vergangenheitsform wie beispielsweise dem Präteritum.
In diesem Kapitel solltest du die Methoden und Techniken, die du zur Durchführung deiner Projektarbeit verwendest, beschreiben und begründen. Hier sollten auch die Einzelheiten zur Durchführung des leitfadengestützten Interviews enthalten sein.
Für das leitfadengestützte Experteninterview solltest du in der Methodik deiner Projektarbeit den Ablauf und die Durchführung des Interviews beschreiben. Hierbei sollst du beispielsweise folgende Punkte erwähnen:


\section{Methodik}
In diesem Kapitel solltest du die Methoden und Techniken, die du zur Durchführung deiner Projektarbeit verwendest, beschreiben und begründen. Hier sollten auch die Einzelheiten zur Durchführung des leitfadengestützten Interviews enthalten sein.
Für das leitfadengestützte Experteninterview solltest du in der Methodik deiner Projektarbeit den Ablauf und die Durchführung des Interviews beschreiben. Hierbei sollst du beispielsweise folgende Punkte erwähnen:
Teilnehmer: Wer wurde für das Interview ausgewählt und warum? Wie viele Experten wurden befragt? 

Die Teilnehmer wurden aus zwei Pools rekrutiert. Der erste Pool bestand aus Kontakten, die eine Schulung für Schüler Online 2.0 besucht haben. Hierbei wurde in der Schulungspräsentation eine Folie vorbereitet, in der die Studie vorgestellt wurde und um Teilnehmer gebeten wurde. Der andere Pool bestand aus einer Kaltakquise. Hierbei wurden telefonisch Schulen aus 2 umliegenden Kreisen angerufen und für die Studie gewonnen. Aufgrund von Verschwiegenheitsvereinbarungen muss hierzu auf eine konkrete Angabe der Region verzichtet werden.
Die Auswahl von fähigen Teilnehmern erfolgte basierend auf mehreren Kriterien. Zum einen war es wichtig, dass Teilnehmer im Berufsleben das Produkt auch tatsächlich sinnvoll verwenden kann (Inklusionskriteritum). Dies prädestiniert Sekretariatsmitarbeiter, Schulverwaltungsassistenten und gegebenenfalls Schulleitungen. Die Teilnehmer mussten aktiv berufstätig in den genannten Berufen sein und nicht vom Beruf ausgeschieden bzw. pensioniert sein (Exklusionskriteritum). Auszubildende sowie Lehrer wurden von der Studie ausgeschlossen (Exklusionskriteritum), da sie zu wenig Berührungspunkte und Erfahrung mit dem Tätigkeitsfeld der Software besitzen. Es war egal, ob der Teilnehmer bereits Vorerfahrungen mit dem Produkt besitzt. Das Alter, das Geschlecht, die Berufserfahrung, der ethnische Hintergrundund der sozioökonomischer Status der Teilnehmer spielte keine Rolle bei der Rekrutierung. 
Die Verfügbarkeit und Willigkeit erforderte lediglich eine freiwillige Teilnahme, also eine nicht zwanghaft delegierte wie beispielsweise eine Teilnahmeverpflichtung durch Vorgesetzte wie die Schulleitung.

Die Durchführung der Beobachtungen und Interviews fanden entsprechend eines Feldtests während der regulären Arbeitszeit an den regulären Arbeitsplätzen der Studienteilnehmern statt. Eine Laborumgebung wurde einerseits aufgrund der fehlenden Realitätstreue und andererseits aufgrund von finanziellen Gründen nicht gewählt. Insbesondere die Ausstattung (wie beispielsweise Computer mit einer langsamen Internetverbindung) sowie die Tatsache, dass beispielsweise Sekretärinnen durch Telefonate typischerweise ihre Tätigkeit unterbrechen müssen waren tragende Kriteritum, weshalb der Feldtest bevorzugt wurde. Die Interviews wurden während der Sommerferien terminiert, dies wird auch noch Gegenstand der Diskussion sein.

Vorbereitung: Wie wurde das Interview vorbereitet? Wurde den Experten im Voraus eine Liste mit Fragen oder Themenbereichen zur Verfügung gestellt?
Interview: Welche verschiedenen Szenarien kommen vor, wenn sich ein Schüler bei Ihnen bewerben möchte - wie muss man sich das vorstellen?
Beispiel - (nicht erwähnen!!): "Ein Schüler sitzt vor mir und ich sitze am Rechner, ich trage seine Daten ein.
Soll Hypothese bzw. Bias belegen: Eine Sekretariatsmitarbeiterin führt die Bewerbung im Beisein und Diallog mit dem Schüler durch.
Wie kann man die Schüler beschreiben, die Ihnen gegenüber sitzen?
Bei Sek 1 + Primarstufe: Charakterisieren sie klassische Elternteile
Verwenden Sie die Software Schüler Online für Bewerbungen?
Wie häufig?
Welche Schwierigkeiten bereitet Ihnen die Software Schüler Online 1.0 (Ist eine Folgefrage, die nur Personen die bereits 1.0 verwenden gestellt werden kann)
=> Vorgehensweise: Ich kann entweder selber der Schüler sein oder ich bitte Chrissy darum, eine Schülerin oder Mutter zu simulieren.
=> Es gibt drei Szenarien, die der Proband durchlaufen sollte, die in der Realität vorkommen:
    1: Guten Tag, ich bin Hans Meier, geboren am 01.05.2008 und möchte auf ihr Berufskolleg
    Ziel: 
    Das Szenario "Mit Betrieb anmelden" existiert nicht nur für die Sek II Anmeldung.
 
 Durchführung: Wie wurde das Interview durchgeführt? Wurde es in persönlichen Gesprächen oder telefonisch durchgeführt? Wie lange dauerte das Interview? Gab es Besonderheiten % oder Schwierigkeiten während des Interviews?
 Aufzeichnung: Wie wurde das Interview aufgezeichnet? Wurden Notizen gemacht, oder wurde das Interview aufgezeichnet und transkribiert?
 Ethik: Welche ethischen Gesichtspunkte wurden bei der Durchführung des Interviews berücksichtigt, z.B. in Bezug auf Datenschutz oder Anonymität?
 
 => Beobachtung
 		- Wie wurde die Beobachtung durchgeführt?
 			- Wer, wie viele, Umgebung, welcher Teil der Anwendung...
 			- Grobstruktur des Evaluations
 			- Feldtest 
 			- Wie lief das "Intro" bei der Beobachtung ab? Was hat der Beobachter den Beobachteten vorher gesagt und erläutert?
 		- Filmen?

material
pc, meine anwendung, 
zur aufnahme: Microsoft Word und Fragebogen
Ich habe es immer zu zweit durchgeführt mit Assistent
ich muss ihn ein bisschen beschrieben und schreiben was er können musste (ist er ne ungelernte Kraft? Er kennt sich mit der Anwendung aus)
Wie haben wir den Fragebogen erstellt? 
Wie wurden die Sekretärinnen akquiriert?
Welche Auswahlkriterien für die Personengruppe?
Schulleiter zeigt nochmal weitere Aspekte auf
Offene Fragen gemacht => Warum offene Fragen
induktives vorgehen
Fragen wurden nicht vorab geschickt, sie wussten aber grob worum es geht
ich habe jedes Sekretärin vorher telefonisch gesprochen
bei begrüßung haben wir grob erklärt was die Anwendung betreibt 
wir haben uns seitlich neben die sekretärin gesetzt und über die schulter geschaut, sodass wir beide den bildschirm sehen konnten
ich habe der sekretärin keine Fragen beantwortet und mich dumm gestellt
Mein Schreiberling wir haben das im nachgang direkt ausformuliert (nur die unmissverständlichen Dinge, welche man nicht fehlinterpretieren konnte)
Schreibling A bei Termin 1,4. Schreiberling B bei Termin 2,3,5,6
Wie werte ich die Sachen aus? Habe ich da was spezielles genutzt? 


Fragebogen ist delegiert ans Team und ein Zwischenergebnis. Ist ein Ergebnis eines Expertenteams für dieses Programm
Experten sind Material

\newpage
\section{Material und Methode}

Die Teilnehmer wurden aus zwei Pools rekrutiert. Der erste Pool bestand aus Teilnehmern oder vermittelnden Kontakten\footnote{Ein vermittelnder Kontakt ist ein Kontakt, der einen potenziellen Studienteilnehmer aus seinem persönlichen Umfeld kontaktiert hat und diesen gefragt hat, ob er an der Studie teilnehmen möchte}, die eine Schulung für\textit{Schüler Online 2.0} besucht haben. Hierbei wurde in der Schulungspräsentation eine Folie vorbereitet, in der die Studie vorgestellt wurde und um Teilnehmer gebeten wurde. Der andere Pool bestand aus einer Kaltakquise. Hierbei wurden telefonisch Schulen aus zwei umliegenden Kreisen angerufen und für die Studie gewonnen. Aufgrund von Verschwiegenheitsvereinbarungen muss hierzu auf eine konkrete Angabe der Region verzichtet werden.
Die Auswahl von fähigen Teilnehmern erfolgte basierend auf mehreren Auswahlkriterien. Zum einen war es wichtig, dass Teilnehmer im Berufsleben das Produkt auch tatsächlich sinnvoll verwenden kann (Inklusionskriteritum). Dies prädestiniert Sekretariatsmitarbeiter, Schulverwaltungsassistenten und gegebenenfalls Schulleitungen. Die Teilnehmer mussten aktiv berufstätig in den genannten Berufen sein und nicht vom Beruf ausgeschieden bzw. pensioniert sein (Exklusionskriteritum). Auszubildende sowie Lehrer wurden von der Studie ausgeschlossen (Exklusionskriteritum), da sie zu wenig Berührungspunkte und Erfahrung mit dem Tätigkeitsfeld der Software besitzen. Es war egal, ob der Teilnehmer bereits Vorerfahrungen mit dem Produkt besitzt. Das Alter, das Geschlecht, die Berufserfahrung, der ethnische Hintergrundund der sozioökonomischer Status der Teilnehmer spielten keine Rolle bei der Rekrutierung. 
Die Verfügbarkeit und Willigkeit erforderte lediglich eine freiwillige Teilnahme, also eine nicht zwanghaft delegierte wie beispielsweise eine Teilnahmeverpflichtung durch Vorgesetzte wie die Schulleitung.

Die Durchführung der Beobachtungen und Interviews fanden entsprechend eines Feldtests während der regulären Arbeitszeit an den regulären Arbeitsplätzen der Studienteilnehmern statt. Eine Laborumgebung wurde einerseits aufgrund der fehlenden Realitätstreue und andererseits aufgrund von finanziellen Gründen nicht gewählt. Insbesondere die Ausstattung (wie beispielsweise Computer mit einer langsamen Internetverbindung) sowie die Tatsache, dass beispielsweise Sekretärinnen durch Telefonate typischerweise ihre Tätigkeit unterbrechen müssen waren tragende Kriteritum, weshalb der Feldtest bevorzugt wurde. Die Interviews wurden während der Sommerferien terminiert, dies wird auch noch Gegenstand der Diskussion sein.

Ausarbeitung des Fragebogens
Für den Fragebogen wurde eine Gruppe von fachlichen Experten herangezogen, um ebenjenen auszuarbeiten beziehungsweise zu formulieren. Diese Gruppe bestand aus Leuten, die im Berufsleben die Anforderungen der Software aufnehmen und definieren und dementsprechend über eine hohe fachliche Expertise verfügten. Die Ausarbeitung der Fragen erfolgte ohne einen Usability Experten, lediglich mit einen Studenten, der im Rahmen seines Studiums das Studienfach \glqq Usability Engineering\grqq{} belegte. Der Fragebogen wurde jedoch noch von einem Professor für Usability Engineering gegengeprüft und auf inhaltliche Eignung geprüft. 
Anforderungen an den Fragebogen und somit auch an die Fragen waren, dass sie klar, verständlich und offen formulierte sein sollten. In der Reihenfolge sollten zunächst leicht zu beantwortende Fragen kommen. Der Sprachschatz sollte den Kenntnissen eines Schulsekretariatsmitarbeitenden entsprechen \cite{Kruse_2015}. 

Szenario
Es wurde jeweils ein Szenario für die drei Aufgaben im System vorab vorbereitet. Die individuellen technischen Voraussetzungen für die Schulen wurden in diesem Zuge geschaffen. So wurde das Anlegen von Bildungsangeboten an der Schule. Die drei Aufgaben lauteten:
Aufgabe 1: \glqq Erstellen Sie bitte für dieses Anmeldeformular von 'Max Müller' eine Bewerbung an Ihrer Schule.\glqq  (Details siehe Anhang \ref{section-Musteranmeldeformular})
Aufgabe 2: \glqq Bearbeiten Sie bitte die Bewerbung von \textit{Lotta Meier} nach eigenem Ermessen.\glqq  Dieser Datensatz war so konzipiert, dass die Bewerbung von einer abgebenden Schule oder der Gemeinde gestellt wurde.
Aufgabe 3: \glqq Bearbeiten Sie bitte die Bewerbung von \textit{Konrad Schulz} nach eigenem Ermessen\grqq{}. Dieser Datensatz war so konzipiert, dass die Bewerbung von den Eltern des Schülers gestellt wurde.
Die Datensätze unterschieden sich von Schule zu Schule insofern, alsdass das Geburtsjahr der Schulstufe angemessen angepasst wurde. Für Bewerbungen an der Primarstufe wurde das Geburtsdatum des Datensatzes so gewählt, dass das Schulkind zum Zeitpunkt des ersten Schultages 6 Jahre sein würde. Bei einer Bewerbung der Sekundarstufe 1 war das Schulkind 10 Jahre alt und für die Sekundarstufe 2 war es 16 Jahre alt. Die Datensätze waren allesamt fiktiv, um Problemem hinsichtlich Vertraulichkeit vorzubeugen. Dies wurde den Teilnehmern auch kommuniziert.

Vorbereitung der Studie 
Der Fragebogen wurde den Studienteilnehmern nicht vorab zugesandt. Es wurde jedem in den vorangegangenen Telefonaten mitgeteilt, dass er oder sie keine Vorbereitung für die Studie treffen muss. Die Software und der Zweck der Studie wurde jedem Teilnehmer in diesem Zuge dargelegt.

Interviewer und Schreiberling
Es gab zwei Rollen vonseiten der Moderation der Studien. Es gab einen Interviewer, welcher besetzt wurde durch eine Person, die über gute fachliche Kenntnisse verfügte, da dieser die Software mehrere Jahre mitentwickelt hat. Bei ihm lagen keine umfangreichen praktischen Vorerfahrungen hinsichtlich Interviewtechniken vor. Die Rolle des Schreiberlings wurde besetzt durch zwei Personen, wobei Schreiberling A an den Interviews 1 und 4 und Schreiberling B an den Interviews 2,3 und 5 teilnahm. Beide Schreiberlinge verfügten über grundlegende fachliche Kenntnisse, die diese im ersten Jahr der Mitentwicklung an der Software erworben haben.

Durchführung der Studie
Für die Studie sind der Student mit der geringfügigen Usability-Vorerfahrung in der Rolle als Interviewer sowie ein Schreiberling vor Ort in die Schulen gegangen und haben sich kurz vorgestellt und grob geschildert, welchen Zweck die Anwendung verfolgt. Anschließend haben sie sich einen Platz gesucht, der so gelegen war, dass man sowohl den Monitor als auch den Probanden im Blickfeld hatte. Der Interviewer sollte sowohl die Fragen des Fragebogens als auch klarifizierende Rückfragen stellen. Der Schreiberling sollte sich hingegen primär darauf konzentrieren, die Antworten und Beobachtungen zu notieren.
Sowohl der Schreiberling als auch der Interviewer hatten die Anweisung, dass sie das Verhalten der Testperson und die Durchführung der Aufgaben beobachten sollen. 
Damit ein realistisches Szenario gewährleistet werden konnte, wurden keine fachlichen Rückfragen vonseiten der Studienteilnehmer beantwortet, es wurde sich dumm gestellt. Die Studienteilnehmer würden in der Realität beziehungsweise der Praxis in der Regel nur die Support Hotline und die Dokumentation als Hilfsmittel in Anspruch nehmen können, nicht jedoch eine unterstützende Person die ihre Fragen unmittelbar beantworten kann - insofern sie keine Kollegen haben, die die Software bereits kennen. 
In der ersten Durchführung mit dem Gymnasium wurde in Anlehnung an die Empfehlung des Leitfaden Usability ein erfahrener Requirements Engineer (der Product Owner der Software) mittels Anruf zugeschaltet, der \glqq in Form einer Supervision die Gesprächssituation beobachtet, bewertet und anschließend mit dem Beobachteten [besprochen hat]\grqq{}\cite[p.~133]{dakks}.
Es wurde den Teilnehmern versichert, dass die Software getestet werden soll und nicht der Anwender, damit konstruktives Feedback erhalten werden kann, die Teilnehmer mutig genug sind Fehler zu machen und dass die Teilnehmer keinen Drang dazu haben, einen fiktiven Test bestehen zu wollen.
Die Dauer der Interviews umfasste eine bis drei Stunden. 
Die Anonymität der Teilnehmer wurde ihnen vorab zugesichert.
Das Interview und die Beobachtung fanden als Einheit statt.
In drei Szenarien (Gymnasium, Realschule und zeitweise bei der Förderschule) gab es eine unvorhergesehene zu treffende Entscheidung. Man traf zwei Mitarbeiter an und es musste geklärt werden, ob beide am Interview teilnehmen würden oder nur die Person, mit der vorab telefoniert wurde. Die Entscheidung hierzu fiel in allen Fällen darauf, das Interview ausschließlich mit der vorab rekrutierten Person zu führen, Kommentare, Diskussionen und Beratungen zwischen den beiden Personen jedoch jederzeit zuzulassen, damit ein realistischeres Szenario erreicht werden würde. In der Praxis arbeiten die beiden Personen schließlich zusammen und würden sich gegenseitig bei Problemen helfen.
Es wurden einzig und allein die Antworten und Beobachtungen des Interviewpartners in den Notizen festgehalten.
Die Interviews fanden in allen Tests in einem Sekretariat mit Sekretariatsmitarbeitern statt. Es wurde direkt in Microsoft Word mitprotokolliert und keine Audio- oder Filmaufnahmen erzeugt (abgesehen von dem Telefonat mit dem Product Owner).

Nachbereitung
Innerhalb von wenigen Stunden wurden Ergebnisse nachbereitet. Notizen, die zuvor zu kurz formuliert wurden, damit man hinterherkommt mit dem Schreiben wurden präzisiert und ausformuliert. Dies erfolgte lediglich bei den Notizen, die unmissverständlich waren und bei denen man keine Fehlinterpretationen beim ausformulieren machen konnte. Bei Punkten, die potenziell fehlinterpretiert werden konnten oder unklar im Nachhinein waren, wurden keine Ausformulierungen durchgeführt, sondern die Notiz so belassen wie sie mitgeschrieben wurde. 
Zur Auswertung wurden keine Analysetools oder dergleichen herangezogen. 

Weiteres Material
Für jede Untersuchung war ein Computer und eine Internetverbindung notwendig. Weiteres Material wurde nicht vorausgesetzt.


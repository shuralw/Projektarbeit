Nutzungskontext
in Abstimmung mit meiner Projektleitung habe ich einige Entscheidungen bezüglich der Fixierung des Nutzungskontext-Ausschnitts getroffen, um einen möglichst großen Nutzen aus dieser Untersuchung zu schlagen.
Der zentrale Fokus liegt nach wie vor auf dem Prozess „Bearbeitung einer eingehenden Bewerbung durch das Sekretariat“. 
Grundsätzlich sind die im Prozess verwendeten Reiter gemäß unseren Erfahrungen einige der wichtigsten und am häufigsten verwendeten Dialoge der Anwendung.

Ich habe der Einfachheit halber meine Fragen an Sie unterstrichen formatiert und würde Sie bitten, eine kurze Stellungnahme Ihrer Sicht zu formulieren.

Aufgaben
Nach sorgfältiger und längerer Abwägung möchte ich drei Aufgaben mit den darunter gelisteten (voraussichtlichen) Tätigkeiten untersuchen.

-	Aufgabe 1: Erstellen Sie bitte für diesen Schüler eine Bewerbung
o	Suchen des Schülers im System
o	Auswahl des Bildungsgangs
o	Entscheidung über die Aufnahme
o	Eintragen der persönlichen Daten
o	Eintragen der schulischen Vorbildung
o	… (diverse weitere Tätigkeiten)
-	Aufgabe 2: Bearbeiten Sie bitte diese Bewerbung eines Schülers, dessen Schülerdaten von einer Schule angelegt wurden, und der sich direkt bei Ihnen beworben hat.
o	Ggf. Daten ergänzen
o	Entscheidung über die Aufnahme 
-	Aufgabe 3: Bearbeiten Sie bitte diese Bewerbung eines Schülers, dessen Schülerdaten vom Schüler selbst bzw. den Eltern angelegt wurden, und der sich direkt bei Ihnen beworben hat.
o	Ggf. Daten ergänzen
o	Entscheidung über die Aufnahme

Aufgabe 1 umfasst sämtliche Dialogfelder, die in Aufgabe 2 unter „Daten ergänzen“ potenziell tangiert werden könnten. Diese Aufgabe ist in der Praxis anteilig zu rund 3,6% vertreten.
Aufgabe 2 ist eine kurze Aufgabe, jedoch derzeit anteilig zu 49,2% vertreten.
Aufgabe 3 ist ebenfalls eine kurze Aufgabe und anteilig zu 23,3% vertreten.

Eine Aufgabe, die ich nicht untersuchen werde, jedoch zu 23,7% vertreten ist, ist die Bearbeitung einer Bewerbung, die durch einen vom Betrieb angelegten Schüler vorgenommen wurde. 
Dies könnte man meiner Auffassung nach in einer separaten oder nachgelagerten Forschung untersuchen. 

Ausrüstung
Ich habe Ende letzter Woche informell ein Berufskolleg gefragt, welche Ausgangssituation bei einer Bewerbung vorliegt. Laut deren Aussage erhalten sie ausschließlich ein schriftliches Bewerbungsformular, das sie dann im System einpflegen.
Laut meinem Projektleiter, der umfassende Praxiserfahrung besitzt, ist es jedoch häufig so, dass ein Elternteil dem Sekretariatsmitarbeitenden in Person gegenübersitzt.

Sollte ich 
a)	Der Vergleichbarkeit halber überall dieselbe Ausgangssituation schaffen, indem ich entweder 5 schriftliche Bewerbungsformular von „Schüler/Eltern-Schauspielern“ vorbereiten lasse oder 5 „Schüler/Eltern-Schauspieler“ mit zum Gespräch mitnehme?
b)	Die reelle Ausgangssituation, die mir die jeweilige Schule in einem Vorgespräch  \textit{SchILD} ert, nachstellen?

Die Voraussetzungen der Aufgaben (insbesondere die Stammdaten der Schule) können von meiner Projektleitung und mir „steril“ vorbereitet werden.
Ich gehe davon aus, dass so wie wir es vorbereiten, es der Realität entspricht. Alternativ könnte ich das Sekretariat im Vorherein darum bitten, dies selbstständig vorzubereiten.
Sollen mein Projektleiter und ich das Szenario selbst vorbereiten oder durch das Sekretariat vorbereiten lassen?

Nutzer
Ich habe meinen Projektleiter schonmal darum gebeten ein paar Schulen zu kontaktieren, sodass mir ein angemessener Pool an Interviewkandidaten zur Verfügung steht.
Hierbei möchte ich zwei Kategorien an Sekretariatsmitarbeitern interviewen
-	2 Anwender, die alte Anwendung bereits verwendet haben 
-	3 Anwender, die die alte Anwendung nie bedient haben 

Begleitend hierzu gibt es einen Elternteil oder den sich bewerbenden Schüler. Diese bedienen jedoch nicht die Anwendung, sondern beantworten ggf. Fragen, sofern sie anwesend sind.
Aus meiner Sicht werden diese nicht Teil des Interviews sein, selbst wenn sie vor Ort sind. Entspricht das auch Ihrer Auffassung?

Unter Berücksichtigung, dass ich nur die Effektivität und Zufriedenstellung untersuchen werde, bin ich zuversichtlich, dass der Umfang dieser Forschung den Anspruch einer Projektarbeit weder über- noch unterschreiten wird.
Das gesamte System ist enorm umfangreich, ich habe einen aus meiner Sicht relativ kleinen und wichtigen Ausschnitt mit dem Genannten gewählt.

Falls Sie sich lieber kurz per Zoom mit mir zusammensetzen wollen – ich richte mich vollständig nach Ihnen und stehe zu beliebiger Zeit an jedem Tag zur Verfügung.


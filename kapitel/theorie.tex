\section{Theoretischer Hintergrund}
\subsection{Definition von Effektivität im Rahmen dieser Ausarbeitung }
Die ISO 9241-110 definiert Effektivität wie folgt: \glqq Effektivität = Die Genauigkeit und Vollständigkeit mit der Benutzer ein bestimmtes Ziel erreichen. [...] [Sie] bezeichnet das Ausmaß der Übereinstimmung von tatsächlichen und angestrebten Ergebnissen.\grqq{}\cite{ISO-9241-110}. Ein \glqq Mangel an Effektivität kann zu Ergebnissen führen, die nutzungsbedingte Schäden nach sich ziehen könnten.\grqq{}\cite{ISO-9241-110}
In Anlehnung an diese Definition kann also festgelegt werden, dass Effektivität im Sinne dieser Ausarbeitung das Ausmaß der Genauigkeit und Vollständigkeit mit dem die Studienteilnehmer die drei an sie gestellten Aufgaben erreichen. 

Erfolgskriterien, die darauf hinweisen, dass eine Aufgabe erfolgreich abgeschlossen wurde, sind: 
\begin{itemize}
    \item Die Navigation zur Aufgabe wurde erfolgreich abgeschlossen (EK1)
    \item Die Pflichtangaben wurden vollständig erfasst (EK2)
    \item Der Aufnahmestatus ist entweder \textit{Aufgenommen}, \textit{Abgelehnt} oder \textit{Warteliste} (EK3)
    \item Die Anmeldung wurde erfolgreich gespeichert (EK4)
\end{itemize}

\subsection{Definition von Zufriedenstellung im Rahmen dieser Ausarbeitung}
Die ISO 9241-110 definiert Zufriedenstellung: \glqq Das Ausmaß der Übereinstimmung der physischen, kognitiven und emotionalen Reaktionen des Benutzers, die aus der Benutzung eines Systems, eines Produkts oder einer Dienstleistung resultieren, mit den Benutzererfordernissen und Benutzererwartungen.\grqq{}\cite{ISO-9241-110} Hieran anknüpfend kann man für die vorliegende Studie festlegen, dass Zufriedenstellung das Ausmaß der Übereinstimmung zwischen den aus der Benutzung von \textit{Schüler Online} entstandenen Reaktionen des Schulpersonals und den Benutzererfordernissen und Benutzererwartungen ist. 

\subsection{Definition von Problemstellen im Rahmen dieser Ausarbeitung}
Im Rahmen dieser Arbeit bezieht sich der Begriff \textit{Problemstelle} auf spezifische Elemente der \textit{Schüler Online}-Anwendung, die bei ihrer Nutzung durch das Schulpersonal zu Defiziten in Bezug auf Effektivität und Zufriedenheit führen.

\subsection{Definition von Nutzerbedürfnissen im Rahmen dieser Ausarbeitung}
Stangl beschreibt den Begriff \textit{Bedürfnis} als \glqq das Verlangen oder der Wunsch, einen empfundenen oder tatsächlichen Mangel Abhilfe zu schaffen.\grqq{}\cite{stangl-beduerfnis} Dies wirft die Frage auf, was ein Mangel im Kontext einer Anmeldung an einer Schule für den Anwender bedeutet. Wenn man herkömmliche Anmeldungen mittels eines Papierformulars betrachtet, kann man hier argumentieren, dass mehrere Aspekte mangelhaft sind. Beispielsweise gibt es keine Validierung der Daten hinsichtlich Korrektheit oder Plausibilität.  Datenvalidierung könnte allerdings durch eine gute Kommunikation mit der abgebenden Schule abgesichert werden. Das Einlesen der Daten ist möglicherweise problematisch, da handschriftliches Ausfüllen unleserlich geschrieben sein kann. Eine Übertragung in verwendete Schulsoftware, wie \textit{SchILD} kann nur durch unkomfortables Abtippen erreicht werden.
Eine auf diesen Argumenten basierende modellhafte Definition im Kontext dieser Arbeit kann also lauten: \glqq Nutzerbedürfnisse ist das Verlangen oder der Wunsch, die Datenerfassung der Anmeldung weder unsicher, inkorrekt, unplausibel noch unkomfortabel zu vollziehen\grqq{}.

%\begin{itemize}
%    \item \textbf{Schlechte Navigationsstrukturen:} Wenn Nutzer Schwierigkeiten haben, sich auf einer Website zurechtzufinden, können sie ihre Ziele nicht effektiv erreichen. Eine klare, konsistente und intuitive Navigationsstruktur ist entscheidend.
%    \item \textbf{Nicht erfüllte Erwartungen:} Wenn das Design oder die Funktionalität der Anwendung nicht den Erwartungen der Nutzer entspricht, können diese ihre Ziele nicht effektiv erreichen. Beispielsweise kann eine Schaltfläche, die aussieht, als würde sie eine bestimmte Aktion auslösen, tatsächlich eine ganz andere Aktion auslösen.
%    \item \textbf{Mangel an Feedback:} Nutzer müssen wissen, was passiert, wenn sie eine Aktion ausführen. Wenn eine Anwendung nicht angemessen auf Nutzereingaben reagiert, kann dies zu Frustration und Ineffektivität führen.
%    \item \textbf{Nicht zugängliches Design:} Webanwendungen sollten für alle Benutzer, einschließlich Menschen mit Behinderungen, nutzbar sein. Eine Anwendung, die nicht die Richtlinien für Barrierefreiheit erfüllt, kann für einige Nutzer ineffektiv sein.
%    \item \textbf{Schlechte Leistung:} Langsame Ladezeiten oder technische Probleme können die Effektivität stark beeinträchtigen, da sie Nutzer daran hindern, ihre Ziele in einer angemessenen Zeit zu erreichen.
%    \item \textbf{Komplizierte oder überladene Benutzeroberflächen:} Wenn eine Benutzeroberfläche zu viele Optionen, zu viel Text oder zu viele Bilder enthält, kann dies Benutzer verwirren und ihre Fähigkeit, ihre Ziele effektiv zu erreichen, beeinträchtigen.
%    \item \textbf{Mangel an Suchfunktion oder ineffektive Suchfunktionen:} Eine effektive Suche ist für viele Webanwendungen entscheidend. Wenn Nutzer nicht finden können, was sie suchen, können sie ihre Ziele nicht effektiv erreichen.
%
%    %Bedienfehler
%    %Entspricht Gewohnten Abläufen
%    %Nimmt Arbeit ab
%    %Schwierige Bedienung
%    %Suchen nach: \glqq Common usability problems in web applications\grqq{} 
%\end{itemize}

%evtl. wie sieht das schulgesetz aus oder der Prozess


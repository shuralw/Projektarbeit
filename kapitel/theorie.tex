\section{Theoretischer Hintergrund}
In diesem Kapitel solltest du den theoretischen Hintergrund deines Themas erläutern und den aktuellen Forschungsstand darstellen. Hierbei kannst du auf Literatur und Quellen zurückgreifen.

Definition von Problemstellen im Rahmen dieser Ausarbeitung 
Definition von Nutzerbedürfnissen im Rahmen dieser Ausarbeitung 
Definition von Effektivität im Rahmen dieser Ausarbeitung 
Definition von Zufriedenstellung 
im Rahmen dieser Ausarbeitung 

\subsection{Hawthorne Effekt}
Der Hawthorne-Effekt besagt, dass Personen ihr Handeln verändern, weil sie wissen, dass sie unter Beobachtung stehen. Er kann bei Teilnehmenden an wissenschaftlichen Experimenten vorkommen, deren Verhalten beobachtet wird. So ist ihr Verhalten unnatürlich.

\subsection{ggf. Soziale Erwünschtheit}
"Beim Effekt der sozialen Erwünschtheit verändern Teilnehmende ihr Verhalten oder ihre Antworten bei Fragebogen, um ein positives Bild von sich selbst abzugeben."

\subsection{DECIDE Ansatz}
\subsection{Beobachtung}
\subsection{Hawthorne Effekt}
\subsection{Loud Thinking}
\subsection{Feldtest / Labortest}

evtl. wie sieht das schulgesetz aus oder der Prozess

